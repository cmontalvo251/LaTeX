\newpage

\section{Introduction}

This textbook describes an instrumentation kit used for an
Instrumentation and Experimental Methods course at the undergraduate
level. This textbook has been designed with the student and faculty member in
mind. The kit contains the CircuitPlayground
Express/Bluefruit running CircuitPython and is used to teach
fundamentals of instrumentation and provide a hands-on way of
learning. Engineering is usually taught in a
traditional lecture format, involving theory in the classroom,
homework outside of class, and routine examinations. Progressive forms
of learning such as flipped classrooms and project based learning
(PBL) have created new and fun ways for professors to interact with
students and for students to be more involved in their learning. PBL
provides a student-centered method of teaching and learning by posing
problems for students to solve with the solution of the project being
the primary goal and the theory a secondary goal.

The course begins with simple plotting and moves into data analysis,
calibration and more complex instrumentation techniques 
such as active filtering and aliasing. This course is designed to get
students away from their pen and paper and build something that blinks
and moves as well as learn to process real data that they themselves
acquire. There is no theory in these projects. It is all applied using
the project based learning method. Students will be tasked with
downloading code, building circuitry, taking data all from the ground
up. By the end of this course students will be well versed in the
desktop version of Python while also the variant CircuitPython
designed specifically for microelectronics from Adafruit. After this
course students will be able to understand Instrumentation at the
fundamental level as well as generate code that can be used in future
projects and research to take and analyze data. Python is such a broad
and useful language that it will be very beneficial for any
undergraduate student to learn this language. To the professors using
this textbook, 1 credit hour labs are often hard to work into a
curriculum and “live” demonstrations in the classroom cost time and
money that take away from other faculty duties. I’ve created this kit
and textbook to be completely stand-alone. Students simply need to
purchase the required materials and follow along with the
lessons. These lessons can be picked apart and taught sequentially or
individually on a schedule suited to the learning speed of the
course. The authors hope that whomever reads and learns from this
textbook will walk away with an excitement to tinker, code and build
future projects using microelectronics and programming. The
implementation of this kit and overall teaching method has received
many positive reviews from students and is reflected in anonymous course
evaluations.

Students can learn in many ways with a variety of different
omodalities \cite{Gardner}. As such, instructors can choose how to
present course material and have students develop content-specific
skill sets. College students enter the classroom with existing skills
in multi-disciplinary learning that is practiced at the secondary
school level \cite{Ralph}. College and university professors have
the opportunity to use these existing skills as a foundation for their
own instructional practices. Traditional lectures provide students
with lectures on theory. The instructor then provides examinations to
the students based on that theory. Using this format, an instructor
can focus on mathematical principles, the so-called building blocks of
engineering or provide specific applications to these
models \cite{Learning_Styles}. In a traditional lecture format,
the instructor presents theory to the students with academic problems
rarely encountered in a students future career, building invaluable
theoretical knowledge of concepts that appear disconnected from
practical knowledge applied in the field \cite{Yong}. Historically
in engineering education, there has been tension between theory and
application \cite{PBL1}.  Applications include case studies or
practical engineering problems that emulate future careers in
engineering. This provides students with engineering phenomena that
they can see and hear. They can reason through problems intuitively,
memorize facts using demos, build mathematical models, or build
tangible objects themselves. With these different presentation
methods, students are exposed to engineering phenomena in multiple
ways rather than just on the
whiteboard\cite{Learning_Styles}. Project Based Learning (PBL)
provides an alternative for conventional teaching by facilitating
problem solving for students in a group setting that requires
communication, critical thinking, and creativity. These types of
problems have shown that heightened learning can happen when students
interact with tangible objects that represent the theory they are
presented with in the classroom \cite{PBL_Book}. Rather than
learning an equation for heat transfer along a one-dimensional pipe,
the students can create a pipe with thermocouples and plot data from
multiple sensors along the pipe. This connection to a tangible object
reinforces learning and allows students to form bridges in
comprehension between their theoretical knowledge and practical
application (practical knowledge) of those theories in the
field\cite{TANGIBLEINTERFACES}. Research has shown that a
classroom that creates an integrated curriculum increases student
satisfaction which immediately correlates to satisfactory student
performance and increased graduation
rates \cite{StudentSatisfaction}. This is an example of the
benefits of active student engagement versus passive student
engagement. While the traditional lecture format is vital to students
understandings, it is understood that application of the concepts
covered in lecture are  
required to reach higher levels of learning \cite{Armstrong}.  

%\subsection{Background}
%This should include summary of prior work
Hands-on projects where students interact with tangible objects is not
a new form of teaching. The laboratory environment has been around for
decades. However, typically the classroom environment is separate from
the laboratory environment.  There has been little synergy between the
two learning environments. A Mechanical Engineering degree is likely
to have around three to four labs in various disciplines. In these
courses, students perform an experiment in groups. They take and
analyze data as well as create a report documenting their
findings. Many of these laboratory experiments of course are
choreographed for the student by the instructor. They follow a script
and perform the experiment without building anything themselves. The
students take no ownership over the experiment and there is no
creativity built-in. Over the years, the nature of laboratories has
changed including the lack of clear learning
objectives \cite{Labs}. Furthermore, this laboratory environment
requires a significant amount of instrumentation and hardware to
implement. For example, a shock tube or steam pump can cost tens of
thousands of dollars. The maintenance and up-keep alone is not
practical for smaller institutions. The teaching investment required
to prepare the lab every year and the lab itself (often on the order
of three hours) can be a time consuming task for a tenure track
faculty member who often has a large research load. 

The so-called "Lab at Home" hardware kits are becoming more and more
common in the classroom to ease the burden on the instructor and
institution itself\cite{LabAtHome}. These kits are small enough to be
purchased and shipped to a student for them to perform experiments
remotely rather than in the classroom itself\cite{LabAtHome1}. They
can also be brought to the classroom to be used as a personal demo
aid\cite{LabAtHome2_EE}. In this sense, the take-home lab kit serves
as a bridge between theory based lectures and a laboratory
setting\cite{LabAtHome3}. This allows both instructors and students to
engage with content in a way that promotes enduring understandings and
practical application of theoretical
knowledge\cite{LabAtHome4_EE}. Since the kits can be used remotely,
they can also be used for distance learning courses or other
asynchronous activities as well as during the COVID-19 pandemic.   

Two home kits in particular are directly related to instrumentation
and circuits. Cyganski and Nicoletti\cite{LabAtHome4_EE} for example
created a new curriculum for first year electrical engineering
students by creating live demonstrations for the students. Manijikian
and Simmons\cite{LabAtHome2_EE} however, combined a popular commercial
microcomputer board with their own custom-designed interface board in
a kit that students retain for the duration of the course for both
in-lab and at-home assignments. The take-home kits however were based
on the Motorola 68HC11EVB microcomputer which is programmed in
assembly language. Assembly language is a rather difficult language to
learn at the undergraduate level. Even with such a difficult learning
curve, however, student feedback on their approach was
positive\cite{LabAtHome2_EE}. 

It is clear that using a lab at home kit is useful for the
instrumentation classroom, however the programming language to be used
is a subject of debate in faculty meetings and computing
committees. There are multiple languages in use today of varying
complexities and use cases. There are scripting languages like Python,
Ruby and MATLAB, object oriented languages like Java and C++ and
compilation languages like Fortran and C. Note, this is not an
exhaustive list. A recent study showed that scripting languages
(Python, Ruby, MATLAB) enable writing more concise code while
compilation languages (C, Fortran) create smaller
executables \cite{codetype}. MATLAB is often considered in engineering
given its success in industry and its ability to perform numerical
simulations and plotting with ease. Python, however, has become more
and more popular with numerical toolboxes like numpy and plotting
toolboxes like matplotlib that are free and easy to install in
integrated development environments like Thonny or
Spyder \cite{IDE}. The Tiobe Index of Programming\cite{Tiobe} has the
top 3 programming languages listed as Python, C, and Java. MATLAB
is \#20 on the list. In 2004, Hans Fangohr wrote that MATLAB is much
better suited than C for engineering computing but the best choice in
terms of clarity and functionality is provided by
Python \cite{CodeComparison}. It seems then that it would be more
practical for educators even in engineering to teach the most popular
language. This helps with transferability of skills and has future
implications for a students' career. The scripting aspect of Python
lowers complexity and allows students to learn the language quickly to
apply it as a tool rather than getting stuck memorizing syntax and
compilation rules. The language also comes at no cost to the
students. This is a plus for students already bearing heavy financial
burdens to attend a university which includes tuition and
institutional fees. 

Given the success of other kits in many classes and the popularity of
Python, the University of South Alabama (USA) has implemented such a
kit for Instrumentation \& Experimental Methods (ME 316) using
CircuitPython. CircuitPython is a derivative of Python written for
embedded systems and designed to simplify experimenting and learning
to code \cite{CircuitPython}. The learning objectives for ME316
include: statistics, dynamic response of measurement systems,
operational amplifiers, signal conditioners and fundamentals of
microprocessors. This course could be taught with theory as the main
focus of the lecture. However, this course includes the use of an
instrumentation kit and Project Based Learning methods in addition to
theory in the classroom. Each student taking the course purchases the
kit and downloads a free accompanying project
list \cite{Gitbook}. Every Friday, the students complete one of the 20
projects. The following week they submit a report which includes a
demo of the working project in the form of photos and videos, any
plots generated if they are required to take data, all code used and a
small write-up explaining their findings. The sections that follow
describe the kit in more detail as well as student evaluations who
have taken the course.

Note that, the Adafruit Learn page contains many tutorials on how to operate the
CircuitPlayground\cite{Adafruit}. However, Adafruit sells much
more than just the CPX and thus it is often difficult for anyone to
find the correct tutorial needed for the CPX. A simple search on the
Adafruit Learn System yields 21 results for "servo" just on the first
page with over 48 pages of results. The tutorial needed to run a servo
with CircuitPython is the 10th result and it's for a different
development board. Although the software works with the CPX, it does
not explicitly say so. Due to the complexity of some of these systems,
documentation was made and freely given to the
community\cite{Gitbook}. This documentation was custom built using a
combination of multiple sources across the internet. All software for
the kit is also
on \href{www.github.com/cmontalvo251/Microcontrollers.git}{Github} in
a separate repository. Each chapter in the book contains one project for the
students to complete on their own. A list of chapters and a brief
description of the assignment is shown in the table of
contents. Currently there are 20+ projects for the students to work on.

\section{Course Description}

The course is designed to coincide with lecture content in a standard
engineering instrumentation course. The students are guided through
numerous projects. The specific course learning objectives, which include
statistics, dynamic responses, operational amplifiers, and electronic
circuits both analog and digital are shown below:
\begin{enumerate}[itemsep=-5pt]
\item Apply statistical concepts of error and uncertainty analysis using normal,“t” distribution and the $\chi^2$ distribution
\item Use propagation of error methods to determine the uncertainty of calculated quantities
\item Apply the concepts of harmonic response to predict the response of measuring systems to input signals
\item Apply the fundamentals of operational amplifiers and electronic circuits to design signal conditioning circuits including %amplification and filtering
\item Apply microprocessor fundamentals to gain an understanding of digital circuits used for digital to analog and analog to digital conversion
\end{enumerate}

This course is not going to be a traditional lecture format and that
hands-on projects will be assigned every Friday using the lab at home
kit. The projects every Friday are created to enhance the course
learning objectives as dictated above.  

The course is taught on a 2+1 style where the course meets three days
a week for 50 minutes each on Monday, Wednesday and Friday. On Monday
and Wednesday the class is similar to what would typically be found in
a standard lecture format. The instructor goes through theory and examples on a
white board. On Fridays, the students come to class with their
hardware kits and laptops to perform one of 20+ projects defined in
the tutorials in this textbook. Fridays are nicknamed "Funday Fridays"
to differentiate the format of the class and to get students excited about the
projects. On Fridays, the instructor can walk around the classroom and
assist students on their projects as well as highlight fundamental
concepts. This is done by showing them what they learned on a
whiteboard with their lab at home kit. During the COVID-19 pandemic
when Universities were meeting via Zoom, class time on Friday was used
to assist students on their projects while they showed their circuits via
webcam in a virtual space. Since all students purchased their own kit,
there was no issue with students getting hands-on experience even
during COVID-19.  

It is important to note here that Project Based Learning (PBL) in this
course did not replace existing teaching frameworks; rather, PBL
supplemented existing standard teaching measures by allowing students
to apply theory to practice. The role of connectivism comes from the
online networking tools that were added to the curriculum. The
tutorials and assignments as well as example code are all hosted
online on Github \cite{Github}. The availability of these
tutorials and software opens the students to an ecosystem of free and
open source software. Furthermore, the Adafruit Learning System is
also used occasionally which opens them to an ecosystem of hardware
and software tutorials as well as Adafruit IO (an internet of things
site) and the Adafruit Forums \cite{Adafruit}. The Adafruit Forums
allow the students to comment and communicate with other groups
outside the university that utilize the same hardware kits. If the
instructor cannot find a solution online in a Wiki or Tutorial page, a
student or the instructor can post on the forum and wait for a
response from the larger Adafruit community. 

Students in this course are also required to do a final project
where they must create something new by using the microcontroller in
the kit plus three other electrical components. One of the three
components must be something not originally included in the kit. The
students may also work in a group. PBL typically consists of
ownership, creativity, collaboration and critical thinking. These four
aspects are clearly a big part of this lab at home kit and the final
project. They can work in groups: collaboration. They must make
something new: creativity. They must build the item themselves:
ownership. They must apply fundamental principles of the course:
critical thinking. In addition to PBL aspects, the students are
exposed to the larger network of learning through Github, the Adafruit
Learning system and Forums as well as general Wikis online. This
network of learning is seen in the connectivism framework.



