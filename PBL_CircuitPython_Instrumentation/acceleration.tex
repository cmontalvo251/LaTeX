\newpage

\section{Integrating Acceleration}

\subsection{Parts List}

\begin{enumerate}[itemsep=-5pt]
\item CPX/CPB
\item USB Cable
\item Laptop
\item Some sort of temporary adhesive
\item Automobile
\end{enumerate}

\subsection{Learning Objectives}
\begin{enumerate}[itemsep=-5pt]
\item Taking acceleration data of real systems
\item Numerical Integration 
\item The dangers of integration bias
\end{enumerate}

\subsection{Getting Started}

The code for this lab is to have the CPX log acceleration. So when you’re done with this lab you will hopefully have a data file with 4 columjns of data: time, acceleration x, acceleration y, acceleration z. The code I’m using is the same as the lab on accelerometers (See chatper \ref{s:modules}). I’m using method 1 for datalogging so I’m just having it print to Serial (See chapter \ref{s:daq}).
\begin{figure}[H]
  \begin{center}
    \includegraphics[width=0.8\textwidth]{Figures/accelerometer_code1.png}
  \end{center}
\end{figure}
The code on Github has a sleep of 0.1 seconds but make sure to have
the CPX take data as fast as possible. A sleep of 0.01 is probably
good. You will probably get a lot of data points for this
experiment. Once your code is working, place the CPX on your dashboard
with one of the axes of the accelerometer pointing towards the nose of
your car. Try and place the CPX on as flat a surface as possible. You
can use 3M tape or duct tape or hot glue. Just make sure you don’t
damage your car and make sure the CPX is well anchored to the
dashboard. This way when the car accelerates, the CPX will measure
that acceleration. Note, if you’d like to do this with a bike or some
other motor vehicle that is just fine. Just make sure to take pictures
and videos when you do the experiment. I suggest you do this in a
parking lot for safety reasons. I am not responsible for any damage
done to your vehicle or anyone else because you are doing this
project. Once you have the CPX anchored, accelerate your vehicle to 20
mph (or however fast you are comfortable driving) and then slam on the
brakes. Once your data is logged, plot your acceleration in Python on
your desktop computer. After doing the experiment myself, this is what
my acceleration plot looks like. I had to clip the time series to only
include the part from where I accelerated and decelerated quickly. I
also subtracted the first data point from each accelerometer axis to
zero it out and subtract off the bias. Since I took some data for a
bit before I started moving I could have averaged the first few data
points to obtain the
bias. \href{https://www.youtube.com/watch?v=e4xs9Ky7_YI&feature=youtu.be}{I’ve
done this in a Youtube video if you’re unsure what I mean}. Instead
just to get something working properly I went ahead and just used the
first data point. 
\begin{figure}[H]
  \begin{center}
    \includegraphics[width=\textwidth]{Figures/accelerometer_plots.png}
  \end{center}
\end{figure}
It’s clear from the plots that the z axis was oriented towards the nose of the car. In this case I am going to have to flip the z-axis since the beginning is acceleration and the end is deceleration. I also through the acceleration in the z-axis through a complementary filter with a filter value of 0.25. I think it makes the acceleration profile a bit less jumpy. I then used a Reimann sum and integrated the acceleration data points to get velocity. The equation itself looks like this:
\begin{equation}
V_i = \sum\limits_{n=0}^{i}{(a_i-a_0)\Delta t}
\end{equation}
This of course assumes the initial velocity is zero. Notice that I take the individual acceleration points and subtract off the bias. Computing that summation by hand is pretty trivial but getting the code to work is another story. For a Reimann sum we’re going to use a for loop where we loop through all the data points. The good news is that the time between data points is the same so we can just treat that as a constant. Once you have acceleration integrated you can plot velocity. This is what mine looks like after I did the experiment. According to my plot I accelerated to about 45 mph. I guess I can’t lie in this instance. I said to accelerate to 20 mph but I really wanted to see a large change in acceleration so I punched it. Notice though that at the end of the time series the velocity is negative. This is because as time goes on you are integrating error and the error just gets worse.
\begin{figure}[H]
  \begin{center}
    \includegraphics[width=\textwidth]{Figures/accelerometer_integration.png}
  \end{center}
\end{figure}
This is why speedometers are used. They are just much more accurate
than integrating acceleration which is prone to bias and
drift. \href{https://github.com/cmontalvo251/Microcontrollers/blob/master/Circuit_Playground/CircuitPython/cpx_assignments/Velocity_from_Acceleration/velocity_from_acceleration.py}{This
folder on Github has some codes that will help with your
project}. {\bf Note, that some of those codes have a bias filter,
truncation filter and complementary filter. That code may not work for
you and you may need to tune the filters for your specific data
set. Make sure to understand what each filter does and think about how
it applies to your data set otherwise your code may throw an error due
to the differences in your data set.} 

\subsection{Assignment}

For this assignment you are to find a safe place to accelerate and decelerate your vehicle while recording acceleration data on the Circuit Playground. I also want you to take GPS data using PhyPhox so you can compare data. I suggest using a temporary adhesive to secure your CPX/CPB to your dashboard (making sure it's level) and have a passenger in the car to help you record data as well as operate Phyphox and verify that you have a GPS lock. Again for safety's sake I suggest finding an empty parking lot and only accelerating to 20 mph or less.

\ \\
\noindent {\large {\bf Grading Rubric}}
\ \\
\ \\
\noindent For every project you must turn in a formal properly formatted engineering report submitted as a PDF. The grading rubric is shown below.

\begin{enumerate}[itemsep=-5pt]
\item Title Page (name, title of project and date) - 5\%
\item Introduction (explanation of project, learning objectives and required outcomes) - 5\%
\item Appendices with all code used - 5\%
\item Figures with appropriate figure captions and supporting text - 5\%
\item Professional video (screen/webcam) recording of you and your screen explaining the module and showing any of the systems operate as asked. Create a public Youtube or Google Drive video and include the link in the appendix. - Pass/Fail
\item Project specific requirements (see below) - 80\%
\end{enumerate}

\noindent{\large {\bf Project Specific Requirements}}

\begin{enumerate}[itemsep=-5pt]
\item Include a picture of your car and your passenger that is helping you record data. In your description write what speed you achieved in your experiment. - 10\%
\item Include a photo of your CPX/CPB mounted to your vehicle indicating which axis of acceleration is pointing forward - 10\%
\item Plot one axis of your accelerometer data vs time which clearly indicates when you accelerated and decelerated. In your description be sure to explain which axis you are plotting and any signal conditioners you applied to get your clean signal - 20\%
\item Integrate acceleration and plot velocity as a function of time. Comment on whether or not the maximum velocity is the same as what you did in your actual car. Make sure to superimpose your Phyphox GPS speed on top of your plot to compare. You may need to have your raw GPS coordinates converted to velocity if it doesn't log speed natively. - 20\%
\item Integrate the velocity and compute position. Plot your position as a function of time and include that in your report. Also include the position of your car using your GPS coordinates keeping in mind that you may need to convert your Lat/Lon coordinates to meters. Although you didn't measure how far you went, comment on the accuracy of the plot and whether or not you think you traveled that far especially given the superimposed data from your GPS. - 20\%
\end{enumerate}
