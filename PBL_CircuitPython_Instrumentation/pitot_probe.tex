\newpage

\section{Wind Speed from Pitot Probe}

\subsection{Parts List}

\begin{enumerate}[itemsep=-5pt]
\item Laptop
\item CPX/CPB
\item USB Cable
\item Alligator Clips (x3)
\item Pitot Probe (Not included in kit at the moment so will need to buy this separately or borrow one)
\item Breadboard
\end{enumerate}

\subsection{Learning Objectives}
\begin{enumerate}[itemsep=-5pt]
\item Understand how pitot probes works
\item Understand the relationship between a voltage signal from a pitot probe to a pressure value
\item Understand the relationship between pressure and windspeed
\end{enumerate}

\subsection{Getting Started}

Although a CPX has numerous sensors built in, you can easily augment the capabilities of the CPX using either I2C or just the ADC on board the CPX. In this lab, if you purchased a \href{https://www.amazon.com/Hobbypower-Airspeed-MPXV7002DP-Differential-controller/dp/B00WSFWO36/ref=sr_1_3?dchild=1&keywords=Airspeed+sensor+kit&qid=1590532161&sr=8-3}{pitot probe} you will be able to do this assignment. Since you don’t need the pitot probe for very long you can always borrow one from some other team. Let’s talk about the hardware and the wiring to get this to work. First, pitot probes work by mechanically changing the dynamic pressure of the incoming airflow as shown in the \href{https://makersportal.com/blog/2019/02/06/arduino-pitot-tube-wind-speed-theory-and-experiment}{figure below} (Courtesy of \href{https://makersportal.com/blog?author=59b036fc6073554c1cfffef7}{Joshua Hrisko}).
\begin{figure}[H]
  \begin{center}
    \includegraphics[width=\textwidth]{Figures/pitot_tube_drawing_transparent_cleaner.png}
  \end{center}
\end{figure}
The pitot probe has two pressure taps that measure static (ambient) pressure and dynamic (stagnation) pressure. These taps move through two silicon tubes to a pressure transducer that has a strain gauge that separates both pressures. When the pressure on one side of the transducer is larger than the other, it will flex the membrane and create strain. This strain runs through a wheatstone bridge with a voltage offset to the pin labeled analog. The analog signal from the analog pin will be denoted as $V_{pitot}$. The pressure transducer used in this lab is the MPXV7002 which is a differential pressure sensor. The \href{https://www.nxp.com/docs/en/data-sheet/MPXV7002.pdf}{datasheet} indicates that the voltage offset of the pitot probe is 2.5V and the change in voltage is proportional to the change in pressure in units of kPa. This means that $\Delta P$ is given by the equation below.
\begin{equation}
\Delta P_{kPa} = V_{pitot} - 2.5
\end{equation}
remembering that $V_{pitot}=3.3D_o/65535$. Also note that 2.5V is the nominal voltage even though your sensor might have something slightly different like 2.4V or 2.6V.  I explain the process of substracting off bias in this \href{https://www.youtube.com/watch?v=e4xs9Ky7_YI&feature=youtu.be}{accelerometer video}. To measure wind speed you can use a variation of Bernoulli's principle. Remember to convert the pressure $P$ from kilopascals (kPa) to Pascals (Pa) by multiplying by $1000$ before using it in this formula.
\begin{equation*}
    U = \sqrt{\frac{2 \Delta P_{Pa}}{\rho}}
\end{equation*}
where $U$ is the wind speed in $m/s$, $\Delta P$ is the differential pressure (in Pascals), and $\rho=1.225~kg/m^3$ is the density of air. I've done this pitot project before and have posted a video on Youtube about \href{https://youtu.be/jSLIRC1cfvE}{Converting Pitot Probe Data to Windspeed}. {\bf \it There is a typo in the video. V1 is supposed to have a sqrt())}. As for wiring up the circuit itself, the transducer has 3 pins, +5V, GND and Analog. The figure below shows the transducer connected to an Arduino (Courtesy of \href{https://www.pinterest.com/pin/create/button/?guid=jWEbJ2QP0Vh9&url=https%3A%2F%2Fmakersportal.com%2Fblog%2F2019%2F02%2F06%2Farduino-pitot-tube-wind-speed-theory-and-experiment&media=https%3A%2F%2Fimages.squarespace-cdn.com%2Fcontent%2Fv1%2F59b037304c0dbfb092fbe894%2F1549565434223-TTB9OMIQ3FMP2JLZ0RX2%2FArduino%2BMPXV7002DP%2BWiring%2Bfor%2BDifferential%2BPressure%2BMeasurement&description=Arduino%20MPXV7002DP%20Wiring%20for%20Differential%20Pressure%20Measurement}{Pinterest}).
\begin{figure}[H]
  \begin{center}
    \includegraphics[width=\textwidth]{Figures/Arduino+MPXV7002DP+Wiring+for+Differential+Pressure+Measurement.png}
  \end{center}
\end{figure}
 It is pretty straightforward how to wire this up to a CPX/CPB but remember that +5V needs to go to VOUT, GND to GND and Analog to any analog pin. I chose pin A2 as shown in the Figure below. 
\begin{figure}[H]
  \begin{center}
    \includegraphics[width=\textwidth]{Figures/pitot_probe_circuit.jpeg}
  \end{center}
\end{figure}
At that point it’s very simple to just print the analog signal in bits to Serial. I’ve done this below. The code is the same \href{https://github.com/cmontalvo251/Microcontrollers/blob/master/Circuit_Playground/CircuitPython/Analog/analog_simple.py}{analog code} that we’ve used in the past.
\begin{figure}[H]
  \begin{center}
    \includegraphics[width=0.5\textwidth]{Figures/analogio.png}
  \end{center}
\end{figure}
The goal of the experiment is to take pitot probe data for 15 seconds with no wind, then 15 seconds of data with a fan on and then 15 seconds of no wind data. You’ll need to use one of the datalogging methods (See chapter \ref{s:daq}) to log both time and pitot probe analog value. Once you have that data, import the data into Python on your desktop computer and convert the signal to windspeed as explained above. Using your data, create a plot of windspeed with time on the x-axis and windspeed on the y-axis. Some steps that might help you as you complete this project. First, have Mu plot the voltage coming from the pitot probe. If you’ve done everything right it will not be zero. The data sheet says there’s an offset voltage of 2.5V so you will hopefully get something around 50,000 when you don’t blow into the pitot probe. 50,000 multiplied by 3.3/$2^{16}$ is around 2.5V. Make a note of that average value you get so you can subtract it off later. Once you’ve verified you’re reading the pitot probe correctly, blow into the pitot probe and using the Plotter or Serial, verify that the analog signal increases. If the signal decreases, it means the pressure taps on the pressure transducer are backwards and you need to flip them. Either that or just flip the sign in your plotting routine on your computer but flipping the tubes might be easier for you. Hopefully when you do this lab you will get some data that looks like this. In this Figure you’ll see that when the fan wasn’t running the signal was something around 49,800 which is fine. It means your bias is around 2.5 volts. Every pitot probe and circuit will be different. You can then convert this signal to voltage then and then pressure and then finally wind speed.
\begin{figure}[H]
  \begin{center}
    \includegraphics[width=\textwidth]{Figures/pitot_probe_data.png}
  \end{center}
\end{figure}
The code to accomplish this is relatively simple and a portion of the code is shown below. You’ll see that when I subtracted the bias from the voltage I also zeroed out any negative values. That is, any delta voltage less than zero was set to zero. A couple of things about this chart. The data from the pitot probe is super noisy which means attaching a \href{https://youtu.be/zTGa4sk6UZE}{complementary filter} is probably a good idea provided you don’t over filter the signal and run into \href{https://youtu.be/8F_8st_8MmA}{aliasing issues}. You can see that I implemented an offline complementary filter and plotted it in the orange line which helps the noise issue quite a bit. You’ll also notice that the noise is about 2 m/s. It turns out that pitot probes are actually not very accurate lower than about 2 m/s. They would be great for an airplane or you driving down the highway but they wouldn’t be very good to take wind data outside on a calm day.
\begin{figure}[H]
  \begin{center}
    \includegraphics[width=\textwidth]{Figures/pitot_probe_final.png}
  \end{center}
\end{figure}

\subsection{Assignment}

For this assignment you are to wire up a pitot probe and record time and raw analog signal from the pitot probe as you turn a fan on and off. My suggestion is you record at least 30 seconds of data with the fan off and then 30 seconds with the fan on and then again 30 seconds with the fan off. Specific requirements are shown below.

\ \\
\noindent {\large {\bf Grading Rubric}}
\ \\
\ \\
\noindent For every project you must turn in a formal properly formatted engineering report submitted as a PDF. The grading rubric is shown below.

\begin{enumerate}[itemsep=-5pt]
\item Title Page (name, title of project and date) - 5\%
\item Introduction (explanation of project, learning objectives and required outcomes) - 5\%
\item Appendices with all code used - 5\%
\item Figures with appropriate figure captions and supporting text - 5\%
\item Professional video (screen/webcam) recording of you and your screen explaining the module and showing any of the systems operate as asked. Create a public Youtube or Google Drive video and include the link in the appendix. - Pass/Fail
\item Project specific requirements (see below) - 80\%
\end{enumerate}

\noindent{\large {\bf Project Specific Requirements}}

\begin{enumerate}[itemsep=-5pt]
\item In addition to the standard format above, you must also return the pitot probe you borrowed in class - Pass/Fail
\item Include a photo of your circuit with your pitot probe wired up to an analog pin - 10\%
\item Include a screenshot of Mu with the Plotter open showing the raw analog signal. The Mu code also needs to show the same analog pin as the circuit above - 10\%
\item Include a plot of the raw analog signal vs time that clearly shows when the fan is on and off - 20\%
\item Convert your raw signal to windspeed and filter your signal using an offline complementary filter. Plot both unfiltered and filtered windspeed on the same plot and include a legend. Make sure you add the saturation filter to prevent a negative in the square root - 20\%
\item What is the maximum windspeed that the CPX/CPB can read? - 10\%
\item The data sheet also suggests that you use a $470~pF$ capacitor to filter the output. Select two resistors for a low pass filter such that the DC Gain would be equal to 1. Assume the total impedance of both resistors is $2~k\Omega$. Also compute the cutoff frequency. - 10\%
\end{enumerate}
