\section{Changes Needed}

\begin{enumerate}[itemsep=-5pt]
\item For each assignment incorporate an actual report (intro with explanation fo objectives and learning outcomes.
\item Code in the appendix, reference section with links to videos.
\item So make title page 5\%, intro 5\%, appendices with code 5\% and formattign (figues with appropriate captions 5\%) so the content is 80\%. Also make the videos pass/fail. If I can't tell if you did it or not how do I know you didn't just copy it?
\item For the servo project report the min and max in a table. Report your data in a table.
\item For the servo assignment have them report the pulse required for 90 degrees.
\item For the same assignment have them compute the r2 value
\item Add a saturation filter so that your regression equation doesn't output a pulse outside of your min and max - servo assignment
\item Figures of results with properly labeled figures and explanatory text.f
\item Make them also always incldue a photo of their circuit with an explanation of the wiring.
\item So in that case remove the explanation of the circuit from the videos and have them do it in text.
\item The video is just to prove the circuit actually works.
\item So they don't need to say their name or say high in the video they just need to prove it works.
\item So change videos to ``Video proof that the circuit works. Show the cirtuit and then show it's functionality"
\item Add homework problems from the instrumentation textbook into the projects here. 
\item Create circuit diagrams for each lesson.
\item More theory is needed in this book or direction to further
  reading for the students
\item Create a lesson on how each sensor works. The thermistor and pitot probe come to mind 
\item The photo of the button lab could be better (i.e add labels to
  each item)
\item For the videos make sure that I can see the system, you and your code or no credit - and then make the videos maybe 10\%?
\item I want at least 10 data points for the servo problem
\item When plotting your regression in servo lab you need to request for a scatter plot for the calibration data and then a line for the regression
\item Consider removing parts 1 and 2 and just making it a more challenging lab or perhaps sub assignments like Assignment 1.1 and 1.2 (Technically parts 1 and 2 but it just sounds more professional) at the same time though if you make every weekly project a report you can just have some easy and some hard ones.
\item The circuit photo for the LSM6DS33 is not correct
\item Equations on thermistor need to be expanded
\item Equations relating voltage from photocell to Lux needs to be included
\item Example plots for light, sound, acceleration, etc needs to be expanded
\item Pendulum lab must be done in one of two ways. Either the pendulum is attached to a potentiometer or the CPX is mounted to the end of a string and data logged on board the CPX itself. The potentiometer is nice because you can record data with your laptop but the string idea is cool because you use the accelerometer. In either case you can make some really long pendulums
\item 3D printing a disc with holes on the outside to eventually mount to a shaft would be a really cool angular velocity sensor lab. Tangibles that teach could easily include a 3D printed disc that can mount to a pencil for ease of rotation. Could also include the CAD drawings so students can print more or even edit the design for better or worse performance.
\item Buying some load cells with the HMC converter and including them in the kit would add a whole lot different labs
\item Buying some magnetometers to measure magnetic field and do some sensor fusion would be neat. Could do roll, pitch and yaw calibration if we included a magnetometer.
\item Right now a lab just on roll and pitch estimation would be possible. Pendulum lab pretty much introduces them to this but could easily do an rc aircraft lab where they build an aircraft out of foam with an elevator and aileron so that the servos responds to roll and pitch change. There is a lab right now with just pitch but perhaps we could add roll to it.
\item Another cool project idea would be for the students to take temperature and light data on a cloudy day. Then have them infer if the amount of sunlight affected the temperature of the thermistor. They could plot the data with light on the x-axis and temperature on the y-axis and draw conclusions based on the plot they generate.
\item Could also have them take temperature and light data over the course of a whole day to plot sunrise and sunset and watch the ambient temperature rise and fall
\item For the aliasing lab, have the students sample as fast as possible and obtain the natural frequency of the system. Then have them sample at 1.0, 2.0 and 3.0 times the natural frequency they obtained. I originally picked 1,10 and 100 as arbitrary sampling frequencies and it would have been better to do 2,4 and 6.
\item One cool lab would be to take light data during sunset and watch light and temperature plummet.
\item \href{https://learn.adafruit.com/circuit-playground-o-phonor/musical-note-basics}{Add this lab on frequency for notes}
\end{enumerate}
