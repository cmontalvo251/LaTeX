\newpage

\section*{Acknowledgements}

The author, Dr. Carlos Montalvo would like to acknowledge a few key
members who made this textbook possible. First and foremost I would
like to thank Adafruit for their entire ecosystem of electronics,
tutorials, blogs and forums. Much of what I have learned here to teach
Instrumentation was from \href{https://www.adafruit.com/}{Adafruit}
and the \href{https://learn.adafruit.com/}{Adafruit Learn} system and 
specifically people like Lady Ada and John Park who have helped shape
\href{https://circuitpython.org/}{CircuitPython} and the
\href{https://www.adafruit.com/product/3333}{CircuitPlayground
  Express} to what it is today. I 
would also like to thank Dr. Saami Yazdani for creating the blueprint
for Instrumentation at my university by creating a laboratory
environment for an otherwise totally theoretical course. His course
was the foundation for this textbook and for that I thank him for
showing the way. I’d like to also thank and acknowledge Tangibles that
Teach for giving me the opportunity to morph this loose set of
projects into a textbook that can be used for multiple universities
and classrooms and of course help students learn and acquire knowledge
through creating.

\section*{About this textbook}

This textbook has been designed with the student and faculty member in
mind. First, this textbook goes hand in hand with Engineering
Instrumentation taught at the undergraduate level at many
universities. The course begins with simple plotting and moves into
data analysis, calibration and more complex instrumentation techniques
such as active filtering and aliasing. This course is designed to get
students away from their pen and paper and build something that blinks
and moves as well as learn to process real data that they themselves
acquire. There is no theory in these projects. It is all applied using
the project based learning method. Students will be tasked with
downloading code, building circuitry, taking data all from the ground
up. By the end of this course students will be well versed in the
desktop version of Python while also the variant CircuitPython
designed specifically for microelectronics from Adafruit. After this
course students will be able to understand Instrumentation at the
fundamental level as well as generate code that can be used in future
projects and research to take and analyze data. Python is such a broad
and useful language that it will be very beneficial for any
undergraduate student to learn this language. To the professors using
this textbook, 1 credit hour labs are often hard to work into a
curriculum and “live” demonstrations in the classroom cost time and
money that take away from other faculty duties. I’ve created this kit
and textbook to be completely stand-alone. Students simply need to
purchase the required materials and follow along with the
lessons. These lessons can be picked apart and taught sequentially or
individually on a schedule suited to the learning speed of the
course. I hope whomever reads and learns from this textbook will walk
away with an excitement to tinker, code and build future projects
using microelectronics and programming.

\newpage

\tableofcontents

