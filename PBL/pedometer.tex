\newpage

\section{Building a Pedometer using an Accelerometer}
\label{s:pendulum}

\subsection{Parts List}

\begin{enumerate}[itemsep=-5pt]
\item Laptop
\item CPX/CPB
\item USB Cable
\item External Battery Pack
\end{enumerate}

\subsection{Learning Objectives}
\begin{enumerate}[itemsep=-5pt]
\item Understand how to run CPB/CPX while not tethered to a computer
\item Reinforce bluetooth tech for data transfer
\item Understand post-processing for debugging to be used for online calculations
\item Understand the fundamentals of how a pedometer works
\end{enumerate}

\subsection{Getting Started}

A pedometer is a device that....

\subsection{Gathering Accelerometer Data}

\subsection{Computing Number of Steps: Post-Processing}

\subsection{Computing Number of Steps: Online}

\subsection{Assignment}

Upload a PDF with all of the photos and text below included. My recommendation is for you to create a Word document and insert all the photos and text into the document. Then export the Word document to a PDF. For videos I suggest uploading the videos to Google Drive, turn on link sharing and include a link in your PDF.

\begin{enumerate}[itemsep=-5pt]
\item Include a video of you gathering accelerometer data via bluetooth - 30\%
\item Include a video of your partner running down the hallway. How many steps did they take? - 30\%
\item Include a plot of accelerometer data and how many steps your partner took according to the accelerometer data - 30\%
\item Compare the results from the accelerometer data and the actual number of steps in the video. Are they the same? Different? Why or why not? - 10\%
\end{enumerate}