\newpage

\section{Building a Pedometer using an Accelerometer}
\label{s:pendulum}

\subsection{Parts List}

\begin{enumerate}[itemsep=-5pt]
\item Laptop
\item CPX/CPB
\item USB Cable
\item External Battery Pack
\end{enumerate}

\subsection{Learning Objectives}
\begin{enumerate}[itemsep=-5pt]
\item Understand how to run CPB/CPX while not tethered to a computer
\item Reinforce bluetooth tech for data transfer
\item Understand post-processing for debugging to be used for online calculations
\item Understand the fundamentals of how a pedometer works
\end{enumerate}

\subsection{Getting Started}

A pedometer is a device that counts the number of steps. Typically
these are worn as watches with popular brands like Garmin or Fitit
owning the market share at the time of this writing. It turns out
though that your phone and apps like Google Fit can also track steps
just by being inside your pocket all day. The way they do this is by
measuring the acceleration, angular velocity and potentially even the
angles (using a magnetometer) to count steps. In this lab this we will
just focus on attempting to get steps using accelerometer data.  

\subsection{Gathering Accelerometer Data}

First we need to make sure we can gather accelerometer data. The low
level accelerometer code is relatively simple and is explained in the
Modules lab (Section \ref{s:Modules}). In order to gather the
accelerometer data while running you'll need to be able to operate the
CPB/CPX untethered from a computer. This means you have to use Method
3 or 4 (Section \ref{s:daq} and \ref{s:Bluetooth}). Remember that
Method 1 and 2 require a computer and running with a computer would be
difficult. Method 3 requires a lot of setup to log data directly to
the disk so for this lab we will just use the Bluetooth module to send
data directly to your phone. The best way is to do this with a
partner. Have the CPX/CPB measure acceleration and place the entire
device with a battery pack inside the runners pocket. Then have your
partner connect to the CPB/CPX with the Adafruit Connect app and log
data using the UART and Export to txt function. Remember not to run
too far because the Bluetooth signal distance is only about 30
feet. See if you can combine the Bluetooth code and the acceleration
code into one code to send time and the 3-axis accelerometer data. If
you're still having trouble, code for this lab can be found
on \href{https://github.com/cmontalvo251/Microcontrollers/tree/master/Circuit_Playground/CircuitPython/pedometer}{Github}.  

\subsection{Computing Number of Steps: Post-Processing}

\subsection{Computing Number of Steps: Online}

\subsection{Assignment}

Upload a PDF with all of the photos and text below included. My recommendation is for you to create a Word document and insert all the photos and text into the document. Then export the Word document to a PDF. For videos I suggest uploading the videos to Google Drive, turn on link sharing and include a link in your PDF.

\begin{enumerate}[itemsep=-5pt]
\item Include a video of you gathering accelerometer data via bluetooth - 30\%
\item Include a video of your partner running down the hallway. How many steps did they take? - 30\%
\item Include a plot of accelerometer data and how many steps your partner took according to the accelerometer data - 30\%
\item Compare the results from the accelerometer data and the actual number of steps in the video. Are they the same? Different? Why or why not? - 10\%
\end{enumerate}
