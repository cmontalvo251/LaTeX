% \iffalse
%% File: soul.dtx  Copyright (C) 1998--2003  Melchior FRANZ
%% $Id: soul.dtx,v 1.128 2003/11/17 22:57:24 m Rel $
%% $Version: 2.4 $
%
%<*batchfile>
%
% on Unix/Linux just run "make" to get the style file and the documentation,
% else generate the driver soul.ins (if you don't already have it):
%
%     $ latex soul.dtx
%
% You'll probably get an error message that you may ignore. Now generate
% the style file:
%
%     $ tex soul.ins
%
% And finally to produce the documentation run LaTeX three times:
%
%     $ latex soul.dtx
%
%
%
%
% DISCLAIMER: note that a Makefile could actually contain malign commands
% that erase your whole account, so having a look at it before won't hurt!
% I take no responsibility for any damage, but I do what I can to make
% using the original Makefile safe.
%
% COPYRIGHT NOTICE:
% This package is free software that can be redistributed and/or modified
% under the terms of the LaTeX Project Public License as specified
% in the file macros/latex/base/lppl.txt on any CTAN archive server,
% such as ftp://ftp.dante.de/.
%
%$
%% ====================================================================
%%  @LaTeX-package-file{
%%     author          = "Melchior FRANZ",
%%     version         = "2.4",
%%     date            = "17 November 2003",
%%     filename        = "soul.dtx",
%%     address         = "Melchior FRANZ
%%                        Rieder Hauptstrasse 52
%%                        A-5212 SCHNEEGATTERN
%%                        AUSTRIA",
%%     telephone       = "++43 7746 3109",
%%     URL             = "http://www.unet.univie.ac.at/~a8603365/",
%%     email           = "a8603365@unet.univie.ac.at",
%%     codetable       = "ISO/ASCII",
%%     keywords        = "spacing out, letterspacing, underlining, striking out,
%%                        highlighting",
%%     supported       = "yes",
%%     docstring       = "This article describes the `soul' package, which
%%                        provides hyphenatable letterspacing (spacing out),
%%                        underlining, and some derivatives.
%%                        All features are based upon a common mechanism
%%                        that allows to typeset text syllable by syllable,
%%                        where the excellent TeX hyphenation algorithm is
%%                        used to find the proper hyphenation points.
%%                        Two examples show how to use the provided interface to
%%                        implement things such as `an-a-lyz-ing syl-la-bles'.
%%                        Although the package is optimized for LaTeX2e,
%%                        it works with Plain TeX and with other
%%                        packages, too.",
%%  }
%% ====================================================================
%
%
%
%
%
\begin{filecontents}{soul.ins}
\def\batchfile{soul.ins}
\input docstrip.tex
\askforoverwritefalse
\keepsilent   % <-- this is for you, Christina   ;-)
\generate{\file{soul.sty}{\from{soul.dtx}{package}}}
\endbatchfile
\end{filecontents}
%</batchfile>
%
%
%
%
%
%<*driver>
\def\fileversion{2.4}
\def\filedate{2003/11/17}
%
%
%
\documentclass{ltxdoc}
%
%
%
\makeatletter\let\SOUL@sethyphenchar\relax\makeatother
\IfFileExists{soul.sty}{%
    \usepackage{soul}[2002/01/10]
    \expandafter\ifx\csname sloppyword\endcsname\relax  % old version?
        \def\sloppyword{\textbf{?? [obsolete soul version]}}
    \fi
    \let\SOULSTYfound\active
}{%
    \GenericWarning{soul.dtx}{%
        Package file `soul.sty' couldn't be found. You should however find^^J^^A
        a file `soul.ins' in the current directory. If you type "tex soul.ins"^^J^^A
        on the command line, `soul.sty' will be generated. After that
        run "latex soul.dtx" again and you won't see this message again.
    }%
}%
%
%
%
\ifx\makehyperref\SOULundefined
    \newcommand*\texorpdfstring[2]{#1}
\else   ^^A for "make hyper"
    \usepackage{hyperref}
    \hypersetup{
        bookmarksopen,
        colorlinks,
        pdftitle={The soul package},
        pdfauthor={Melchior FRANZ},
        pdfsubject={${}$Id: soul.dtx,v 1.128 2003/11/17 22:57:24 m Rel ${}$},
        pdfkeywords={space out, letterspacing, underlining, overstriking, highlighting}
    }
    \usepackage[pdftex]{graphicx,color}
\fi
%
%
%
%\RecordChanges
%
\begin{document}
\setcounter{tocdepth}{2}
\DocInput{soul.dtx}
\end{document}
%</driver>
% \fi
%
%
%
%
%
%
%
% \CheckSum{2006}
% \CharacterTable
%  {Upper-case    \A\B\C\D\E\F\G\H\I\J\K\L\M\N\O\P\Q\R\S\T\U\V\W\X\Y\Z
%   Lower-case    \a\b\c\d\e\f\g\h\i\j\k\l\m\n\o\p\q\r\s\t\u\v\w\x\y\z
%   Digits        \0\1\2\3\4\5\6\7\8\9
%   Exclamation   \!     Double quote  \"     Hash (number) \#
%   Dollar        \$     Percent       \%     Ampersand     \&
%   Acute accent  \'     Left paren    \(     Right paren   \)
%   Asterisk      \*     Plus          \+     Comma         \,
%   Minus         \-     Point         \.     Solidus       \/
%   Colon         \:     Semicolon     \;     Less than     \<
%   Equals        \=     Greater than  \>     Question mark \?
%   Commercial at \@     Left bracket  \[     Backslash     \\
%   Right bracket \]     Circumflex    \^     Underscore    \_
%   Grave accent  \`     Left brace    \{     Vertical bar  \|
%   Right brace   \}     Tilde         \~}
%
%
%
% \title{The \texttt{soul} package}
%
% \author{Melchior \caps{FRANZ}}
%
% \date{November 17, 2003}
%
%^^A=====================================================
%
%^^A  These messy macros allow to typeset the `example' sections
%^^A  conveniently. You'd better ignore them. I do!   :-)
%
% \makeatletter
%
% \def\squarebull{\rule[.2ex]{.8ex}{.8ex}}
%
% \newenvironment{examples}
%   {\parindent\z@\small\fontencoding{OT1}\selectfont}
%   {\rule{\hsize}{.4pt}}
%
% \def\soultest#1|{\bgroup\rule[.5ex]{\hsize}{.4pt}\par
%   \parbox[t]{.34\hsize}{\raggedright\textit{#1\unskip.}}%
%   \hspace{1.5em}$\vtop\bgroup\hb@xt@.4\hsize\bgroup
%   \llap{\squarebull\hspace{.4em}}\soulttest}
%
% {\catcode`\|=13\gdef\soulttest{%
%   \bgroup\def\do##1{\catcode`##1=12}\dospecials\ttfamily
%   \catcode`\|=13\unskip\def|{\hss\egroup\egroup\soultttest}}}
%
% \def\soultttest#1{\hbox{\vtop{\hsize.4\hsize\hbadness\@M
%   \leavevmode\llap{\squarebull\hspace{.4em}}#1\null}}%
%   \egroup$\hspace{1.5em}\parbox[t]{1mm}{\hyphenpenalty-\@M
%   \exhyphenpenalty-\@M\hbadness\@M\hfuzz\maxdimen
%   \strut\llap{\squarebull\hspace{.4em}}#1\null}%
%   \goodbreak\vspace{2ex}
%   \egroup}
%
% \newcommand*\DescribeOption[1]{\marginpar{\raggedleft\textsf{#1}\ignorespaces}}
%
%
%^^A  similar to the `description' environment
%
% \def\describemacro{^^A
%   \bgroup
%   \let\do\@makeother
%   \dospecials
%   \catcode`{=1
%   \catcode`}=2
%   \SOUL@@@descmacro
% }
%
% \def\SOUL@@@descmacro#1{^^A
%   \texttt{#1}^^A
%   \DescribeMacro{#1}^^A
%   \expandafter\edef\expandafter\x\expandafter{\expandafter\@gobble#1}^^A
%   \expandafter\label{cmd:\x}^^A
%   \egroup
% }
%
%
% \def\SOUL@@@verbitem[#1: {^^A
%   \bgroup
%   \let\do\@makeother
%   \dospecials
%   \SOUL@@@verbscan{#1}^^A
% }
%
% \def\SOUL@@@verbscan#1#2]{^^A
%   \egroup
%   \goodbreak
%   \def\@currentlabel{\S\,\the\SOUL@@@itemnr}^^A
%   \label{par:#1}^^A
%   \SOUL@@@item[\textbf{\@currentlabel\hskip.6em#1:}]\hfil\break
%   Example: \texttt{#2}\hfil\break^^A
%   \advance\SOUL@@@itemnr1
% }
%
% \let\SOUL@@@item\item
% \newcount\SOUL@@@itemnr
%
% \newenvironment{verblist}[1]{^^A
%   \SOUL@@@itemnr=#1
%   \list{}{^^A
%       \settowidth{\labelwidth}{\indent\indent}^^A
%       \leftmargin\labelwidth
%       \advance\leftmargin\labelsep
%       \def\makelabel##1{##1}^^A
%       \let\item\SOUL@@@verbitem
%   }^^A
% }{^^A
%   \endlist
% }
%
% \newenvironment{labeling}[1]{^^A
%   \list{}{^^A
%       \settowidth{\labelwidth}{\textbf{#1}}^^A
%       \leftmargin\labelwidth\advance\leftmargin\labelsep
%       \def\makelabel##1{\textbf{##1}\hfil}^^A
%   }^^A
% }{^^A
%   \endlist
% }
%
% \newenvironment{syntax}{^^A
%   \par\medskip\def\<##1>{$\langle$\syn{##1}$\rangle$}^^A
%       \indent\begin{tabular}{l}^^A
%   }{^^A
%       \end{tabular}^^A
%       \noindentafter\medbreak
%   }
%
%
% \newenvironment{example}[1][.9\textwidth]
%   {\par\medskip\indent\begin{tabular}{p{#1}l}}
%   {\end{tabular}\noindentafter\medbreak}
%
% \newcommand*\noindentafter{^^A
%   \global\everypar{{\setbox\z@\lastbox}\everypar{}}}
%
% \newcommand*\errsquare{\leavevmode\vrule height.8em depth.2em width1em\relax}
%
%
% \ifx\SOULSTYfound\active
%^^A  analyze syllables---described somewhere below
%
% \DeclareRobustCommand*\sy{^^A
%   \SOUL@setup
%   \def\SOUL@preamble{\lefthyphenmin0\righthyphenmin0}^^A
%   \def\SOUL@everysyllable{\the\SOUL@syllable}^^A
%   \def\SOUL@everyspace##1{##1\space}^^A
%   \def\SOUL@everyhyphen{^^A
%       \discretionary{^^A
%           \SOUL@setkern\SOUL@hyphkern
%           \SOUL@sethyphenchar
%       }{}{^^A
%           \hbox{\kern1pt$\cdot$}^^A
%       }^^A
%   }^^A
%   \def\SOUL@everyexhyphen##1{^^A
%       \SOUL@setkern\SOUL@hyphkern
%       \hbox{##1}^^A
%       \discretionary{}{}{^^A
%           \SOUL@setkern\SOUL@charkern
%       }^^A
%   }^^A
%   \SOUL@}
%
%^^A  analyze the engine---described somewhere below, too
%
% \DeclareRobustCommand*\an{^^A
%   \def\SOUL@preamble{$^{^P}$}^^A
%   \def\SOUL@everyspace##1{##1\texttt{\char`\ }}^^A
%   \def\SOUL@postamble{$^{^E}$}^^A
%   \def\SOUL@everyhyphen{$^{^-}$}^^A
%   \def\SOUL@everyexhyphen##1{##1$^{^=}$}^^A
%   \def\SOUL@everysyllable{$^{^S}$}^^A
%   \def\SOUL@everytoken{\the\SOUL@token$^{^T}$}^^A
%   \def\SOUL@everylowerthan{$^{^L}$}^^A
%   \SOUL@}
%
%^^A  magazine-like (truly awful) paragraphs
%^^A  If you know what you're doing, you can copy it to your personal `soul.cfg' file.
%
%  \DeclareRobustCommand*\magstylepar{\SOUL@sosetup
%    \def\SOUL@preamble{^^A
%      \edef\SOUL@soletterskip{\z@\@plus\fontdimen2\font\@minus\z@}^^A
%      \edef\SOUL@soinnerskip{\the\fontdimen2\font
%        \@plus\the\fontdimen3\font\@minus\the\fontdimen4\font}^^A
%      \let\SOUL@soouterskip\SOUL@soinnerskip
%      \SOUL@sopreamble}^^A
%    \lefthyphenmin2\righthyphenmin2\SOUL@}
%
%\else
%
%^^A  The package has not been found, so we're providing some dummies, just
%^^A  to avoid all these `Undefined command sequence' messages.
%
%   \def\SOUL@@@X#1{\textbf{??}}%
%   \let\so\SOUL@@@X
%   \let\textso\SOUL@@@X
%   \let\caps\SOUL@@@X
%   \let\textcaps\SOUL@@@X
%   \let\ul\SOUL@@@X
%   \let\textul\SOUL@@@X
%   \let\st\SOUL@@@X
%   \let\textst\SOUL@@@X
%   \let\hl\SOUL@@@X
%   \let\texthl\SOUL@@@X
%   \let\sy\SOUL@@@X
%   \let\an\SOUL@@@X
%   \let\magstylepar\SOUL@@@X
%   \let\sloppyword\SOUL@@@X
%   \def\sodef#1#2#3#4{\let#1\relax\SOUL@@@X}%
% \fi
%
% \newcommand*\xpath{^^A
%   \bgroup
%   \let\do\@makeother
%   \dospecials
%   \catcode`\{=1
%   \catcode`\}=2
%   \def\{{\char"7B}^^A
%   \def\}{\char"7D}^^A
%   \SOUL@@@code
% }
%
% \newcommand*\SOUL@@@code[1]{\normalfont\texttt{#1}\egroup}
%
% \let\cs\xpath
% \let\option\textsf
% \def\package#1{{\normalfont\texttt{#1}}}
% \let\program\texttt
% \let\bibtitle\textit
% \let\syn\textit
%
% \sodef\person{\scshape}{0.06em}{0.45em}{0.4em}
% \sodef\SOUL@@@versal{\upshape}{0.125em}{0.4583em}{0.5833em}
% \DeclareRobustCommand*\versal[1]{^^A
%   \MakeUppercase{\SOUL@@@versal{#1}}^^A
% }
%
% \def\ConTeXt{Con\TeX t}
% \def\CTAN{{\small\caps{CTAN}}}
% \def\soul{\package{soul}}
%
% ^^A  set some parameters as used in Plain TeX
% \def\plainsetup{^^A
%   \pretolerance100
%   \tolerance200
%   \hbadness1000
%   \linepenalty10
%   \hyphenpenalty50
%   \exhyphenpenalty50
%   \doublehyphendemerits10000
%   \finalhyphendemerits5000
%   \adjdemerits10000
%   \hfuzz.1pt
%   \overfullrule5pt
% }
%
% \def\FIXME#1{\message{<FIXME>}#1}
%
% \makeatother
%
%
% \lefthyphenmin2
% \righthyphenmin3
% \hyphenation{Le-se-ty-po-gra-phie Ver-bin-dung fak-si-mi-le}
%
%
%^^A=====================================================
%
%
% \changes{v1.0}{1998/10/16}{Initial version}%
%^^A  due to an error in the documentation of v1.0, there's no v1.1    :-(
% \changes{v1.1a}{1998/12/08}{fixed a bunch of bugs; `magstylepar command
%   banned; `caps introduced; `so and `caps recognize following spaces;
%   error message added; `so parameters are mandatory}
%
% \changes{v1.2}{1999/01/11}{fixed the newline bug; added the `\(>\) command}
%
% \changes{v1.3}{1999/05/15}{changed allowhyphen, preambles; added a paragraph
%   in the `features' section}
%
% \changes{v2.0}{2002/01/03}{(almost) complete rewrite}
%
% \changes{v2.1}{2002/01/10}{Happy (64th) birthday, Don!
%   ``The now-traditional fix for the
%   now-traditional brown-paper-bag major release.''
%   as Eric S. RAYMOND commented on his CML2.0.1 release. ;-)}
%
% \changes{v2.2}{2002/05/12}{fixed a couple of bugs; added support for
%   the Plain TeX color.sty wrapper}
%
% \changes{v2.3}{2002/05/29}{compatibility with cmbright/ccfonts}
%
% \changes{v2.4}{2003/11/17}{Fix the font bug. Fix a scanner bug.
%   Other bugfixes; change caps set handling; add footnote and
%   textsuperscript support}
%
%
%
%
%
% \maketitle
%
%
%
% \begin{abstract}
% This article describes the \soul\ package^^A
%^^A%%
%   \footnote{This file has version number \fileversion, last revised \filedate.},
%^^A%%
% which provides \so{hyphenatable letterspacing (spacing out),} \ul{underlining}
% and some derivatives such as \st{overstriking} and highlighting.
% Although the package is optimized for \LaTeXe, it also works with
% Plain \TeX\ and with other flavors of \TeX\ like, for instance, \ConTeXt.
% By the way, the package name |soul| is only a combination
% of the two macro names \cs{\so} (space out) and \cs{\ul}
% (underline)---nothing poetic at all.^^A  :-(
% \end{abstract}
%
%
% {\setlength\parskip{0pt}\tableofcontents }
% \addtocontents{toc}{\protect\begin{multicols}{2}}
%
%
%
%
%
%
%
%
%
% \section{Typesetting rules}
% \label{sec:typesetting}
%
% There are several possibilities to emphasize parts of a paragraph,
% not all of which are considered good style. While underlining
% is commonly rejected, experts dispute about whether letterspacing
% should be used or not, and in which cases. If you are not interested
% in such debates, you may well skip to the next section.
%
%
% \subsection*{Theory \dots}
% \label{sec:theory}
%
% To understand the experts' arguments we have to know about the
% conception of \emph{page grayness.} The sum of all characters on
% a page represents a certain amount of grayness, provided that
% the letters are printed black onto white paper.
%
% \person{Jan Tschichold} \cite{Tschichold}, a well known and recognized
% typographer, accepts only forms of emphasizing, which do not disturb this
% grayness. This is only true of italic shape, caps, and
% caps-and-small-caps fonts, but not of ordinary letterspacing, underlining,
% bold face type and so on, all of which appear as either dark or light
% spots in the text area. In his opinion emphasized text shall not catch the eye when
% running over the text, but rather when actually reading the respective words.
%
% Other, less restrictive typographers \cite{Willberg} call this kind of emphasizing
% `integrated' or `aesthetic', while they describe `active' emphasizing apart from it,
% which actually \emph{has} to catch the reader's eye. To the latter group belong commonly
% despised things like letterspacing, demibold face type and even underlined and colored text.
%
% On the other hand, \person{Tschichold} suggests
% to space out caps and caps-and-small-caps fonts on title pages, headings and running headers from
% 1\,pt up to 2\,pt.
% Even in running text legibility of uppercase letters should be improved with slight
% letterspacing, since (the Roman) majuscules don't look right, if they are spaced
% like (the Carolingian) minuscules.\footnote{This suggestion is followed throughout this article,
% although Prof.~\person{Knuth} already considered slight letterspacing with his |cmcsc| fonts.}
%
%
%
%\subsection*{\dots\ and Practice}
%
% However, in the last centuries letterspacing was excessively used,
% underlining at least sometimes, because capitals and italic shape could
% not be used together with the \emph{Fraktur} font and other black-letter fonts,
% which are sometimes also called ``old German'' fonts.
% This tradition is widely continued until today. The same limitations apply still today
% to many languages with non-latin glyphs, which is why letterspacing has a strong
% tradition in eastern countries where Cyrillic fonts are used.
%
% The \person{Duden} \cite{Duden}, a well known German dictionary, explains how to space out properly:
% \emph{Punctuation marks are spaced out like letters, except quotation marks and periods.
% Numbers are never spaced out. The German syllable \mbox{\emph{-sche}} is not spaced
% out in cases like \emph{``der {\so{Virchow{sche}}} Versuch''}\footnote{the \person{Virchow} experiment}.
% In the old German \emph{Fraktur} fonts the ligatures \emph{|ch|, |ck|, |sz|~(\ss)} and~\emph{|tz|} are
% not broken within spaced out text.}
%
% While some books follow all these rules \cite{Muszynski}, others don't
% \cite{Reglement}. In fact, most books in my personal library do \emph{not} space out commas.
%
%
%
%
%
%
%
%
%
% \section{Short introduction and common rules}
%
% The \soul\ package provides five commands that are aimed at emphasizing
% text parts. Each of the commands takes one argument that can either be
% the text itself or the name of a macro that contains text (e.\,g.~|\so\text|)^^A
% ^^A
% \footnote{See~\ref{par:Unexpandable material in command sequences} for
%    some additional information about the latter mode.}^^A
% ^^A
% .
% See table~\ref{tab:survey} for a complete command survey.
%
% {\catcode`\|=12
% \begin{center}
% \begin{tabular}{r@{\hspace{1.5em}}l}
% \verb+\so{letterspacing}+&\so{letterspacing}\\
% \verb+\caps{CAPITALS, Small Capitals}+&\caps{CAPITALS, Small Capitals}\\
% \verb+\ul{underlining}+&\ul{underlining}\\
% \verb+\st{overstriking}+&\st{overstriking}\\
% \verb+\hl{highlighting}+&highlighting\footnotemark\\
% \end{tabular}
% \end{center}
% \footnotetext{The look of highlighting is nowhere demonstrated
%   in this documentation, because it requires a Postscript aware
%   output driver and would come out as ugly black bars on other
%   devices, looking very much like censoring bars. Think of it as
%   the effect of one of those coloring text markers.}
% }
%
% \noindent
% The \cs{\hl} command does only highlight if the \package{color} package
% was loaded, otherwise it falls back to underlining.\footnote{Note that
% you can also use \LaTeX's \package{color} package with Plain \TeX.
% See~\ref{sec:plain} for details.} The highlighting
% color is by default yellow, underlines and overstriking lines are by
% default black. The colors can be changed using the following commands:
%
% {\catcode`\|=12
% \begin{center}
% \begin{tabular}{r@{\hspace{1.5em}}l}
% \verb+\setulcolor{red}+&set underlining color\\
% \verb+\setstcolor{green}+&set overstriking color\\
% \verb+\sethlcolor{blue}+&set highlighting color\\
% \end{tabular}
% \end{center}
% }
%
% \noindent
% |\setulcolor{}| and |\setstcolor{}|  turn coloring off.
% There are only few colors predefined by the \package{color}
% package, but you can easily add custom color definitions.
% See the \package{color} package documentation~\cite{color} for further
% information.
%
% \begin{example}
% |\usepackage{color,soul}|\\
% |\definecolor{lightblue}{rgb}{.90,.95,1}|\\
% |\sethlcolor{lightblue}|\\
% |...|\\
% |\hl{this is highlighted in light blue color}|\\
% \end{example}
%
%
%
%
%
%
%
% \subsection[Some things work]{Some things work \dots}
%
% The following examples may look boring and redundant, because they describe
% nothing else than common \LaTeX\ notation with a few exceptions, but this is
% only the half story: The \soul\ package has to pre-process the argument
% before it can split it into characters and syllables, and all described
% constructs are only allowed because the package explicitly implements them.
%
% \begin{verblist}{1}
% \item[Accents: \so{na\"\i ve}]
%   Accents can be used naturally.
%   Support for the following accents is built-in: |\`|, |\'|, |\^|, |\"|, |\~|,
%   |\=|, |\.|, |\u|, |\v|, |\H|, |\t|, |\c|, |\d|, |\b|, and |\r|. Additionally,
%   if the \package{german} package \cite{german} is loaded you can also use the |"|
%   accent command and write |\so{na"ive}|. See section~\ref{sec:soulaccent} for how to add
%   further accents.
% \item[Quotes: \so{``quotes''}]
%   The \soul\ package recognizes the quotes ligatures |``|, |''| and |,,|.
%   The Spanish ligatures |!`| and |?`| are not recognized and have, thus,
%   to be written enclosed in braces like in |\caps{{!`}Hola!}|.
% \item[Mathematics: \so{foo$x^3$bar}]
%   Mathematic formulas are allowed, as long as they are
%   surrounded by~\texttt\$. Note that the \LaTeX\
%   equivalent |\(...\)| does not work.
% \item[Hyphens and dashes: \so{re-sent}]
%   Explicit hyphens as well as en-dashes~(|--|), em-dashes~(|---|)
%   and the |\slash| command work as usual.
% \item[Newlines: \so{new\\line}]
%   The |\\|~command fills the current line with white space
%   and starts a new line. Spaces or linebreaks afterwards are ignored.
%   Unlike the original \LaTeX\ command \soul's version does not handle
%   optional parameters like in |\\*[1ex]|.
% \item[Breaking lines: \so{foo\linebreak bar}]
%   The \cs{\linebreak} command breaks the line without
%   filling it with white space at the end. \soul's version
%   does not handle optional parameters like in |\linebreak[1]|.
%   \cs{\break} can be used as a synonym.
% \item[Unbreakable spaces: \so{don't~break}]
%   The |~|~command sets an unbreakable space.
% \item[Grouping: \so{Virchow{sche}}]
%   A pair of braces can be used to let a group of characters
%   be seen as one entity, so that \soul\ does
%   for instance not space it out. The contents must, however,
%   not contain potential hyphenation points. (See~\ref{par:Protecting})
% \item[Protecting: \so{foo \mbox{little} bar}]
%   An \cs{\mbox} also lets \soul\ see the contents as one
%   item, but these may even contain hyphenation points. \cs{\hbox} can
%   be used as a synonym.
% \item[Omitting: \so{\soulomit{foo}}]
%   The contents of \cs{\soulomit} bypass the soul core and are typeset
%   as is, without being letterspaced or underlined. Hyphenation points are
%   allowed within the argument. The current font remains active, but can be
%   overridden with \cs{\normalfont} etc.
% \item[Font switching commands: \so{foo \texttt{bar}}]
%   All standard \TeX\ and \LaTeX\ font switching commands are allowed, as
%   well as the \package{yfonts} package \cite{yfonts} font commands like \cs{\textfrak} etc.
%   Further commands have to be registered using the \cs{\soulregister}
%   command (see section~\ref{sec:soulregister}).
% \item[Breaking up ligatures: \ul{Auf{}lage}]
%   Use |{}| or \cs{\null} to break up ligatures like `fl' in \cs{\ul}, \cs{\st} and \cs{\hl}
%   arguments.
%   This doesn't make sense for \cs{\so} and \cs{\caps}, though, because they break up
%   every unprotected (ungrouped\slash unboxed) ligature, anyway, and would
%   then just add undesirable extra space around the additional item.
% \end{verblist}
%
%
%
%
% \subsection{\texorpdfstring{\dots\ }{... }others don't}
%
% Although the new \soul\ is much more robust and forgiving than
% versions prior to~2.0, there are
% still some things that are not allowed in arguments.
% This is due to the complex engine, which has to read and inspect every
% character before it can hand it over to \TeX's paragraph builder.
%
% \begin{verblist}{20}
% \item[Grouping hyphenatable material: \so{foo {little} bar}]
%   Grouped characters must not contain hyphenation points. Instead of
%   |\so{foo {little}}| write |\so{foo \mbox{little}}|. You get a
%   \texttt{`Re\-con\-struction failed'} error and a black square like
%   \errsquare\ in the \caps{DVI} file where you violated this rule.
% \item[Discretionary hyphens: \so{Zu\discretionary{k-}{}{c}ker}]
%   The argument must not contain discretionary hyphens. Thus, you have to
%   handle cases like the German word |Zu\discretionary{k-}{}{c}ker| by yourself.
% \item[Nested soul commands: \ul{foo \so{bar} baz}]
%   \soul\ commands must not be nested. If you really
%   need such, put the inner stuff in a box and use this box. It will, of
%   course, not get broken then.\\
%   \null\qquad|\newbox\anyboxname|\\
%   \null\qquad|\sbox\anyboxname{ \so{the worst} }|\\
%   \null\qquad|\ul{This is by far{\usebox\anyboxname}example!}|\\
%   yields:\\
%   \newbox\anyboxname
%   \sbox\anyboxname{ \so{the worst} }
%   \null\qquad\ul{This is by far{\usebox\anyboxname}example!}
% \item[Leaking font switches: \def\foo{\bf bar} \so{\foo baz}]
%   A hidden font switching command that leaks into its neighborship
%   causes a \texttt{`Reconstruction failed'} error. To avoid this either
%   register the `container' (|\soulregister{\foo}{0}|), or limit its
%   scope as in |\def\foo{{\bf bar}}|. Note that both
%   solutions yield slightly different results.
% \item[Material that needs expansion: \so{\romannumeral\year}]
%   In this example \cs{\so} would try to put space between \cs{\romannumeral}
%   and \cs{\year}, which can, of course, not work.
%   You have to expand the argument before you feed it to \soul, or even better:
%   Wrap the material up in a command sequence and let \soul\ expand it:
%   |\def\x{\romannumeral\year}| |\so\x|. \soul\ tries hard to expand
%   enough, yet not too much.
% \item[Unexpandable material in command sequences: \def\foo{\bar} \so\foo]
%   Some macros might not be expandable in an \cs{\edef} definition^^A
%   \footnote{Try \texttt{\string\edef\string\x\char`\{\string\copyright\char`\}}.
%   Yet \texttt{\string\copyright} works in \soul\ arguments, because it is
%   explicitly taken care of by the package}
%   and have to be protected with \cs{\noexpand} in front. This is automatically done
%   for the following tokens: |~|, \cs{\,}, \cs{\TeX}, \cs{\LaTeX},
%   \cs{\S}, \cs{\slash}, \cs{\textregistered}, \cs{\textcircled}, and \cs{\copyright},
%   as well as for all registered fonts and accents. Instead of putting \cs{\noexpand}
%   manually in front of such commands, as in |\def\foo{foo {\noexpand\bar} bar}| |\so\foo|,
%   you can also register them as special (see section \ref{sec:soulregister}).
% \item[Other weird stuff: \so{foo \verb|\bar| baz}]
%   \soul\ arguments must not contain \LaTeX\ environments, command definitions,
%   and fancy stuff like |\vadjust|. \soul's |\footnote| command replacement
%   does not support optional arguments. As long as you are writing simple,
%   ordinary `horizontal' material, you are on the safe side.
% \end{verblist}
%
%
%
%
%
%
%
% \begin{table}
% \begin{center}
% \catcode`\|=12
% \newcommand*\pref[1]{{\footnotesize\pageref{cmd:#1}}}
% \newcommand*\Ast{\rlap{$^\ast$}}
% \let\m\cs
% \let\ti\textit
% \begin{tabular}{r@{\hspace{.6em}}c@{\hspace{.6em}}l}
% &\hbox to0pt{\hss\footnotesize page\hss}&\\[.5ex]
% \verb+\so{letterspacing}+&                \pref{so}           &\so{letterspacing}\\
% \verb+\caps{CAPITALS, Small Capitals}+&   \pref{caps}         &\caps{CAPITALS, Small Capitals}\\
% \verb+\ul{underlining}+&                  \pref{ul}           &\ul{underlining}\\
% \verb+\st{striking out}+&                 \pref{st}           &\st{striking out}\\
% \verb+\hl{highlighting}+&                 \pref{hl}           &highlighting\\
% \\
% \verb+\soulaccent{\cs}+&                  \pref{soulaccent}   &\ti{add accent} \cs{\cs} \ti{to accent list}\\
% \verb+\soulregister{\cs}{0}+&             \pref{soulregister} &\ti{register command} \m{\cs}\\
% \verb+\sloppyword{text}+&                 \pref{sloppyword}   &\ti{typeset} \m{text} \ti{with stretchable spaces}\\
% \\
% \verb+\sodef\cs{1em}{2em}{3em}+&          \pref{sodef}        &\ti{define new spacing command} \m{\cs}\\
% \verb+\resetso+&                          \pref{resetso}      &\ti{reset} \m{\so} \ti{dimensions}\\
% \verb+\capsdef{////}{1em}{2em}{3em}+\Ast& \pref{capsdef}      &\ti{define (default)} \m{\caps} \ti{data entry}\\
% \verb+\capssave{name}+\Ast&               \pref{capssave}     &\ti{save} \m{\caps} \ti{database under name} \rlap{\texttt{name}}\\
% \verb+\capsselect{name}+\Ast&             \pref{capssave}     &\ti{restore} \m{\caps} \ti{database of name} \rlap{\texttt{name}}\\
% \verb+\capsreset+\Ast&                    \pref{capsreset}    &\ti{clear caps database}\\
% \verb+\setul{1ex}{2ex}+&                  \pref{setul}        &\ti{set} \m{\ul} \ti{dimensions}\\
% \verb+\resetul+&                          \pref{resetul}      &\ti{reset} \m{\ul} \ti{dimensions}\\
% \verb+\setuldepth{y}+&                    \pref{setuldepth}   &\ti{set underline depth to depth of an} y\\
% \verb+\setuloverlap{1pt}+&                \pref{setuloverlap} &\ti{set underline overlap width}\\
% \verb+\setulcolor{red}+&                  \pref{setulcolor}   &\ti{set underline color}\\
% \verb+\setstcolor{green}+&                \pref{setstcolor}   &\ti{set overstriking color}\\
% \verb+\sethlcolor{blue}+&                 \pref{sethlcolor}   &\ti{set highlighting color}\\
% \end{tabular}
% \caption{List of all available commands. The number points to the
%          page where the command is described. Those marked
%          with a little asterisk are only available when the package
%          is used together with \LaTeX, because they rely on the
%          \emph{New Font Selection Scheme (NFSS)} used in \LaTeX.}
% \label{tab:survey}
% \end{center}
% \end{table}
%
%
%
%
%
%
%
% \subsection{Troubleshooting}
%
% Unfortunately, there's just one helpful error message provided by
% the \soul\ package, that actually describes the underlying problem. All other
% messages are generated directly by \TeX\ and show the low-level
% commands that \TeX\ wasn't happy with. They'll hardly point you
% to the violated rule as described in the paragraphs above.
% If you get such a mysterious error message for a line that contains
% a \soul\ statement, then comment that statement out and see if the message
% still appears.
% \texttt{`Incomplete }\cs{\ifcat}\texttt{'} is such a non-obvious message.
% If the message doesn't appear now, then check the argument for
% violations of the rules as listed in~\S\S\,20--26.
%
%
%
% \subsubsection{\texttt{`Reconstruction failed'}}
%
% This message appears, if \ref{par:Grouping hyphenatable material} or
% \ref{par:Leaking font switches} was
% violated. It is caused by the fact that the reconstruction pass
% couldn't collect tokens with an overall width of the syllable that
% was measured by the analyzer. This does either occur when you
% grouped hyphenatable text or used an unregistered command
% that influences the syllable width. Font switching commands belong
% to the latter group. See the above cited sections for how to fix
% these problems.
%
%
%
% \subsubsection{Missing characters}
%
% If you have redefined the internal font as described in section \ref{sec:internalfont},
% you may notice that some characters are omitted without any error message
% being shown. This happens if you have chosen, let's say, a font with
% only 128~characters like the |cmtt10| font, but are using characters
% that aren't represented in this font, e.\,g.~characters with codes
% greater than~127.
%
%
%
%
%
%
%
%
%
% \section{\texorpdfstring{\so{Letterspacing}}{Letterspacing}}
%
% \subsection{How it works}
% \label{sec:so}
%
% The base macro for letterspacing is called \describemacro{\so}.
% It typesets the given argument with \syn{inter-letter space}
% between every two characters, \syn{inner space} between words
% and \syn{outer space} before and after the spaced out text.
% If we let ``$\cdot$'' stand for \syn{inter-letter space}, ``$\ast$''
% for \syn{inner spaces} and ``$\bullet$'' for \syn{outer
% spaces,} then the input on the left side of the following table
% will yield the schematic output on the right side:
%
% \begin{center}
% \def\.{$\cdot$}
% \def\-{\kern1pt$\ast$\kern1pt}
% \def\*{$\bullet$}
% \def\_{\texttt{\char"20}}
% \begin{tabular}{ccc}
% 1.& \verb*|XX\so{aaa bbb ccc}YY|&            \textsf{XXa\.a\.a\-b\.b\.b\-c\.c\.cYY}\\
% 2.& \verb*|XX \so{aaa bbb ccc} YY|&          \textsf{XX\*a\.a\.a\-b\.b\.b\-c\.c\.c\*\kern-1ptYY}\\
% 3.& \verb*|XX {\so{aaa bbb ccc}} YY|&        \textsf{XX\*a\.a\.a\-b\.b\.b\-c\.c\.c\*\kern-1ptYY}\\
% 4.& \verb*|XX \null{\so{aaa bbb ccc}}{} YY|& \textsf{XX\_a\.a\.a\-b\.b\.b\-c\.c\.c\_YY}\\
% \end{tabular}
% \end{center}
% ^^A* %                     fool vim (fixes syntax highlighting)
%
% \noindent
% Case~1 shows how letterspacing macros (\cs{\so} and \cs{\caps}) behave if
% they aren't following or followed by a space: they omit outer space around
% the \soul\ statement. Case~2 is what you'll mostly need---letterspaced
% text amidst running text. Following and leading space get replaced by
% \syn{outer space}. It doesn't matter if there are opening braces before
% or closing braces afterwards. \soul\ can see through both of them (case~3).
% Note that leading space has to be at least |5sp| wide to be recognized as
% space, because \LaTeX\ uses tiny spaces generated by |\hskip1sp| as marker.
% Case~4 shows how to enforce normal spaces instead of \syn{outer spaces:}
% Preceding space can be hidden by |\kern0pt| or \cs{\null} or any character.
% Following space can also be hidden by any token, but note that a typical macro name
% like \cs{\relax} or \cs{\null} would also hide the space thereafter.
%
%
% The values are predefined for typesetting facsimiles mainly with
% \emph{Fraktur} fonts.
% You can define your own spacing
% macros or overwrite the original \cs{\so} meaning using the macro
% \describemacro{\sodef}:
%
% \begin{syntax}
% |\sodef|\<cmd>|{|\<font>|}{|\<inter-letter space>|}{|\<inner space>|}{|\<outer space>|}|
% \end{syntax}
%
% The space dimensions, all of which are mandatory, should be defined in terms of |em|
% letting them grow and shrink with the respective fonts.
%
% \begin{example}
% |\sodef\an{}{.4em}{1em plus1em}{2em plus.1em minus.1em}|
% \end{example}
%
% After that you can type `|\an{example}|' to get
% {\sodef\an{}{.4em}{1em plus1em}{2em plus.1em minus.1em}^^A
% `\an{example}'.}
% The \describemacro{\resetso} command resets \cs{\so}
% to the default values.
%
%
%
%
% \subsection{Some examples}
%
%^^A=====================================================
% \begin{examples}
%
% \soultest{Ordinary text}
%   |\so{electrical industry}|
%   {\so{electrical industry}}
%
% \soultest{Use \texttt{\string\-} to mark hyphenation points}
%   |\so{man\-u\-script}|
%   {\so{man\-u\-script}}
%
% \soultest{Accents are recognized}
%   |\so{le th\'e\^atre}|
%   {\so{le th\'e\^atre}}
%
% \soultest{\texttt{\string\mbox} and \texttt{\string\hbox} protect material that
%   contains hyphenation points. The contents are treated as one, unbreakable entity}
%   |\so{just an \mbox{example}}|
%   {\so{just an \mbox{example}}}
%
% \soultest{Punctuation marks are spaced out, if they are
%   put into the group}
%   |\so{inside.} \& \so{outside}.|
%   {\so{inside.} \& \so{outside}.}
%
% \soultest{Space-out skips may be removed by typing \texttt{\string\<}.
% It's, however, desirable to put the quotation marks out of
% the argument}
%   |\so{``\<Pennsylvania\<''}|
%   {\so{``\<Pennsylvania\<''}}
%
% \soultest{Numbers should never be spaced out}
%   |\so{1\<3 December {1995}}|
%   {\so{1\<3 December {1995}}}
%
% \soultest{Explicit hyphens like |-|, |--| and |---| are recognized.
%   \texttt{\string\slash} outputs a slash and enables \TeX\ to break the line
%   afterwards}
%   |\so{input\slash output}|
%   {\so{input\slash output}}
%
% \soultest{To keep \TeX\ from breaking lines between the hyphen and `jet'
%   you have to protect the hyphen. This is no \soul\ restriction
%   but normal \TeX\ behavior}
%   |\so{\dots and \mbox{-}jet}|
%   {\so{\dots and \mbox{-}jet}}
%
% \soultest{The \texttt{\~} command inhibits line breaks}
%   |\so{unbreakable~space}|
%   {\so{unbreakable~space}}
%
% \soultest{\texttt{\string\\} works as usual. Additional arguments
%   like \texttt{*} or vertical space are not accepted, though}
%   |\so{broken\\line}|
%   {\so{broken\\line}}
%
% \soultest{\texttt{\string\break} breaks the line without filling it with white space}
%   |\so{pretty awful\break test}|
%   {\so{pretty awful\break test}}
%
% \end{examples}
%^^A=====================================================
%
%
%
%
%
%
%
%
%
%
%
% \subsection[Typesetting \texorpdfstring{\caps{caps-and-small-caps}}{caps-and-small-caps} fonts]
%       {Typesetting capitals-and-small-capitals fonts}
%
% There is a special letterspacing command called \describemacro{\caps},
% which differs from \cs{\so} in that it switches to caps-and-small-caps
% font shape, defines only slight spacing and is able to select spacing
% value sets from a database. This is a requirement for high-quality
% typesetting \cite{Tschichold}. The following lines show the effect
% of \cs{\caps} in comparison with the normal textfont and with
% small-capitals shape:
%
% \def\sampletext{DONAUDAMPFSCHIFFAHRTSGESELLSCHAFT}
% \medskip\noindent
% \begin{tabular}{rl}
% |\normalfont|&\sampletext\\
% |\scshape|&{\scshape\sampletext}\\
% |\caps|&\caps\sampletext\\
% ^^A|\person|&\person\sampletext\\
% \end{tabular}
%
% \medbreak\noindent
% The \cs{\caps} font database is by default empty, i.\,e., it contains
% just a single default entry, which yields the result as shown in the
% example above.
% New font entries may be added \emph{on top} of this list using the \describemacro{\capsdef}
% command, which takes five arguments: The first argument describes the font with
% \emph{encoding, family, series, shape,} and \emph{size,}^^A
% \footnote{as defined by the \caps{NFSS}, the ``New Font Selection Scheme''}
% each optionally
% (e.\,g.~|OT1/cmr/m/n/10| for this very font, or only |/ppl///12| for all
% \emph{palatino} fonts at size 12\,pt). The \emph{size} entry may also contain
% a size range (\texttt{5-10}), where zero is assumed for an omitted lower
% boundary (\texttt{-10}) and a very, very big number for an omitted upper
% boundary (\texttt{5-}). The
% upper boundary is not included in the range, so, in the example below, all
% fonts with sizes greater or equal 5\,pt and smaller than 15\,pt are accepted
% ($5\,\mbox{pt}\le size<15\,\mbox{pt}$).
% The second argument may contain font switching commands such as \cs{\scshape},
% it may as well be empty or contain debugging commands (e.\,g.~|\message{*}|).
% The remaining three, mandatory arguments are the spaces as described in
% section~\ref{sec:so}.
%
% \begin{example}
% |\capsdef{T1/ppl/m/n/5-15}{\scshape}{.16em}{.4em}{.2em}|
% \end{example}
%
% The \cs{\caps} command goes through the data list from top to bottom
% and picks up the first matching set, so the order of definition is essential.
% The last added entry is examined first, while the pre-defined default entry
% will be examined last and will match any font, if no entry was taken before.
%
% To override the default values, just define a new default entry using
% the identifier |{////}|. This entry should be defined first, because no
% entry after it can be reached.
%
% The \cs{\caps} database can be cleared with the \describemacro{\capsreset}
% command and will only contain the default entry thereafter. The
% \describemacro{\capssave} command saves the whole current database
% under the given name. \describemacro{\capsselect} restores such a database.
% This allows to predefine different groups of \cs{\caps} data sets:
%
% \begin{example}
% |\capsreset|\\
% |\capsdef{/cmss///12}{}{12pt}{23pt}{34pt}|\\
% |\capsdef{/cmss///}{}{1em}{2em}{3em}|\\
% |...|\\
% |\capssave{wide}|\\
% \end{example}
% \indent
% \begin{example}
% |%---------------------------------------|\\
% |\capsreset|\\
% |\capsdef{/cmss///}{}{.1em}{.2em}{.3em}|\\
% |...|\\
% |\capssave{narrow}|\\
% \end{example}
% \indent
% \begin{example}
% |%---------------------------------------|\\
% |{\capsselect{wide}|\\
% |\title{\caps{Yet Another Silly Example}}|\\
% |}|\\
% \end{example}
%
% See the `|example.cfg|' file for a detailed example.
% If you have defined a bunch of sets for different fonts and sizes,
% you may lose control over what fonts are used by the package. With the
% package option \DescribeOption{capsdefault}\option{capsdefault} selected,
% \cs{\caps} prints its argument underlined, if no set was specified for a
% particular font and the default set had to be used.
%
%
%
%
%
%
%
%
%
%
%
% \subsection{Typesetting Fraktur}
% \label{sec:fraktur}
%
% Black letter fonts^^A
%^^A%%
%   \footnote{See the great black letter fonts, which \person{Yannis Haralambous}
%   kindly provided, and the \package{oldgerm} and \package{yfonts} package~\cite{yfonts}
%   as their \LaTeX\ interfaces.}
%^^A%%
% deserve some additional considerations. As stated in section~\ref{sec:typesetting},
% the ligatures |ch|, |ck|, |sz|~(\cs{\ss}), and~|tz| have to remain unbroken in spaced out
% \emph{Fraktur} text.  This may look strange at first glance, but you'll get used to it:
%
% \begin{example}
% |\textfrak{\so{S{ch}u{tz}vorri{ch}tung}}|
% \end{example}
%
% You already know that grouping keeps the |soul| mechanism from separating such ligatures.
% This is quite important for |s:|, |a*|, and~|"a|. As hyphenation is stronger than
% grouping, especially the |sz| may cause an error, if hyphenation happens to occur between
% the letters |s| and~|z|. (\TeX\ hyphenates the German word |auszer| wrongly like
% |aus-zer| instead of like |au-szer|, because the German hyphenation patterns
% do, for good reason, not see |sz| as `\cs{\ss}'.) In such cases you can protect tokens with the
% sequence e.\,g.~|\mbox{sz}| or a properly defined command. The \cs{\ss} command,
% which is defined by the \package{yfonts} package, and similar commands will suffice as well.
%
%
%
%
%
%
%
% \subsection{Dirty tricks}
% \label{sec:dirtytricks}
%
% Narrow columns are hard to set, because they don't allow much spacing
% flexibility, hence long words often cause overfull boxes. A macro
% could use \cs{\so} to insert stretchability between the single
% characters. Table~\ref{tab:dirtytricks} shows some text typeset with such
% a macro at the left side and under \emph{plain} conditions at
% the right side, both with a width of~6\,pc.
%
% \def\sampletext{Some magazines and newspapers prefer this kind of spacing
% because it reduces hyphenation problems to a minimum\<. Unfortunately\<, such
% paragraphs aren't especially beautiful\<.}
% \newbox\dirtytrick
% \setbox\dirtytrick\vbox{
% \batchmode     ^^A  we don't want to see all these overfull boxes ...
% \leavevmode\hspace{0ptplus1fil}
% \hbox{\parindent0pt\plainsetup\let\<\relax
%   \vtop{\hsize6pc\raggedright\sampletext}\hskip1em
%   \vtop{\hsize6pc\magstylepar\sampletext}\hskip1em
%   \vtop{\hsize6pc\sampletext}\hss}
% \errorstopmode}
%
% \begin{table}
% \begin{center}
% \overfullrule5pt
% \usebox\dirtytrick
% \caption{Ragged-right, magazine style (using \soul), and block-aligned
%          in comparison. But, frankly, none of them is really acceptable.
%          (Don't do this at home, children!)}
% \label{tab:dirtytricks}
% \end{center}
% \end{table}
%
%
%
%
%
%
%
% \section{\texorpdfstring{\ul{Underlining}}{Underlining}}
%
% The underlining macros are my answer to Prof.~\person{Knuth'{\normalfont s}}
% exercise 18.26 from his \TeX{}book~\cite{DEK}. \texttt{:-)} Most of what
% is said about the macro \describemacro{\ul} is also true of the
% striking out macro \describemacro{\st} and the highlighting macro \describemacro{\hl},
% both of which are in fact derived from the former.
%
%
%
%
% \subsection{Settings}
%
% \subsubsection{Underline depth and thickness}
%
% The predefined \syn{underline depth} and \syn{thickness}
% work well with most fonts. They can be changed using the macro \describemacro{\setul}.
%
% \begin{syntax}
% |\setul{|\<underline depth>|}{|\<underline thickness>|}|
% \end{syntax}
%
% Either dimension can be omitted, in which case there has to be
% an empty pair of braces.
% Both values should be defined in terms of |ex|, letting them
% grow and shrink with the respective fonts.
% The \describemacro{\resetul} command restores the standard values.
%
% Another way to set the \syn{underline depth} is to use the macro
% \describemacro{\setuldepth}. It sets the depth such that the
% underline's upper edge lies 1\,pt beneath the given argument's
% deepest depth. If the argument is empty, all
% letters---i.\,e.\ all characters whose \cs{\catcode} currently
% equals 11---are taken. Examples:
%
% \begin{example}
% |\setuldepth{ygp}|\\
% |\setuldepth\strut|\\
% |\setuldepth{}|\\
% \end{example}
%
%
%
% \subsubsection{Line color}
%
% The underlines are by default black. The color can be changed by
% using the \describemacro{\setulcolor} command. It takes one argument that can be any
% of the color specifiers as described in the |color| package. This package
% has to be loaded explicitly.
%
% \indent
% \begin{example}
% |\documentclass{article}|\\
% |\usepackage{color,soul}|\\
% |\definecolor{darkblue}{rgb}{0,0,0.5}|\\
% |\setulcolor{darkblue}|\\
% \end{example}
%
% \indent
% \begin{example}
% |\begin{document}|\\
% |...|\\
% |\ul{Cave: remove all the underlines!}|\\
% |...|\\
% |\end{document}|\\
% \end{example}
%
%
% The colors for overstriking lines and highlighting are likewise set
% with \describemacro{\setstcolor} (default: black) and \describemacro{\sethlcolor}
% (default: yellow). If the \package{color} package wasn't loaded,
% underlining and overstriking color are black, while highlighting
% is replaced by underlining.
%
%
%
%
%
%
%
% \subsubsection{The \program{dvips} problem}
% \label{sec:dvips}
%
% \ul{Underlining}, \st{striking out} and highlighting build up
% their lines with many short line segments. If you used the `\program{dvips}'
% program with default settings, you would get little gaps on some places, because
% the \emph{maxdrift} parameter allows the single objects to drift
% this many pixels from their real positions.
% \bigbreak
%
% \noindent
% There are two ways to avoid the problem, where the \soul\ package
% chooses the second by default:
% \begin{enumerate}
% \item
%   Set the \emph{maxdrift} value to zero, e.\,g.: |dvips -e 0 file.dvi|.
%   This is probably not a good idea, since the letters may then no longer be
%   spaced equally on low resolution printers.
% \item
%   Let the lines stick out by a certain amount on each side so that they
%   overlap. This overlap amount can be set using the \describemacro{\setuloverlap}
%   command. It is set to 0.25\,pt by default. |\setuloverlap{0pt}| turns overlapping off.
% \end{enumerate}
%
%
%
%
%
%
% \subsection{Some examples}
%
%^^A=====================================================
% \begin{examples}
%
% \soultest{Ordinary text}
%   |\ul{electrical industry}|
%   {\ul{electrical industry}}
%
% \soultest{Use \texttt{\string\-} to mark hyphenation points}
%   |\ul{man\-u\-script}|
%   {\ul{man\-u\-script}}
%
% \soultest{Accents are recognized}
%   |\ul{le th\'e\^atre}|
%   {\ul{le th\'e\^atre}}
%
% \soultest{\texttt{\string\mbox} and \texttt{\string\hbox} protect material that
%   contains hyphenation points. The contents are treated as one, unbreakable entity}
%   |\ul{just an \mbox{example}}|
%   {\ul{just an \mbox{example}}}
%
% \soultest{Explicit hyphens like |-|, |--| and |---| are recognized.
%   \texttt{\string\slash} outputs a slash and enables \TeX\ to break the line
%   afterwards}
%   |\ul{input\slash output}|
%   {\ul{input\slash output}}
%
% \soultest{To keep \TeX\ from breaking lines between the hyphen and `jet'
%   you have to protect the hyphen. This is no \soul\ restriction
%   but normal \TeX\ behavior}
%   |\ul{\dots and \mbox{-}jet}|
%   {\ul{\dots and \mbox{-}jet}}
%
% \soultest{The \texttt{\~} command inhibits line breaks}
%   |\ul{unbreakable~space}|
%   {\ul{unbreakable~space}}
%
% \soultest{\texttt{\string\\} works as usual. Additional arguments
%   like \texttt{*} or vertical space are not accepted, though}
%   |\ul{broken\\line}|
%   {\ul{broken\\line}}
%
% \soultest{\texttt{\string\break} breaks the line without filling it with white space}
%   |\ul{pretty awful\break test}|
%   {\ul{pretty awful\break test}}
%
% \soultest{Italic correction needs to be set manually}
%   |\ul{foo \emph{bar\/} baz}|
%   {\ul{foo \emph{bar\/} baz}}
%
% \end{examples}
%^^A=====================================================
%
%
%
%
%
%
%
%
%
%
%
%
%
% \section{Customization}
%
% \subsection{Adding accents}
% \label{sec:soulaccent}
%
% The \soul\ scanner generally sees every input token separately.
% It has to be taught that some tokens belong together. For accents this is done
% by registering them via the \describemacro{\soulaccent} macro.
%
% \begin{syntax}
% |\soulaccent{|\<accent command>|}|
% \end{syntax}
%
% The standard accents, however, are already pre-registered:
% |\`|, |\'|, |\^|, |\"|, |\~|, |\=|, |\.|, |\u|, |\v|, |\H|, |\t|,
% |\c|, |\d|, |\b|, and |\r|. If used together with the \package{german}
% package, \soul\ automatically adds the |"| command.
% Let's assume you have defined |\%| to put some weird accent on
% the next character. Simply put the following line into your |soul.cfg|
% file (see section~\ref{sec:config}):
%
% \begin{example}
% |\soulaccent{\%}|
% \end{example}
%
% Note that active characters like the |"| command have already
% to be \cs{\active} when they are stored or they won't be recognized
% later. This can be done temporarily, as in |{\catcode\`"\active\soulaccent{"}}|.
%
%
%
%
% \subsection{Adding font commands}
% \label{sec:soulregister}
%
% To convince \soul\ not to feed font switching (or other)
% commands to the analyzer, but rather to execute them immediately,
% they have to be registered, too. The \describemacro{\soulregister} macro
% takes the name of a command name and either |0| or |1| for the number
% of arguments:
%
% \begin{syntax}
% |\soulregister{|\<command name>|}{|\<number of arguments>|}|
% \end{syntax}
%
% If \cs{\bf} and \cs{\emph} weren't already registered, you would
% write the following into your |soul.cfg| configuration file:
%
% \begin{example}
% |\soulregister{\bf}{0}        % {\bf foo}| \\
% |\soulregister{\emph}{1}      % \emph{bar}|\\
% \end{example}
%
%
% All standard \TeX\ and \LaTeX\ font commands, as well as the
% \package{yfonts} commands are already pre-registered:
%
% \begin{example}
% |\em, \rm, \bf, \it, \tt, \sc, \sl, \sf, \emph, \textrm,|\\
% |\textsf, \texttt, \textmd, \textbf, \textup, \textsl,|\\
% |\textit, \textsc, \textnormal, \rmfamily, \sffamily,|\\
% |\ttfamily, \mdseries, \upshape, \slshape, \itshape,|\\
% |\scshape, \normalfont, \tiny, \scriptsize, \footnotesize,|\\
% |\small, \normalsize, \large, \Large, \LARGE, \huge, \Huge,|\\
% |\MakeUppercase, \textsuperscript, \footnote,|\\
% |\textfrak, \textswab, \textgoth, \frakfamily,|\\
% |\swabfamily, \gothfamily|\\
% \end{example}
%
% You can also register other commands as fonts, so the
% analyzer won't see them. This may be necessary for some
% macros that \soul\ refuses to typeset correctly.
% But note, that \cs{\so} and \cs{\caps} won't put their
% letter-skips around then.
%
%
%
%
%
%
% \subsection{Changing the internal font}
% \label{sec:internalfont}
%
% The \soul\ package uses the |ectt1000| font while it analyzes
% the syllables. This font is used, because it
% has 256~mono-spaced characters without any kerning.
% It belongs to \person{J\"org Knappen'\textrm{s}}
% \caps{EC}-fonts, which should be part of every modern \TeX\ installation.
% If \TeX\ reports ``\texttt{I can't find file `ectt1000'}'' you don't
% seem to have this font installed. It is recommended that
% you install at least the file |ectt1000.tfm| which has less than 1.4\,kB\null.
% Alternatively, you can let the \soul\ package use the |cmtt10| font that
% is part of any installation, or some other mono-spaced font:
%
% \begin{example}
% |\font\SOUL@tt=cmtt10|
% \end{example}
%
% Note, however, that \soul\ does only handle characters,
% for which the internal font has a character with the same
% character code. As |cmtt10| contains only characters with codes
% 0 to~127, you can't typeset characters with codes 128 to~255.
% These 8-bit character codes are used by many fonts with non-ascii
% glyphs. So the |cmtt10| font will, for example, not work for |T2A|
% encoded cyrillic characters.
%
%
%
%
%
%
%
% \subsection{The configuration file}
% \label{sec:config}
%
% If you want to change the predefined settings or add new features,
% then create a file named `|soul.cfg|' and put it in a directory, where \TeX\
% can find it. This configuration file will then be loaded
% at the end of the |soul.sty| file, so you may redefine
% any settings or commands therein, select package options and even
% introduce new ones. But if you intend to give
% your documents to others, don't forget to give them the
% required configuration files, too! That's how such a file
% could look like:
%
% \indent
% \begin{example}
% |% define macros for logical markup|\\
% |\sodef\person{\scshape}{0.125em}{0.4583em}{0.5833em}|\\
% \\
% |\sodef\SOUL@@@versal{\upshape}{0.125em}{0.4583em}{0.5833em}|\\
% |\DeclareRobustCommand*\versal[1]{%|\\
% |    \MakeUppercase{\SOUL@@@versal{#1}}%|\\
% |}|\\
% \end{example}
%
% \indent
% \begin{example}
% |% load the color package and set|\\
% |% a different highlighting color|\\
% |\RequirePackage{color}|\\
% |\definecolor{lightblue}{rgb}{.90,.95,1}|\\
% |\sethlcolor{lightblue}|\\
% |\endinput|
% \end{example}
%
% You can safely use the |\SOUL@@@| namespace for internal macros---it
% won't be used by the \soul\ package in the future.
%
%
%
%
%
%
%
%
%
%
% \section{Miscellaneous}
%
% \subsection{Using \texorpdfstring{\soul\ }{soul }with other flavors of \texorpdfstring{\TeX}{TeX}}
% \label{sec:plain}
%
% This documentation describes how to use \soul\ together
% with \LaTeXe, for which it is optimized. It works, however, with
% all other flavors of \TeX, too. There are just some minor restrictions
% for Non-\LaTeX\ use:
%
% The \cs{\caps} command doesn't use a database, it is only a dumb
% definition with fixed values. It switches to \cs{\capsfont}, which---unless
% defined explicitly like in the following example---won't really change
% the used font at all. The commands \cs{\capsreset} and \cs{\capssave}
% do nothing.
%
% \begin{example}
% |\font\capsfont=cmcsc10|\\
% |\caps{Tschichold}|\\
% \end{example}
%
% None of the commands are made `robust', so they have to be
% explicitly protected in fragile environments like in \cs{\write}
% statements.
% To make use of colored underlines or highlighting you have to
% use the \package{color} package wrapper from \caps{CTAN}^^A
% \footnote{\texttt{CTAN:/macros/plain/graphics/\char`\{miniltx.tex,color.tex\char`\}}},
% instead of the \package{color} package directly:
%
% \begin{example}
% |\input color|\\
% |\input soul.sty|\\
% |\hl{highlighted}|\\
% |\bye|\\
% \end{example}
%
% The \option{capsdefault} package option is mapped to a simple command
% \describemacro{\capsdefault}.
%
%
%
%
%
%
%
% \subsection{Using \texorpdfstring{\soul\ }{soul }commands for logical markup}
% \label{sec:markup}
%
% It's generally a bad idea to use font style commands like \cs{\textsc}
% in running text. There should always be some reasoning behind changing
% the style, such as ``names of persons shall be typeset in a caps-and-small-caps
% font''. So you declare in your text just that some words are the name of a
% person, while you define in the preamble or, even better, in a separate
% style file how to deal with persons:
%
% \begin{example}
% |\newcommand*\person{\textsc}|\\
% |...|\\
% |``I think it's a beautiful day to go to the zoo and feed|\\
% |the ducks. To the lions.'' --~\person{Brian Kantor}|\\
% \end{example}
%
% It's quite simple to use \soul\ commands that way:
%
% \begin{example}
% |\newcommand\comment*{\ul}             % or \let\comment=\ul|\\
% |\sodef\person{\scshape}{0.125em}{0.4583em}{0.5833em}|\\
% \end{example}
%
% Letterspacing commands like \cs{\so} and \cs{\caps} have to
% check whether they are followed by white space, in which case
% they replace that space by \syn{outer space}. Note that \soul\
% does look through closing braces. Hence you can conveniently bury
% a \soul\ command within another macro like in the following
% example. Use any other token to hide following space if necessary,
% for example the \cs{\null} macro.
%
% \begin{example}
% |\DeclareRobustCommand*\versal[1]{%|\\
% |    \MakeUppercase{\SOUL@@@versal{#1}}%|\\
% |}|\\
% |\sodef\SOUL@@@versal{\upshape}{0.125em}{0.4583em}{0.5833em}|\\
% \end{example}
%
%
% But what if the \soul\ command is for some reason not the last one
% in that macro definition and thus cannot look ahead at the following token?
%
%
% \begin{example}
% |\newcommand*\somsg[1]{\so{#1}\message{#1}}|\\
% |...|\\
% |foo \somsg{bar} baz       % wrong spacing after `bar'!|\\
% \end{example}
%
% In this case you won't get the following space replaced by \emph{outer space}
% because when \soul\ tries to look ahead, it only sees the token
% \cs{\message} and consequently decides that there is no space to replace.
% You can get around this by explicitly calling the space scanner again.
%
% \begin{example}
% |\newcommand*\somsg[1]{{%|\\
% |    \so{#1}%|\\
% |    \message{bar}%|\\
% |    \let\\\SOUL@socheck|\\
% |    \\%|\\
% |}}|\\
% \end{example}
%
% However, \cs{\SOUL@socheck} can't be used directly, because it would discard
% any normal space. \cs{\\} doesn't have this problem.
% The additional pair of braces avoids that its definition leaks out
% of this macro. In the example above you could, of course, simply
% have put \cs{\message} in front, so you hadn't needed to
% use the scanner macro \cs{\SOUL@socheck} at all.
%
% Many packages do already offer logical markup commands that default
% to some standard \LaTeX\ font commands or to \cs{\relax}. One example
% is the \package{jurabib} package~\cite{jurabib}, which makes the use of
% \soul\ a challenge. This package implements lots of
% formatting macros. Let's have a look at one of them, \cs{\jbauthorfont},
% which is used to typeset author names in citations.
% The attempt to simply
% define |\let\jbauthorfont\caps| fails, because the macro isn't directly
% applied to the author name as in |\jbauthorfont{Don Knuth}|, but
% to another command sequence: |\jbauthorfont{\jb@@author}|. Not even
% \cs{\jb@@author} contains the name, but instead further commands that
% at last yield the requested name. That's why we have to expand
% the contents first. This is quite tricky, because we must not
% expand too much, either. Fortunately, we can offer the contents
% wrapped up in yet another macro, so that \soul\ knows that it has to
% use its own macro expansion mechanism:
%
% \begin{example}
% |\renewcommand*\jbauthorfont[1]{{%|\\
% |    \def\x{#1}%|\\
% |    \caps\x|\\
% |}}|\\
% \end{example}
%
% Some additional kerning after |\caps\x| wouldn't hurt, because
% the look-ahead scanner is blinded by further commands that follow
% in the \package{jurabib} package. Now we run into the next problem:
% cited names may contain commands
% that must not get expanded. We have to register them as special
% command:
%
% \begin{example}
% |\soulregister\jbbtasep{0}|\\
% |...|\\
% \end{example}
%
% But such registered commands bypass \soul's kernel and we don't
% get the correct spacing before and afterwards. So we end up
% redefining \cs{\jbbtasep}, whereby you should, of course, use
% variables instead of numbers:
%
% \begin{example}
% |\renewcommand*\jbbtasep{%|\\
% |    \kern.06em|\\
% |    \slash|\\
% |    \hskip.06em|\\
% |    \allowbreak|\\
% |}|\\
% \end{example}
%
% Another problem arises: bibliography entries that must not get
% teared apart are supposed to be enclosed in additional braces.
% This, however, won't work with \soul\ because of
% \ref{par:Grouping hyphenatable material}. A simple trick will
% get you around that problem: define a dummy command that only
% outputs its argument, and register that command:
%
% \begin{example}
% |\newcommand*\together[1]{#1}|\\
% |\soulregister\together{1}|\\
% \end{example}
%
% Now you can write ``|Author = {\together{Don Knuth}}|'' and
% \package{jurabib} won't dare to reorder the parts of the name.
% And what if some name shouldn't get letterspaced at all? Overriding
% a conventional font style like \cs{\textbf} that was globally
% set is trivial, you just have to specify the style that you
% prefer in that very bibliography entry. In our example, if we
% wanted to keep \soul\ from letterspacing a particular entry,
% although they are all formatted by our \cs{\jbauthorfont}
% and hence fed to \cs{\caps}, we'd use the following construction:
%
% \begin{example}
% |Author = {\soulomit{\normalfont\huge Donald E. Knuth}}|\\
% \end{example}
%
% The \package{jurabib} package is probably one of the more
% demanding packages to collaborate with \soul. Everything else
% can just become easier.
%
%
%
%
%
%
%
%
% \subsection{Typesetting long words in narrow columns}
% \label{sec:sloppyword}
%
% Narrow columns are best set |flushleft|, because not even the best
% hyphenation algorithm can guarantee acceptable line breaks without
% overly stretched spaces.
% However, in some rare cases one may be \emph{forced} to typeset
% block aligned. When typesetting in languages like German, where
% there are really long words, the \describemacro{\sloppyword} macro
% might help a little bit. It adds enough stretchability between the
% single characters to make the hyphenation algorithm happy, but
% is still not as ugly as the example in section~\ref{sec:dirtytricks}
% demonstrates. In the following example the left column was typeset
% as ``|Die \sloppyword{Donau...novelle} wird ...|'':
%
% \begin{center}
% \def\word{Do\-nau\-dampf\-schiff\-fahrts\-ge\-sell\-schafts\-^^A
%   ka\-pi\-t\"ans\-wit\-wen\-pen\-si\-ons\-ge\-setz\-no\-vel\-le}
% \begin{minipage}{1.5in}
% \plainsetup
% Die
% \expandafter\sloppyword\expandafter{\word}
% wird mit sofortiger Wirkung au\ss er Kraft gesetzt.
% \end{minipage}
% \hspace{1em}
% \batchmode
% \begin{minipage}{1.5in}
% \plainsetup
% Die \word\ wird mit sofortiger Wirkung au\ss er Kraft gesetzt.
% \end{minipage}
% \errorstopmode
% \end{center}
%
%
%
%
%
%
%
% \subsection{Using \texorpdfstring{\soul\ }{soul }commands in section headings}
%
% Letterspacing was often used for section titles in the past,
% mostly centered and with a closing period. The following example
% shows how to achieve this using the \package{titlesec}
% package \cite{titlesec}:
%
% \begin{example}
% |\newcommand*\periodafter[2]{#1{#2}.}|\\
% |\titleformat{\section}[block]|\\
% |    {\normalfont\centering}|\\
% |    {\thesection.}|\\
% |    {.66em}|\\
% |    {\periodafter\so}|\\
% |...|\\
% |\section{Von den Maassen und Maassst\"aben}|\\
% \end{example}
%
% \bigbreak
% This yields the following output:
%
% \bigskip
% \newbox\examplebox
% \sbox\examplebox{
% \begin{minipage}{.9\textwidth}
% \small
% \bigskip
% \begin{center}
% \so{1\<. Von den Maassen und Maassst\"aben}.
% \bigskip
% \end{center}
% \end{minipage}}
% \fbox{\usebox\examplebox}
% \bigbreak
%
% \noindent
% The \cs{\periodafter} macro adds a period to the title, but not to
% the entry in the table of contents. It takes the name of a command as
% argument, that shall be applied to the title, for example~\cs{\so}.
% Here's a more complicated and complete example:
%
% \begin{example}
% |\documentclass{article}|\\
% |\usepackage[latin1]{inputenc}|\\
% |\usepackage[T1]{fontenc}|\\
% |\usepackage{german,soul}|\\
% |\usepackage[indentfirst]{titlesec}|\\
% \end{example}
%
% \indent
% \begin{example}
% |\newcommand*\sectitle[1]{%|\\
% |    \MakeUppercase{\so{#1}.}\\[.66ex]|\\
% |    \rule{13mm}{.4pt}}|\\
% |\newcommand*\periodafter[2]{#1{#2.}}|\\
% \end{example}
%
% \indent
% \begin{example}
% |\titleformat{\section}[display]|\\
% |    {\normalfont\centering}|\\
% |    {\S. \thesection.}|\\
% |    {2ex}|\\
% |    {\sectitle}|\\
% \end{example}
%
% \indent
% \begin{example}
% |\titleformat{\subsection}[block]|\\
% |    {\normalfont\centering\bfseries}|\\
% |    {\thesection.}|\\
% |    {.66em}|\\
% |    {\periodafter\relax}|\\
% \end{example}
%
% \indent
% \begin{example}
% |\begin{document}|\\
% |\section{Von den Maassen und Maassst\"aben}|\\
% |\subsection{Das L\"angenmaass im Allgemeinen}|\\
% \\
% |Um L\"angen genau messen und vergleichen zu k\"onnen,|\\
% |bedarf es einer gewissen, bestimmten Einheit, mit der|\\
% |man untersucht, wie oft sie selbst, oder ihre Theile,|\\
% |in der zu bestimmenden L\"ange enthalten sind.|\\
% |...|\\
% |\end{document}|
% \end{example}
%
% \bigbreak
% This example gives you roughly the following output,
% which is a facsimile from~\cite{Muszynski}.
%
% \bigskip
% \sbox\examplebox{
% \begin{minipage}{.9\textwidth}
% \small
% \bigskip
% \begin{center}
% \S. 1.\\[2ex]
% \so{VON DEN MAASSEN UND MAASSST\"ABEN}.\\[.66ex]
% \rule{12mm}{.4pt}\\[1.66ex]
% \textbf{1. Das L\"angenmaass im Allgemeinen.}\\[.66em]
% \end{center}
% \leavevmode\qquad
% Um L\"angen genau messen und vergleichen zu k\"onnen,
% bedarf es einer gewissen, bestimmten Einheit, mit der
% man untersucht, wie oft sie selbst, oder ihre Theile,
% in der zu bestimmenden L\"ange enthalten sind.
% \bigskip
% \end{minipage}}
% \fbox{\usebox\examplebox}
% \bigbreak
%
% \noindent
% Note that the definition of \cs{\periodafter} decides if
% the closing period shall be spaced out with the title (1), or
% follow without space (2):
%
% \begin{example}
% 1.\qquad|\newcommand*\periodafter[2]{#1{#2.}}|\\
% 2.\qquad|\newcommand*\periodafter[2]{#1{#2}.}|\\
% \end{example}
%
%
%
% \noindent
% If you need to underline section titles, you can easily
% do it with the help of the \package{titlesec} package. The following
% example underlines the section title, but not the section
% number:
%
% \begin{example}
% |\titleformat{\section}|\\
% |    {\LARGE\titlefont}|\\
% |    {\thesection}|\\
% |    {.66em}|\\
% |    {\ul}|\\
% \end{example}
%
% \noindent
% The \cs{\titlefont} command is provided by the \caps{``\small{KOMA}}~script''
% package. You can write |\normalfont\sffamily\bfseries| instead.
% The following example does additionally underline the section number:
%
% \begin{example}
% |\titleformat{\section}|\\
% |    {\LARGE\titlefont}|\\
% |    {\ul{\thesection{\kern.66em}}}|\\
% |    {0pt}|\\
% |    {\ul}|\\
% \end{example}
%
%
%
%
%
%
%
%
%
%
%
% \section{How the package works}
%
% \subsection{The kernel}
% \so{Letterspacing,} \ul{underlining}, \st{striking out} and highlighting
% use the same kernel. It lets a \emph{word scanner} run over the given argument,
% which inspects every token. If a token is a command registered via \cs{\soulregister},
% it is executed immediately. Other tokens are only counted and trigger some action
% when a certain number is reached (quotes and dashes). Three subsequent `|-|', for example,
% trigger |\SOUL@everyexhyphen{---}|. A third group leads to special actions,
% like |\mbox| that starts reading-in a whole group to protect its contents and let them be
% seen as one entity. All other tokens, mostly characters and digits, are collected in
% a word register, which is passed to the analyzer, whenever a whole word was read in.
%
% The analyzer typesets the word in a 1\,sp ($=\frac1{65536}$\,pt) wide \cs{\vbox},
% hence encouraging \TeX\ to break lines at every possible hyphenation point. It
% uses the mono-spaced \cs{\SOUL@tt} font (|ectt1000|), so as to avoid any inter-character
% kerning. Now the \cs{\vbox} is decomposed splitting off \cs{\hbox} after \cs{\hbox}
% from the bottom. All boxes, each of which contains one syllable, are pushed onto a
% stack, which is provided by \TeX's grouping mechanism. When returning from the
% recursion, box after box is fetched from the stack, its width measured and fed to the
%``reconstructor''.
%
% This reconstruction macro (\cs{\SOUL@dosyllable}) starts to read tokens
% from the just analyzed word until the given syllable
% width is obtained. This is repeated for each syllable. Every time the engine
% reaches a relevant state, the corresponding driver macro is executed
% and, if necessary, provided with some data. There is a macro that is
% executed for each token, one for each syllable, one for each space etc\null.
%
% The engine itself doesn't know how to letterspace or to underline. It
% just tells the selected driver about the structure of the given argument.
% There's a default driver (\cs{\SOUL@setup}) that does only set the
% interface macros to a reasonable default state, but doesn't really do anything.
% Further drivers can safely inherit these settings and only need to
% redefine what they want to change.
%
%
%
%
%
% \subsection{The interface}
% \label{sec:interface}
%
% \subsubsection{The registers}
%
% The package offers eight interface macros that can be used to define
% the required actions. Some of the macros receive data as macro parameter
% or in special \emph{token} or \emph{dimen} registers. Here is a list of
% all available registers:
%
% \begin{labeling}{\hspace{.36\hsize}}
% \item[\texttt{\string\SOUL@token}]
%   This token register contains the current token. It has to be used as |\the\SOUL@token|.
%   The macro \cs{\SOUL@gettoken} reads the next token into \cs{\SOUL@token} and
%   can be used in any interface macro. If you don't want to lose the old meaning,
%   you have to save it explicitly. \cs{\SOUL@puttoken} pushes the token
%   back into the queue, without changing \cs{\SOUL@token}. You can only
%   put one token back, otherwise you get an error message.
% \item[\texttt{\string\SOUL@lasttoken}]
%   This token register contains the last token.
% \item[\texttt{\string\SOUL@syllable}]
%   This token register contains all tokens that were already collected for
%   the current syllable. When used in \cs{\SOUL@everysyllable}, it
%   contains the \emph{whole} syllable.
% \item[\texttt{\string\SOUL@charkern}]
%   This dimen register contains the kerning value between the current and the next character.
%   Since most character pairs don't require a kerning value to be applied and the
%   output in the logfile shouldn't be cluttered with |\kern0pt| it is
%   recommended to write |\SOUL@setkern\SOUL@charkern|, which sets
%   kerning for non-zero values only.
% \item[\texttt{\string\SOUL@hyphkern}]
%   This dimen register contains the kerning value between the current character
%   and the hyphen character or, when used in \cs{\SOUL@everyexhyphen}, the
%   kerning between the last character and the explicit hyphen.
% \end{labeling}
%
%
%
%
%
%
% \subsubsection{The interface macros}
%
% The following list describes each of the interface macros and which
% registers it can rely on. The mark between label and description
% will be used in section \ref{sec:interfaceexamples} to show when
% the macros are executed. The addition |#1| means that the macro
% takes one argument.
%
% \begin{labeling}{\hspace{.36\hsize}}
% \def\m#1{\leavevmode\llap{\hbox to1em{\hss#1\hss}\hskip.7em}}
% \item[\texttt{\string\SOUL@preamble}]\m{$P$}^^A
%   executed once at the beginning
% \item[\texttt{\string\SOUL@postamble}]\m{$E$}^^A
%   executed once at the end
% \item[\texttt{\string\SOUL@everytoken}]\m{$T$}^^A
%   executed after scanning a token; It gets that
%   token in \cs{\SOUL@token} and has to care for inserting
%   the kerning value \cs{\SOUL@charkern} between this and the next character.
%   To look at the next character, execute \cs{\SOUL@gettoken}, which
%   replaces \cs{\SOUL@token} by the next token. This token has to
%   be put back into the queue using \cs{\SOUL@puttoken}.
% \item[\texttt{\string\SOUL@everysyllable}]\m{$S$}^^A
%   This macro is executed after scanning a whole syllable. It gets the
%   syllable in \cs{\SOUL@syllable}.
% \item[\texttt{\string\SOUL@everyhyphen}]\m{$-$}^^A
%   This macro is executed at every implicit hyphenation point.
%   It is responsible for setting the hyphen and will likely do this
%   in a \cs{\discretionary} statement. It has to care about the
%   kerning values. The registers \cs{\SOUL@lasttoken}, \cs{\SOUL@syllable},
%   \cs{\SOUL@charkern} and \cs{\SOUL@hyphkern} contain useful information.
%   Note that \cs{\discretionary} inserts \cs{\exhyphenpenalty}
%   if the first part of the discretionary is empty, and
%   \cs{\hyphenpenalty} else.
% \item[\texttt{\string\SOUL@everyexhyphen\#1}]\m{$=$}^^A
%   This macro is executed at every explicit hyphenation point. The
%   hyphen `character' (one of hyphen, en-dash, em-dash or \cs{\slash})
%   is passed as parameter |#1|. A minimal implementation
%   would be |{#1\penalty\exhyphenpenalty}|. The kerning value
%   between the last character and the hyphen is passed in \cs{\SOUL@hyphkern},
%   that between the hyphen and the next character in \cs{\SOUL@charkern}.
%   The last syllable can be found in \cs{\SOUL@syllable}, the last
%   character in \cs{\SOUL@lasttoken}.
% \item[\texttt{\string\SOUL@everyspace\#1}]\m{\texttt{\char`\ }}^^A
%   This macro is executed between every two words. It is responsible for
%   setting the space. The engine submits a \cs{\penalty} setting as
%   parameter |#1| that should be put in front of the space. The
%   macro should at least do |{#1\space}|. Further information can be found in
%   \cs{\SOUL@lasttoken} and \cs{\SOUL@syllable}. Note that this macro does not
%   care for the leading and trailing space. This is the job of
%   \cs{\SOUL@preamble} and \cs{\SOUL@postamble}.
% \end{labeling}
%
%
%
%
%
%
% \subsubsection{Some examples}
% \label{sec:interfaceexamples}
%
% The above list's middle column shows a mark that indicates in the
% following examples, when the respective macros are executed:\nopagebreak
%
% \begin{labeling}{\hspace{.36\hsize}}
% \item[\normalfont\an{word}]
%   \cs{\SOUL@everytoken}$^T$ is executed for every token. \cs{\SOUL@everysyllable}$^S$
%   is \emph{additionally} executed for every syllable. You will mostly just
%   want to use either of them.
% \item[\normalfont\an{one two}]
%   The macro |\SOUL@everyspace| is executed at every space within
%   the \soul\ argument. It has to take one argument, that can either
%   be empty or contain a penalty, that should be applied to the space.
% \item[\normalfont\an{example}\kern-1em]
%   The macro |\SOUL@everyhyphen| is executed at every possible
%   implicit hyphenation point.
% \item[\normalfont\an{beta-test}]
%   Explicit hyphens trigger \cs{\SOUL@everyexhyphen}.
% \end{labeling}
%
%
%
% \bigbreak
% \noindent
% It's only natural that these examples, too,
% were automatically typeset by the |soul| package
% using a special driver:
%
% \begin{example}
% |\DeclareRobustCommand*\an{%|\\
% |    \def\SOUL@preamble{$^{^P}$}%|\\
% |    \def\SOUL@everyspace##1{##1\texttt{\char`\ }}%|\\
% |    \def\SOUL@postamble{$^{^E}$}%|\\
% |    \def\SOUL@everyhyphen{$^{^-}$}%|\\
% |    \def\SOUL@everyexhyphen##1{##1$^{^=}$}%|\\
% |    \def\SOUL@everysyllable{$^{^S}$}%|\\
% |    \def\SOUL@everytoken{\the\SOUL@token$^{^T}$}%|\\
% |    \def\SOUL@everylowerthan{$^{^L}$}%|\\
% |    \SOUL@}|\\
% \end{example}
%
%
%
%
%
%
%
%
% \subsection{A driver example}
%
% Let's define a \soul\ driver that allows to typeset text
% with a \cs{\cdot} at every potential hyphenation point. The name of
% the macro shall be \cs{\sy} (for \emph{syllables}).
% Since the \soul\ mechanism is highly fragile, we use the \LaTeX\
% command \cs{\DeclareRobustCommand}, so that the \cs{\sy} macro
% can be used even in section headings etc. The \cs{\SOUL@setup}
% macro sets all interface macros to reasonable default definitions.
% This could of course be done manually, too. As we won't
% make use of \cs{\SOUL@everytoken} and \cs{\SOUL@postamble}
% and both default to \cs{\relax}, anyway, we don't have to
% define them here.
%
% \begin{example}
% |\DeclareRobustCommand*\sy{%|\\
% |    \SOUL@setup|\\
% \end{example}
%
% We only set \cs{\lefthyphenmin} and \cs{\righthyphenmin} to zero
% at the beginning. All changes are restored automatically,
% so there's nothing to do at the end.
%
% \begin{example}
% |    \def\SOUL@preamble{\lefthyphenmin=0 \righthyphenmin=0 }%|\\
% \end{example}
%
% We only want simple spaces. Note that these are not provided
% by default! \cs{\SOUL@everyspace} may get a penalty to be
% applied to that space, so we set it before.
%
% \begin{example}
% |    \def\SOUL@everyspace##1{##1\space}%|\\
% \end{example}
%
% There's nothing to do for \cs{\SOUL@everytoken}, we rather let
% \cs{\SOUL@everysyllable} handle a whole syllable at once.
% This has the advantage, that we don't have to deal with
% kerning values, because \TeX\ takes care of that.
%
% \begin{example}
% |    \def\SOUL@everysyllable{\the\SOUL@syllable}|\\
% \end{example}
%
% The \TeX\ primitive \cs{\discretionary}
% takes three arguments: 1.~pre-hyphen material
% 2.~post-hyphen material, and 3.~no-hyphenation material.
%
% \begin{example}
% |    \def\SOUL@everyhyphen{%|\\
% |        \discretionary{%|\\
% |            \SOUL@setkern\SOUL@hyphkern|\\
% |            \SOUL@sethyphenchar|\\
% |        }{}{%|\\
% |            \hbox{\kern1pt$\cdot$}%|\\
% |        }%|\\
% |    }%|\\
%\end{example}
%
% Explicit hyphens like dashes and slashes shall be set normally.
% We just have to care for kerning. The hyphen has to be put in
% a box, because, as \cs{\hyphenchar}, it would yield its own, internal
% \cs{\discretionary}. We need to set ours instead, though.
%
% \begin{example}
% |    \def\SOUL@everyexhyphen##1{%|\\
% |        \SOUL@setkern\SOUL@hyphkern|\\
% |        \hbox{##1}%|\\
% |        \discretionary{}{}{%|\\
% |            \SOUL@setkern\SOUL@charkern|\\
% |        }%|\\
% |    }|%\\
% \end{example}
%
% Now that the interface macros are defined, we can start the scanner.
%
% \begin{example}
% |    \SOUL@|\\
% |}|\\
% \end{example}
%
% \hyphenation{al-go-rithm lin-guists ex-cel-lent} ^^A correct?
% \noindent
% \emph{\sy{This little macro will hardly be good enough
% for linguists, although it uses {\TeX's} excellent hyphenation algorithm,
% but it is at least a nice alternative to the}} \cs{\showhyphens} \emph{\sy{command}.}
%
%
%
%
%
%
%
% \section*{Acknowledgements}
%
% A big thank you goes to \person{Stefan Ulrich} for his tips and bug reports
% during the development of versions 1.\lower2pt\hbox{*} and for his lessons on high quality
% typesetting. The \cs{\caps} mechanism was very much influenced by his
% suggestions. Thanks to \person{Alexander Shibakov} and \person{Frank Mittelbach,}
% who sent me a couple of bug reports and feature requests, and finally encouraged
% me to (almost) completely rewrite \soul. \person{Thorsten Manegold} contributed
% a series of bug reports, helping to fix \soul's macro expander and hence making
% it work together with the \package{jurabib} package.
% Thanks to \person{Axel Reichert, Anshuman Pandey,} and \person{Peter Kreynin} for
% detailed bug reports.
% \person{Rowland McDonnel} gave useful hints for how to improve the documentation,
% but I'm afraid he will still not be satisfied, and rightfully so. If only documentation
% writing weren't that boring.~~\texttt{;-)}
%
%
%
%
%
%
%
% \begin{thebibliography}{00}
% \raggedright
%
% \bibitem{jurabib}{\person{Berger, Jens:} \bibtitle{The jurabib package.} \CTAN-Archive, 2002, v0.52h.}
%
% \bibitem{titlesec}{\person{Bezos, Javier:} \bibtitle{The titlesec and titletoc package.}
%   \CTAN-Archive, 1999, v2.1.}
%
% \bibitem{color}{\person{Carlisle, D. P.:} \bibtitle{The color package.} \CTAN-Archive, 1997, v1.0d.}
%
% \bibitem{Duden}{Duden, Volume 1. \bibtitle{Die Rechtschreibung.} Bibliographisches Institut,
%   Mann\-heim--\hskip0pt Wien--Z\"urich, 1986, 19th~edition.}
%
% \bibitem{DEK}{\person{Knuth, Donald E.:} \bibtitle{The \TeX book.}
%   Addison--Wesley Publishing Company, Reading/Massachusetts, 1989, 16th~edition.}
%
% \bibitem{Muszynski}{\person{Muszynski, Carl} and \person{P\v rihoda, Eduard:}
%   \bibtitle{Die Terrainlehre in Verbindung mit der Darstellung, Beurtheilung und
%   Beschreibung des Terrains vom milit\"arischen
%   Standpunkte.}
%   L.\,W.~Seidel \&\ Sohn, Wien, 1872.}
%
% \bibitem{Reglement}{Normalverordnungsblatt f\"ur das k.\,u.\,k.~Heer.
%   \bibtitle{Exercier-Reglement f\"ur die k.\,u.\,k.~Cavallerie, I. Theil.}
%   Wien, k.\,k.~Hof- und Staatsdruckerei, 1898, 4th~edition.}
%
% \bibitem{german}{\person{Raichle, Bernd:} \bibtitle{The german package.} \CTAN-Archive, 1998, v2.5e.}
%
% \bibitem{yfonts}{\person{Schmidt, Walter:} \bibtitle{Ein Makropaket f\"ur die gebrochenen
%   Schriften.} \CTAN-Archive, 1998, v1.2.}
%
% \bibitem{Tschichold}{\person{Tschichold, Jan:} \bibtitle{Ausgew\"ahlte Aufs\"atze \"uber Fragen
%   der Gestalt des Buches und der Typographie.} Birkh\"auser, Basel,
%   1987, 2nd~edition.}
%
% \bibitem{Willberg}{\person{Willberg, Hans Peter} and \person{Forssmann, Friedrich:}
%   \bibtitle{Lesetypographie.} H. Schmidt, Mainz, 1997.}
%
% \end{thebibliography}
%
%
% \StopEventually{\addtocontents{toc}{\protect\end{multicols}}}
%
%
%
%
%
%
%
%
%^^A max 72 columns
%^^A--------------------------------------------------------------------
%
%
%
%
% \section{The implementation}
%
% \subsection*{The package preamble}
%
% This piece of code makes sure that the package is only loaded
% once. While this is guaranteed by \LaTeX, we have to do it
% manually for all other flavors of \TeX.
%
%    \begin{macrocode}
\expandafter\ifx\csname SOUL@\endcsname\relax\else
    \expandafter\endinput
\fi
%    \end{macrocode}
%
%
%
%
%^^A--------------------------------------------------------------------
%
%
%
%
% \noindent
% Fake some of the \LaTeX\ commands if we were loaded by another flavor
% of \TeX. This might break some previously loaded packages, though,
% if e.\,g.~\cs{\mbox} was already in use. But we don't care \dots
%
%    \begin{macrocode}
\ifx\documentclass\SOULundefined
    \chardef\atcode=\catcode`@
    \catcode`\@=11
    \def\DeclareRobustCommand*{\def}
    \let\newcommand\DeclareRobustCommand
    \def\DeclareOption#1#2{\expandafter\def\csname#1\endcsname{#2}}
    \def\PackageError#1#2#3{{%
        \newlinechar`^^J%
        \errorcontextlines\z@
        \edef\\{\errhelp{#3}}\\%
        \errmessage{Package #1 error: #2}%
    }}
    \def\@height{height}
    \def\@depth{depth}
    \def\@width{width}
    \def\@plus{plus}
    \def\@minus{minus}
    \font\SOUL@tt=ectt1000
    \let\@xobeysp\space
    \let\linebreak\break
    \let\mbox\hbox
%    \end{macrocode}
%
%
%
%
%^^A--------------------------------------------------------------------
%
%
%
%
% \noindent
% \soul\ tries to be a good \LaTeX\ citizen if used under \LaTeX\ and
% declares itself properly. Most command sequences in the package
% are protected by the |SOUL@| namespace, all other macros are first
% defined to be empty. This will give us an error message \emph{now}
% if one of those was already used by another package.
%
%    \begin{macrocode}
\else
    \NeedsTeXFormat{LaTeX2e}
    \ProvidesPackage{soul}
        [2003/11/17 v2.4 letterspacing/underlining  (mf)]
    \newfont\SOUL@tt{ectt1000}
    \newcommand*\sodef{}
    \newcommand*\resetso{}
    \newcommand*\capsdef{}
    \newcommand*\capsfont{}
    \newcommand*\setulcolor{}
    \newcommand*\setuloverlap{}
    \newcommand*\setul{}
    \newcommand*\resetul{}
    \newcommand*\setuldepth{}
    \newcommand*\setstcolor{}
    \newcommand*\sethlcolor{}
    \newcommand*\so{}
    \newcommand*\ul{}
    \newcommand*\st{}
    \newcommand*\hl{}
    \newcommand*\caps{}
    \newcommand*\soulaccent{}
    \newcommand*\soulregister{}
    \newcommand*\soulfont{}
    \newcommand*\soulomit{}
\fi
%    \end{macrocode}
%
%
%
%
%^^A--------------------------------------------------------------------
%
%
%
%
% \noindent
% Other packages wouldn't be happy if we reserved piles of \cs{\newtoks} and
% \cs{\newdimen}, so we try to get away with their \cs{\...def} counterparts
% where possible.
% Local registers are always even, while global ones are odd---this is a
% \TeX\ convention.
%
%    \begin{macrocode}
\newtoks\SOUL@word
\newtoks\SOUL@lasttoken
\toksdef\SOUL@syllable\z@
\newtoks\SOUL@cmds
\newtoks\SOUL@buffer
\newtoks\SOUL@token
\dimendef\SOUL@syllgoal\z@
\dimendef\SOUL@syllwidth\tw@
\dimendef\SOUL@charkern=4
\dimendef\SOUL@hyphkern=6
\countdef\SOUL@minus\z@
\countdef\SOUL@comma\tw@
\countdef\SOUL@apo=4
\countdef\SOUL@grave=6
\newskip\SOUL@spaceskip
\newif\ifSOUL@ignorespaces
%    \end{macrocode}
%
%
%
%
%^^A--------------------------------------------------------------------
%
%
%
%
% ^^A\FIXME{\newpage}
% \begin{macro}{\soulomit}
% \begin{macro}{\SOUL@ignorem}
% \begin{macro}{\SOUL@ignore}
% \begin{macro}{\SOUL@stopm}
% \begin{macro}{\SOUL@stop}
% \begin{macro}{\SOUL@relaxm}
% \begin{macro}{\SOUL@lowerthanm}
% \begin{macro}{\SOUL@hyphenhintm}
% These macros are used as markers. To be able to check for such a
% marker with \cs{\ifx} we have also to create a macro that contains
% the marker. \cs{\SOUL@spc} shall contain a normal space with a
% \cs{\catcode} of~10.
%
%    \begin{macrocode}
\def\soulomit#1{#1}
\def\SOUL@stopM{\SOUL@stop}
\let\SOUL@stop\relax
\def\SOUL@lowerthan{}
\def\SOUL@lowerthanM{\<}
\def\SOUL@hyphenhintM{\-}
\def\SOUL@n*{\let\SOUL@spc= }\SOUL@n* %
%    \end{macrocode}
% \end{macro}
% \end{macro}
% \end{macro}
% \end{macro}
% \end{macro}
% \end{macro}
% \end{macro}
% \end{macro}
%
%
%
%
%^^A--------------------------------------------------------------------
%
%
%
%
% \subsection{The kernel}
%
% \begin{macro}{\SOUL@}
% This macro is the entry to \soul. Using it does only make
% sense after setting up a \soul\ driver. The next token after
% the \soul\ command will be assigned to \cs{\SOUL@@}. This can be
% some text enclosed in braces, or the name of a macro that contains text.
%
%    \begin{macrocode}
\def\SOUL@{%
    \futurelet\SOUL@@\SOUL@expand
}
%    \end{macrocode}
% \end{macro}
%
%
%
%
%^^A--------------------------------------------------------------------
%
%
%
%
% \begin{macro}{\SOUL@expand}
% If the first token after the \soul\ command was an opening
% brace we start scanning. Otherwise,
% if the first token was a macro name, we expand that macro and
% call \cs{\SOUL@} with its contents again. Unfortunately, we have to
% exclude some macros therein from expansion.
%
%    \begin{macrocode}
\def\SOUL@expand{%
    \ifcat\bgroup\noexpand\SOUL@@
        \let\SOUL@n\SOUL@start
    \else
        \bgroup
            \def\\##1##2{\def##2{\noexpand##2}}%
            \the\SOUL@cmds
            \SOUL@buffer={%
                \\\TeX\\\LaTeX\\\soulomit\\\mbox\\\hbox\\\textregistered
                \\\slash\\\textcircled\\\copyright\\\S\\\,\\\<\\\>\\~%
                \\\\%
            }%
            \def\\##1{\def##1{\noexpand##1}}%
            \the\SOUL@buffer
            \let\protect\noexpand
            \xdef\SOUL@n##1{\noexpand\SOUL@start{\SOUL@@}}%
        \egroup
    \fi
    \SOUL@n
}
\long\def\SOUL@start#1{{%
    \let\<\SOUL@lowerthan
    \let\>\empty
    \def\soulomit{\noexpand\soulomit}%
    \gdef\SOUL@eventuallyexhyphen##1{}%
    \let\SOUL@soeventuallyskip\relax
    \SOUL@spaceskip=\fontdimen\tw@\font\@plus\fontdimen\thr@@\font
        \@minus\fontdimen4\font
    \SOUL@ignorespacesfalse
    \leavevmode
    \SOUL@preamble
    \SOUL@lasttoken={}%
    \SOUL@word={}%
    \SOUL@minus\z@
    \SOUL@comma\z@
    \SOUL@apo\z@
    \SOUL@grave\z@
    \SOUL@do{#1}%
    \SOUL@postamble
}}
\long\def\SOUL@do#1{%
    \SOUL@scan#1\SOUL@stop
}
%    \end{macrocode}
% \end{macro}
%
%
%
%
%^^A--------------------------------------------------------------------
%
%
%
%
% \subsection{The scanner}
%
% \begin{macro}{\SOUL@scan}
% This is the entry point for the scanner. It calls \cs{\SOUL@eval}
% and will in turn be called by \cs{\SOUL@eval} again for every
% new token to be scanned.
%
%    \begin{macrocode}
\def\SOUL@scan{%
    \futurelet\SOUL@@\SOUL@eval
}
%    \end{macrocode}
% \end{macro}
%
%
%
%
%^^A--------------------------------------------------------------------
%
%
%
%
% \begin{macro}{\SOUL@eval}
% And here it is: the scanner's heart. It cares for quotes and dashes
% ligatures and handles all commands that must not be fed to the
% analyzer.
%
%    \begin{macrocode}
\def\SOUL@eval{%
    \def\SOUL@n*##1{\SOUL@scan}%
    \if\noexpand\SOUL@@\SOUL@spc
    \else
        \SOUL@ignorespacesfalse
    \fi
    \ifnum\SOUL@minus=\thr@@
        \SOUL@flushminus
    \else\ifnum\SOUL@comma=\tw@
        \SOUL@flushcomma
    \else\ifnum\SOUL@apo=\tw@
        \SOUL@flushapo
    \else\ifnum\SOUL@grave=\tw@
        \SOUL@flushgrave
    \fi\fi\fi\fi
    \ifx\SOUL@@-\else\SOUL@flushminus\fi
    \ifx\SOUL@@,\else\SOUL@flushcomma\fi
    \ifx\SOUL@@'\else\SOUL@flushapo\fi
    \ifx\SOUL@@`\else\SOUL@flushgrave\fi
    \ifx\SOUL@@-%
        \advance\SOUL@minus\@ne
    \else\ifx\SOUL@@,%
        \advance\SOUL@comma\@ne
    \else\ifx\SOUL@@'%
        \advance\SOUL@apo\@ne
    \else\ifx\SOUL@@`%
        \advance\SOUL@grave\@ne
    \else
        \SOUL@flushminus
        \SOUL@flushcomma
        \SOUL@flushapo
        \SOUL@flushgrave
        \ifx\SOUL@@\SOUL@stop
            \def\SOUL@n*{%
                \SOUL@doword
                \SOUL@eventuallyexhyphen\null
            }%
        \else\ifx\SOUL@@\par
            \def\SOUL@n*\par{\par\leavevmode\SOUL@scan}%
        \else\if\noexpand\SOUL@@\SOUL@spc
            \SOUL@doword
            \SOUL@eventuallyexhyphen\null
            \ifSOUL@ignorespaces
            \else
                \SOUL@everyspace{}%
            \fi
            \def\SOUL@n* {\SOUL@scan}%
        \else\ifx\SOUL@@\\%
            \SOUL@doword
            \SOUL@eventuallyexhyphen\null
            \SOUL@everyspace{\unskip\nobreak\hfil\break}%
            \SOUL@ignorespacestrue
        \else\ifx\SOUL@@~%
            \SOUL@doword
            \SOUL@eventuallyexhyphen\null
            \SOUL@everyspace{\nobreak}%
        \else\ifx\SOUL@@\slash
            \SOUL@doword
            \SOUL@eventuallyexhyphen{/}%
            \SOUL@exhyphen{/}%
        \else\ifx\SOUL@@\mbox
            \def\SOUL@n*{\SOUL@addprotect}%
        \else\ifx\SOUL@@\hbox
            \def\SOUL@n*{\SOUL@addprotect}%
        \else\ifx\SOUL@@\soulomit
            \def\SOUL@n*\soulomit##1{%
                \SOUL@doword
                {\spaceskip\SOUL@spaceskip##1}%
                \SOUL@scan
            }%
        \else\ifx\SOUL@@\break
            \SOUL@doword
            \break
        \else\ifx\SOUL@@\linebreak
            \SOUL@doword
            \SOUL@everyspace{\linebreak}%
        \else\ifcat\bgroup\noexpand\SOUL@@
            \def\SOUL@n*{\SOUL@addgroup{}}%
        \else\ifcat$\noexpand\SOUL@@
            \def\SOUL@n*{\SOUL@addmath}%
        \else
            \def\SOUL@n*{\SOUL@dotoken}%
        \fi\fi\fi\fi\fi\fi\fi\fi\fi\fi\fi\fi\fi
    \fi\fi\fi\fi
    \SOUL@n*%
}
%    \end{macrocode}
% \end{macro}
%
%
%
%
%^^A--------------------------------------------------------------------
%
%
%
%
% \begin{macro}{\SOUL@flushminus}
% \begin{macro}{\SOUL@flushcomma}
% \begin{macro}{\SOUL@flushapo}
% \begin{macro}{\SOUL@flushgrave}
% As their names imply, these macros flush special tokens or token
% groups to the word register. They don't do anything if the respective
% counter equals zero. \cs{\SOUL@minus} does also flush the word
% register, because hyphens disturb the analyzer.
%
%    \begin{macrocode}
\def\SOUL@flushminus{%
    \ifcase\SOUL@minus
    \else
        \SOUL@doword
        \SOUL@eventuallyexhyphen{-}%
        \ifcase\SOUL@minus
        \or
            \SOUL@exhyphen{-}%
        \or
            \SOUL@exhyphen{--}%
        \or
            \SOUL@exhyphen{---}%
        \fi
        \SOUL@minus\z@
    \fi
}
\def\SOUL@flushcomma{%
    \ifcase\SOUL@comma
    \or
        \edef\x{\SOUL@word={\the\SOUL@word,}}\x
    \or
        \edef\x{\SOUL@word={\the\SOUL@word{{,,}}}}\x
    \fi
    \SOUL@comma\z@
}
\def\SOUL@flushapo{%
    \ifcase\SOUL@apo
    \or
        \edef\x{\SOUL@word={\the\SOUL@word'}}\x
    \or
        \edef\x{\SOUL@word={\the\SOUL@word{{''}}}}\x
    \fi
    \SOUL@apo\z@
}
\def\SOUL@flushgrave{%
    \ifcase\SOUL@grave
    \or
        \edef\x{\SOUL@word={\the\SOUL@word`}}\x
    \or
        \edef\x{\SOUL@word={\the\SOUL@word{{``}}}}\x
    \fi
    \SOUL@grave\z@
}
%    \end{macrocode}
% \end{macro}
% \end{macro}
% \end{macro}
% \end{macro}
%
%
%
%
%^^A--------------------------------------------------------------------
%
%
%
%
% \begin{macro}{\SOUL@dotoken}
% Command sequences from the \cs{\SOUL@cmds} list are handed over
% to \cs{\SOUL@docmd}, everything else is added to \cs{\SOUL@word},
% which will be fed to the analyzer every time a word is completed.
% Since \emph{robust} commands come with an additional space, we
% have also to examine if there's a space variant. Otherwise we
% couldn't detect pre-expanded formerly robust commands.
%
%    \begin{macrocode}
\def\SOUL@dotoken#1{%
    \def\SOUL@@{\SOUL@addtoken{#1}}%
    \def\\##1##2{%
        \edef\SOUL@x{\string#1}%
        \edef\SOUL@n{\string##2}%
        \ifx\SOUL@x\SOUL@n
            \def\SOUL@@{\SOUL@docmd{##1}{#1}}%
        \else
            \edef\SOUL@n{\string##2\space}%
            \ifx\SOUL@x\SOUL@n
                \def\SOUL@@{\SOUL@docmd{##1}{#1}}%
            \fi
        \fi
    }%
    \the\SOUL@cmds
    \SOUL@@
}
%    \end{macrocode}
% \end{macro}
%
%
%
%
%^^A--------------------------------------------------------------------
%
%
%
%
% \begin{macro}{\SOUL@docmd}
% Here we deal with commands that were registered with \cs{\soulregister}
% or \cs{\soulaccent} or were already predefined in \cs{\SOUL@cmds}.
% Commands with identifier |9| are accents that are put in a
% group with their argument. Identifier |8| is reserved for the \cs{\footnote}
% command, and |7| for the \cs{\textsuperscript} or similar commands.
% The others are mostly (but not necessarily)
% font switching commands, which may (|1|) or may not (|0|) take
% an argument. A registered command leads to the current word
% buffer being flushed to the analyzer, after which the command
% itself is executed.
%
% Font switching commands which take an argument need special
% treatment: They need to increment the level counter, so that
% \cs{\SOUL@eval} knows where to stop scanning. Furthermore the
% scanner has to be enabled to see the next token after the opening
% brace.
%
%
%    \begin{macrocode}
\def\SOUL@docmd#1#2{%
    \ifx9#1%
        \def\SOUL@@{\SOUL@addgroup{#2}}%
    \else\ifx8#1%
        \SOUL@doword
        \def\SOUL@@##1{%
            \SOUL@token={\footnotemark}%
            \SOUL@everytoken
            \SOUL@syllable={\footnotemark}%
            \SOUL@everysyllable
            \footnotetext{##1}%
            \SOUL@doword
            \SOUL@scan
        }%
    \else\ifx7#1%
        \SOUL@doword
        \def\SOUL@@##1{%
            \SOUL@token={#2{##1}}%
            \SOUL@everytoken
            \SOUL@syllable={#2{##1}}%
            \SOUL@everysyllable
            \SOUL@doword
            \SOUL@scan
        }%
    \else\ifx1#1%
        \SOUL@doword
        \def\SOUL@@##1{%
            #2{\protect\SOUL@do{##1}}%
            \SOUL@scan
        }%
    \else
        \SOUL@doword
        #2%
        \let\SOUL@@\SOUL@scan
    \fi\fi\fi\fi
    \SOUL@@
}
%    \end{macrocode}
% \end{macro}
%
%
%
%
%^^A--------------------------------------------------------------------
%
%
%
%
% \begin{macro}{\SOUL@addgroup}
% \begin{macro}{\SOUL@addmath}
% \begin{macro}{\SOUL@addprotect}
% \begin{macro}{\SOUL@addtoken}
% The macro names say it all. Each of these macros adds some
% token to the word buffer \cs{\SOUL@word}. Setting \cs{\protect}
% is necessary to make things like |\so{{a\itshape b}}| work.
%
%    \begin{macrocode}
\def\SOUL@addgroup#1#2{%
    {\let\protect\noexpand
    \edef\x{\global\SOUL@word={\the\SOUL@word{{\noexpand#1#2}}}}\x}%
    \SOUL@scan
}
\def\SOUL@addmath$#1${%
    {\let\protect\noexpand
    \edef\x{\global\SOUL@word={\the\SOUL@word{{\hbox{$#1$}}}}}\x}%
    \SOUL@scan
}
\def\SOUL@addprotect#1#2{%
    {\let\protect\noexpand
    \edef\x{\global\SOUL@word={\the\SOUL@word{{\hbox{#2}}}}}\x}%
    \SOUL@scan
}
\def\SOUL@addtoken#1{%
    \edef\x{\SOUL@word={\the\SOUL@word\noexpand#1}}\x
    \SOUL@scan
}
%    \end{macrocode}
% \end{macro}
% \end{macro}
% \end{macro}
% \end{macro}
%
%
%
%
%^^A--------------------------------------------------------------------
%
%
%
%
% \begin{macro}{\SOUL@exhyphen}
% Dealing with explicit hyphens can't be done before we know the
% following character, because we need to know if a kerning value
% has to be inserted, hence we delay the \cs{\SOUL@everyexhyphen} call.
% Unfortunately, the word scanner has no look-ahead mechanism.
%
%    \begin{macrocode}
\def\SOUL@exhyphen#1{%
    \SOUL@getkern{\the\SOUL@lasttoken}{\SOUL@hyphkern}{#1}%
    \gdef\SOUL@eventuallyexhyphen##1{%
        \SOUL@getkern{#1}{\SOUL@charkern}{##1}%
        \SOUL@everyexhyphen{#1}%
        \gdef\SOUL@eventuallyexhyphen####1{}%
    }%
}
%    \end{macrocode}
% \end{macro}
%
%
%
%
%^^A--------------------------------------------------------------------
%
%
%
%
% \begin{macro}{\SOUL@cmds}
% Here is a list of pre-registered commands that the analyzer
% cannot handle, so the scanner has to look after them. Every
% entry consists of a handle (\cs{\\}), an identifier and the
% macro name. The class identifier can be |9| for accents,
% |8| for the \cs{\footnote} command, |7| for the
% \cs{\textsuperscript} command,
% |0| for commands without arguments and |1| for commands that
% take one argument. Commands with two or more arguments are
% not supported.
%
%    \begin{macrocode}
\SOUL@cmds={%
    \\9\`\\9\'\\9\^\\9\"\\9\~\\9\=\\9\.%
    \\9\u\\9\v\\9\H\\9\t\\9\c\\9\d\\9\b\\9\r
    \\1\emph\\1\textrm\\1\textsf\\1\texttt\\1\textmd\\1\textbf
    \\1\textup\\1\textsl\\1\textit\\1\textsc\\1\textnormal
    \\0\rmfamily\\0\sffamily\\0\ttfamily\\0\mdseries\\0\upshape
    \\0\slshape\\0\itshape\\0\scshape\\0\normalfont
    \\0\em\\0\rm\\0\bf\\0\it\\0\tt\\0\sc\\0\sl\\0\sf
    \\0\tiny\\0\scriptsize\\0\footnotesize\\0\small
    \\0\normalsize\\0\large\\0\Large\\0\LARGE\\0\huge\\0\Huge
    \\1\MakeUppercase\\7\textsuperscript\\8\footnote
    \\1\textfrak\\1\textswab\\1\textgoth
    \\0\frakfamily\\0\swabfamily\\0\gothfamily
}
%    \end{macrocode}
% \end{macro}
%
%
%
%
%^^A--------------------------------------------------------------------
%
%
%
%
% \begin{macro}{\soulregister}
% \begin{macro}{\soulfont}
% \begin{macro}{\soulaccent}
% Register a font switching command (or some other command) for the
% scanner. The first argument is the macro name, the second is
% the number of arguments (|0|~or~|1|). Example: |\soulregister{\bold}{0}|.
% \cs{\soulaccent} has only one argument---the accent macro name.
% Example: |\soulaccent{\~}|. It is a shortcut for |\soulregister{\~}{9}|.
% The \cs{\soulfont} command is a synonym for \cs{\soulregister}
% and is kept for compatibility reasons.
%
%    \begin{macrocode}
\def\soulregister#1#2{{%
    \edef\x{\global\SOUL@cmds={\the\SOUL@cmds
        \noexpand\\#2\noexpand#1}}\x
}}
\def\soulaccent#1{\soulregister{#1}9}
\let\soulfont\soulregister
%    \end{macrocode}
% \end{macro}
% \end{macro}
% \end{macro}
%
%
%
%
%^^A--------------------------------------------------------------------
%
%
%
%
% \subsection{The analyzer}
%
% \begin{macro}{\SOUL@doword}
% The only way to find out, where a given word can be broken into
% syllables, is to let \TeX\ actually typeset the word under conditions
% that enforce every possible hyphenation. The result is a paragraph with one
% line for every syllable.
%
%    \begin{macrocode}
\def\SOUL@doword{%
    \edef\x{\the\SOUL@word}%
    \ifx\x\empty
    \else
        \SOUL@buffer={}%
        \setbox\z@\vbox{%
            \SOUL@tt
            \hyphenchar\font`\-
            \hfuzz\maxdimen
            \hbadness\@M
            \pretolerance\m@ne
            \tolerance\@M
            \leftskip\z@
            \rightskip\z@
            \hsize1sp
            \everypar{}%
            \parfillskip\z@\@plus1fil
            \hyphenpenalty-\@M
            \noindent
            \hskip\z@
            \relax
            \the\SOUL@word}%
        \let\SOUL@errmsg\SOUL@error
        \let\-\relax
        \count@\m@ne
        \SOUL@analyze
        \SOUL@word={}%
    \fi
}
%    \end{macrocode}
% \end{macro}
%
%
%
%
%^^A--------------------------------------------------------------------
%
%
%
%
% \noindent
% We store the hyphen width of the |ectt1000| font,
% because we will need it in |\SOUL@doword|. (|ectt1000| is a mono-spaced
% font, so every other character would have worked, too.)
%
%    \begin{macrocode}
\setbox\z@\hbox{\SOUL@tt-}
\newdimen\SOUL@ttwidth
\SOUL@ttwidth\wd\z@
\def\SOUL@sethyphenchar{%
    \ifnum\hyphenchar\font=\m@ne
    \else
        \char\hyphenchar\font
    \fi
}
%    \end{macrocode}
%
%
%
%
%^^A--------------------------------------------------------------------
%
%
%
%
% \begin{macro}{\SOUL@analyze}
% This macro decomposes the box that |\SOUL@doword| has built.
% Because we have to start at the bottom, we put every
% syllable onto the stack and execute ourselves recursively. If there
% are no syllables left, we return from the recursion and pick syllable
% after syllable from the stack again---this time from top to bottom---and
% hand the syllable width |\SOUL@syllgoal| over to |\SOUL@dosyllable|.
% All but the last syllable end with the hyphen character, hence
% we subtract the hyphen width accordingly. After processing a
% syllable we calculate the hyphen kern (i.\,e.~the kerning amount
% between the last character and the hyphen). This might be needed
% by \cs{\SOUL@everyhyphen}, which we call now.
%
%    \begin{macrocode}
\def\SOUL@analyze{{%
    \setbox\z@\vbox{%
        \unvcopy\z@
        \unskip
        \unpenalty
        \global\setbox\@ne=\lastbox}%
    \ifvoid\@ne
    \else
        \setbox\@ne\hbox{\unhbox\@ne}%
        \SOUL@syllgoal=\wd\@ne
        \advance\count@\@ne
        \SOUL@analyze
        \SOUL@syllwidth\z@
        \SOUL@syllable={}%
        \ifnum\count@>\z@
            \advance\SOUL@syllgoal-\SOUL@ttwidth
            \SOUL@dosyllable
            \SOUL@getkern{\the\SOUL@lasttoken}{\SOUL@hyphkern}%
                {\SOUL@sethyphenchar}%
            \SOUL@everyhyphen
        \else
            \SOUL@dosyllable
        \fi
    \fi
}}
%    \end{macrocode}
% \end{macro}
%
%
%
%
%^^A--------------------------------------------------------------------
%
%
%
%
% \begin{macro}{\SOUL@dosyllable}
% This macro typesets token after token from \cs{\SOUL@word}
% until \cs{\SOUL@syllwidth} has reached
% the requested width \cs{\SOUL@syllgoal}. Furthermore the kerning
% values are prepared in case \cs{\SOUL@everytoken} needs them.
% The \cs{\<} command used by \cs{\so} and \cs{\caps} needs some
% special treatment: It has to be checked for, even before
% we can end a syllable.
%
%    \begin{macrocode}
\def\SOUL@dosyllable{%
    \SOUL@gettoken
    \SOUL@eventuallyexhyphen{\the\SOUL@token}%
    \edef\x{\the\SOUL@token}%
    \ifx\x\SOUL@hyphenhintM
        \let\SOUL@n\SOUL@dosyllable
    \else\ifx\x\SOUL@lowerthanM
        \SOUL@gettoken
        \SOUL@getkern{\the\SOUL@lasttoken}{\SOUL@charkern}
            {\the\SOUL@token}%
        \SOUL@everylowerthan
        \SOUL@puttoken
        \let\SOUL@n\SOUL@dosyllable
    \else\ifdim\SOUL@syllwidth=\SOUL@syllgoal
        \SOUL@everysyllable
        \SOUL@puttoken
        \let\SOUL@n\relax
    \else\ifx\x\SOUL@stopM
        \SOUL@errmsg
        \global\let\SOUL@errmsg\relax
        \let\SOUL@n\relax
    \else
        \setbox\tw@\hbox{\SOUL@tt\the\SOUL@token}%
        \advance\SOUL@syllwidth\wd\tw@
        \global\SOUL@lasttoken=\SOUL@token
        \SOUL@gettoken
        \SOUL@getkern{\the\SOUL@lasttoken}{\SOUL@charkern}
            {\the\SOUL@token}%
        \SOUL@puttoken
        \global\SOUL@token=\SOUL@lasttoken
        \SOUL@everytoken
        \edef\x{\SOUL@syllable={\the\SOUL@syllable\the\SOUL@token}}\x
        \let\SOUL@n\SOUL@dosyllable
    \fi\fi\fi\fi
    \SOUL@n
}
%    \end{macrocode}
% \end{macro}
%
%
%
%
%^^A--------------------------------------------------------------------
%
%
%
%
% \begin{macro}{\SOUL@gettoken}
% Provide the next token in \cs{\SOUL@token}. If there's already one
% in the buffer, use that one first.
%
%    \begin{macrocode}
\def\SOUL@gettoken{%
    \edef\x{\the\SOUL@buffer}%
    \ifx\x\empty
        \SOUL@nexttoken
    \else
        \global\SOUL@token=\SOUL@buffer
        \global\SOUL@buffer={}%
    \fi
}
%    \end{macrocode}
% \end{macro}
%
%
%
%
%^^A--------------------------------------------------------------------
%
%
%
%
% \begin{macro}{\SOUL@puttoken}
% The possibility to put tokens back makes the scanner design much
% cleaner. There's only room for one token, though, so we issue
% an error message if \cs{\SOUL@puttoken} is told to put a token
% back while the buffer is still in use. Note that \cs{\SOUL@debug}
% is actually undefined. This won't hurt as it can only happen
% during driver design. No user will ever see this message.
%
%    \begin{macrocode}
\def\SOUL@puttoken{%
    \edef\x{\the\SOUL@buffer}%
    \ifx\x\empty
        \global\SOUL@buffer=\SOUL@token
        \global\SOUL@token={}%
    \else
        \SOUL@debug{puttoken called twice}%
    \fi
}
%    \end{macrocode}
% \end{macro}
%
%
%
%
%^^A--------------------------------------------------------------------
%
%
%
%
% \begin{macro}{\SOUL@nexttoken}
% \begin{macro}{\SOUL@splittoken}
% If the word buffer \cs{\SOUL@word} is empty, deliver a \cs{\SOUL@stop},
% otherwise take the next token.
%
%    \begin{macrocode}
\def\SOUL@nexttoken{%
    \edef\x{\the\SOUL@word}%
    \ifx\x\empty
        \SOUL@token={\SOUL@stop}%
    \else
        \expandafter\SOUL@splittoken\the\SOUL@word\SOUL@stop
    \fi
}
\def\SOUL@splittoken#1#2\SOUL@stop{%
    \global\SOUL@token={#1}%
    \global\SOUL@word={#2}%
}
%    \end{macrocode}
% \end{macro}
% \end{macro}
%
%
%
%
%^^A--------------------------------------------------------------------
%
%
%
%
% \begin{macro}{\SOUL@getkern}
% Assign the kerning value between the first and the third argument
% to the second, which has to be a \cs{\dimen} register.
% |\SOUL@getkern{A}{\dimen0}{V}| will assign the kerning value
% between `A' and `V' to |\dimen0|.
%
%    \begin{macrocode}
\def\SOUL@getkern#1#2#3{%
    \setbox\tw@\hbox{#1#3}%
    #2\wd\tw@
    \setbox\tw@\hbox{#1\null#3}%
    \advance#2-\wd\tw@
}
%    \end{macrocode}
% \end{macro}
%
%
%
%
%^^A--------------------------------------------------------------------
%
%
%
%
% \begin{macro}{\SOUL@setkern}
% Set a kerning value if it doesn't equal 0\,pt. Of course, we could
% also set a zero value, but that would needlessly clutter the
% logfile.
%
%    \begin{macrocode}
\def\SOUL@setkern#1{\ifdim#1=\z@\else\kern#1\fi}
%    \end{macrocode}
% \end{macro}
%
%
%
%
%^^A--------------------------------------------------------------------
%
%
%
%
% \begin{macro}{\SOUL@error}
% This error message will be shown once for every word that couldn't
% be reconstructed by \cs{\SOUL@dosyllable}.
%
%    \begin{macrocode}
\def\SOUL@error{%
    \vrule\@height.8em\@depth.2em\@width1em
    \PackageError{soul}{Reconstruction failed}{%
        I came across hyphenatable material enclosed in group
        braces,^^Jwhich I can't handle. Either drop the braces or
        make the material^^Junbreakable using an \string\mbox\space
        (\string\hbox). Note that a space^^Jalso counts as possible
        hyphenation point. See page 4 of the manual.^^JI'm leaving
        a black square so that you can see where I am right now.%
    }%
}
%    \end{macrocode}
% \end{macro}
%
%
%
%
%^^A--------------------------------------------------------------------
%
%
%
%
% \begin{macro}{\SOUL@setup}
% This is a null driver, that will be used as the basis for
% other drivers. These have then to redefine only interface commands
% that shall differ from the default.
%
%    \begin{macrocode}
\def\SOUL@setup{%
    \let\SOUL@preamble\relax
    \let\SOUL@postamble\relax
    \let\SOUL@everytoken\relax
    \let\SOUL@everysyllable\relax
    \def\SOUL@everyspace##1{##1\space}%
    \let\SOUL@everyhyphen\relax
    \def\SOUL@everyexhyphen##1{##1}%
    \let\SOUL@everylowerthan\relax
}
\SOUL@setup
%    \end{macrocode}
% \end{macro}
%
%
%
%
%^^A--------------------------------------------------------------------
%
%
%
%
% \subsection{The \texorpdfstring{\so{letterspacing}}{letterspacing} driver}
%
% \begin{macro}{\SOUL@sosetletterskip}
% A handy helper macro that sets the inter-letter skip with a
% draconian \cs{\penalty}.
%
%    \begin{macrocode}
\def\SOUL@sosetletterskip{\nobreak\hskip\SOUL@soletterskip}
%    \end{macrocode}
% \end{macro}
%
%
%
%
%^^A--------------------------------------------------------------------
%
%
%
%
% \begin{macro}{\SOUL@sopreamble}
% If letterspacing (\cs{\so} or \cs{\caps}) follows a white space, we
% replace it with our \syn{outer space}. \LaTeX\ uses |\hskip1sp| as
% marker in tabular entries, so we ignore tiny skips.
%
%    \begin{macrocode}
\def\SOUL@sopreamble{%
    \ifdim\lastskip>5sp
        \unskip
        \hskip\SOUL@soouterskip
    \fi
    \spaceskip\SOUL@soinnerskip
}
%    \end{macrocode}
% \end{macro}
%
%
%
%
%^^A--------------------------------------------------------------------
%
%
%
%
% \begin{macro}{\SOUL@sopostamble}
% Start the look-ahead scanner \cs{\SOUL@socheck} outside the \cs{\SOUL@}
% scope. That's why we make the \syn{outer space} globally available in
% \cs{\skip@}.
%
%    \begin{macrocode}
\def\SOUL@sopostamble{%
    \global\skip@=\SOUL@soouterskip
    \aftergroup\SOUL@socheck
}
%    \end{macrocode}
% \end{macro}
%
%
%
%
%^^A--------------------------------------------------------------------
%
%
%
%
% \begin{macro}{\SOUL@socheck}
% \begin{macro}{\SOUL@sodoouter}
% Read the next token after the \soul\ command into \cs{\SOUL@@}
% and examine it. If it's some kind of space, replace it with
% \syn{outer space} and the appropriate penalty, else if it's
% a closing brace, continue scanning. If it is neither: do nothing.
%
%    \begin{macrocode}
\def\SOUL@socheck{%
    \futurelet\SOUL@@\SOUL@sodoouter
}
\def\SOUL@sodoouter{%
    \def\SOUL@n*##1{\hskip\skip@}%
    \ifcat\egroup\noexpand\SOUL@@
        \unkern
        \egroup
        \def\SOUL@n*{\afterassignment\SOUL@socheck\let\SOUL@x=}%
    \else\ifx\SOUL@spc\SOUL@@
        \def\SOUL@n* {\hskip\skip@}%
    \else\ifx~\SOUL@@
        \def\SOUL@n*~{\nobreak\hskip\skip@}%
    \else\ifx\ \SOUL@@
    \else\ifx\space\SOUL@@
    \else\ifx\@xobeysp\SOUL@@
    \else
        \def\SOUL@n*{}%
        \let\SOUL@@\relax
    \fi\fi\fi\fi\fi\fi
    \SOUL@n*%
}
%    \end{macrocode}
% \end{macro}
% \end{macro}
%
%
%
%
%^^A--------------------------------------------------------------------
%
%
%
%
% \begin{macro}{\SOUL@soeverytoken}
% Typeset the token and put an unbreakable inter-letter skip
% thereafter. If the token is \cs{\<} then remove the last skip instead.
% Gets the character kerning value between the actual and the
% next token in \cs{\SOUL@charkern}.
%
%    \begin{macrocode}
\def\SOUL@soeverytoken{%
    \edef\x{\the\SOUL@token}%
    \ifx\x\SOUL@lowerthanM
    \else
        \global\let\SOUL@soeventuallyskip\SOUL@sosetletterskip
        \the\SOUL@token
        \SOUL@gettoken
        \edef\x{\the\SOUL@token}%
        \ifx\x\SOUL@stopM
        \else
            \SOUL@setkern\SOUL@charkern
            \SOUL@sosetletterskip
            \SOUL@puttoken
        \fi
    \fi
}
%    \end{macrocode}
% \end{macro}
%
%
%
%
%^^A--------------------------------------------------------------------
%
%
%
%
% \begin{macro}{\SOUL@soeveryspace}
% This macro sets an \syn{inner space}. The argument may contain
% penalties and is used for the |~| command. This construction was
% needed to make colored underlines work, without having to put
% any of the coloring commands into the core. |\kern\z@| prevents
% in subsequent \cs{\so} commands that the second discards the
% \syn{outer space} of the first. To remove the space simply
% use |\unkern\unskip|.
%
%    \begin{macrocode}
\def\SOUL@soeveryspace#1{#1\space\kern\z@}
%    \end{macrocode}
% \end{macro}
%
%
%
%
%^^A--------------------------------------------------------------------
%
%
%
%
% \begin{macro}{\SOUL@soeveryhyphen}
% Sets implicit hyphens. The kerning value between the current token
% and the hyphen character is passed in \cs{\SOUL@hyphkern}.
%
%    \begin{macrocode}
\def\SOUL@soeveryhyphen{%
    \discretionary{%
        \unkern
        \SOUL@setkern\SOUL@hyphkern
        \SOUL@sethyphenchar
    }{}{}%
}
%    \end{macrocode}
% \end{macro}
%
%
%
%
%^^A--------------------------------------------------------------------
%
%
%
%
% \begin{macro}{\SOUL@soeveryexhyphen}
% Sets the explicit hyphen that is passed as argument.
% \cs{\SOUL@soeventuallyskip} equals \cs{\SOUL@sosetletterskip},
% except when a \cs{\<} had been detected. This is necessary because
% \cs{\SOUL@soeveryexhyphen} wouldn't know otherwise, that it
% follows a~\cs{\<}.
%
%    \begin{macrocode}
\def\SOUL@soeveryexhyphen#1{%
    \SOUL@setkern\SOUL@hyphkern
    \SOUL@soeventuallyskip
    \hbox{#1}%
    \discretionary{}{}{%
        \SOUL@setkern\SOUL@charkern
    }%
    \SOUL@sosetletterskip
    \global\let\SOUL@soeventuallyskip\relax
}
%    \end{macrocode}
% \end{macro}
%
%
%
%
%^^A--------------------------------------------------------------------
%
%
%
%
% \begin{macro}{\SOUL@soeverylowerthan}
% Let \cs{\<} remove the last inter-letter skip. Set the kerning value
% between the token before and that after the \cs{\<} command.
%
%
%    \begin{macrocode}
\def\SOUL@soeverylowerthan{%
    \unskip
    \unpenalty
    \global\let\SOUL@soeventuallyskip\relax
    \SOUL@setkern\SOUL@charkern
}
%    \end{macrocode}
% \end{macro}
%
%
%
%
%^^A--------------------------------------------------------------------
%
%
%
%
% \begin{macro}{\SOUL@sosetup}
% Override all interface macros by our letterspacing versions. The
% only unused macro is \cs{\SOUL@everysyllable}.
%
%    \begin{macrocode}
\def\SOUL@sosetup{%
    \SOUL@setup
    \let\SOUL@preamble\SOUL@sopreamble
    \let\SOUL@postamble\SOUL@sopostamble
    \let\SOUL@everytoken\SOUL@soeverytoken
    \let\SOUL@everyspace\SOUL@soeveryspace
    \let\SOUL@everyhyphen\SOUL@soeveryhyphen
    \let\SOUL@everyexhyphen\SOUL@soeveryexhyphen
    \let\SOUL@everylowerthan\SOUL@soeverylowerthan
}
%    \end{macrocode}
% \end{macro}
%
%
%
%
%^^A--------------------------------------------------------------------
%
%
%
%
% \begin{macro}{\SOUL@setso}
% A handy macro for internal use.
%
%    \begin{macrocode}
\def\SOUL@setso#1#2#3{%
    \def\SOUL@soletterskip{#1}%
    \def\SOUL@soinnerskip{#2}%
    \def\SOUL@soouterskip{#3}%
}
%    \end{macrocode}
% \end{macro}
%
%
%
%
%^^A--------------------------------------------------------------------
%
%
%
%
% \begin{macro}{\sodef}
% This macro assigns the letterspacing skips as well as an optional
% font switching command to a command sequence name. \cs{\so} itself
% will be defined using this macro.
%
%    \begin{macrocode}
\def\sodef#1#2#3#4#5{%
    \DeclareRobustCommand*#1{\SOUL@sosetup
        \def\SOUL@preamble{%
            \SOUL@setso{#3}{#4}{#5}%
            #2%
            \SOUL@sopreamble
        }%
        \SOUL@
    }%
}
%    \end{macrocode}
% \end{macro}
%
%
%
%
%^^A--------------------------------------------------------------------
%
%
%
%
% \begin{macro}{\resetso}
% Let \cs{\resetso} define reasonable default values for letterspacing.
%
%    \begin{macrocode}
\def\resetso{%
    \sodef\textso{}{.25em}{.65em\@plus.08em\@minus.06em}%
        {.55em\@plus.275em\@minus.183em}%
}
\resetso
%    \end{macrocode}
% \end{macro}
%
%
%
%
%^^A--------------------------------------------------------------------
%
%
%
%
% \begin{macro}{\sloppyword}
% Set up a letterspacing macro that inserts slightly stretchable
% space between the characters. This can be used to typeset long
% words in narrow columns, where ragged paragraphs are undesirable.
% See section~\ref{sec:sloppyword}.
%
%    \begin{macrocode}
\sodef\sloppyword{%
    \linepenalty10
    \hyphenpenalty10
    \adjdemerits\z@
    \doublehyphendemerits\z@
    \finalhyphendemerits\z@
    \emergencystretch.1em}%
    {\z@\@plus.1em}%
    {.33em\@plus.11em\@minus.11em}%
    {.33em\@plus.11em\@minus.11em}
%    \end{macrocode}
% \end{macro}
%
%
%
%
%^^A--------------------------------------------------------------------
%
%
%
%
% \subsection[The \texorpdfstring{\caps{caps}}{caps} driver]{The caps driver}
%
% \begin{macro}{\caps}
% Unless run under \LaTeX, make \cs{\caps} just another simple letterspacing
% macro that selects a font \cs{\capsfont} (defaulting to \cs{\relax}) but
% doesn't have any special capabilities.
%
%    \begin{macrocode}
\ifx\documentclass\@undefined
\let\capsfont\relax
\let\capsreset\relax
\def\capsdef#1#2#3#4#5{}
\def\capssave#1{}
\def\capsselect#1{}
\sodef\textcaps{\capsfont}
    {.028em\@plus.005em\@minus.01em}%
    {.37em\@plus.1667em\@minus.111em}%
    {.37em\@plus.1em\@minus.14em}
%    \end{macrocode}
% \end{macro}
%
%
%
%
%^^A--------------------------------------------------------------------
%
%
%
%
% \begin{macro}{\capsreset}
% \dots\ else, if run under \LaTeX\ prepare a set of macros that
% maintain a database with certain letterspacing values for different
% fonts. \cs{\capsreset} clears the database and inserts a default rule.
%
%    \begin{macrocode}
\else
\DeclareRobustCommand*\capsreset{%
    \let\SOUL@capsbase\empty
    \SOUL@capsdefault
}
%    \end{macrocode}
% \end{macro}
%
%
%
%
%^^A--------------------------------------------------------------------
%
%
%
%
% \begin{macro}{\capsdef}
% Add an entry to the database, which is of course nothing else than
% a \TeX\ macro. See section ``List macros'' of appendix~D in the
% \TeX{}book~\cite{DEK} for details.
%
%    \begin{macrocode}
\def\capsdef#1#2#3#4#5{{%
    \toks\z@{\\{#1/#2/#3/#4/#5}}%
    \toks\tw@=\expandafter{\SOUL@capsbase}%
    \xdef\SOUL@capsbase{\the\toks\z@\the\toks\tw@}%
}}
%    \end{macrocode}
% \end{macro}
%
%
%
%
%^^A--------------------------------------------------------------------
%
%
%
%
% \begin{macro}{\capssave}
% \begin{macro}{\capsselect}
% Save the current database in a macro within the |SOUL@| namespace
% and let |\capsselect| restore this database.
%
%    \begin{macrocode}
\DeclareRobustCommand*\capssave[1]{%
    \expandafter\global\expandafter\let
        \csname SOUL@db@#1\endcsname\SOUL@capsbase
}
\DeclareRobustCommand*\capsselect[1]{%
    \expandafter\let\expandafter\SOUL@capsbase
        \csname SOUL@db@#1\endcsname
}
%    \end{macrocode}
% \end{macro}
% \end{macro}
%
%
%
%
%^^A--------------------------------------------------------------------
%
%
%
%
% \begin{macro}{\SOUL@capsfind}
% \begin{macro}{\SOUL@caps}
% Go through the database entries and pick the first entry that matches
% the currently active font. Then define an internal macro that uses
% the respective spacing values in a macro that is equivalent to the
% \cs{\textso} command.
%
%    \begin{macrocode}
\def\SOUL@capsfind#1/#2/#3/#4/#5/#6/#7/#8/#9/{%
    \let\SOUL@match=1%
    \SOUL@chk{#1}\f@encoding
    \SOUL@chk{#2}\f@family
    \SOUL@chk{#3}\f@series
    \SOUL@chk{#4}\f@shape
    \SOUL@dimchk{#5}\f@size
    \if\SOUL@match1%
        \let\\\@gobble
        \gdef\SOUL@caps{%
            \SOUL@sosetup
            \def\SOUL@preamble{\SOUL@setso{#7}{#8}{#9}#6%
                \SOUL@sopreamble}%
            \SOUL@}%
    \fi
}
%    \end{macrocode}
% \end{macro}
% \end{macro}
%
%
%
%
%^^A--------------------------------------------------------------------
%
%
%
%
% \begin{macro}{\SOUL@chk}
% Sets the \cs{\SOUL@match} flag if both parameters are equal.
% This is used for all \caps{\small NFSS} elements except the font size.
%
%    \begin{macrocode}
\def\SOUL@chk#1#2{%
    \if$#1$%
    \else
        \def\SOUL@n{#1}%
        \ifx#2\SOUL@n\else\let\SOUL@match=0\fi
    \fi
}
%    \end{macrocode}
% \end{macro}
%
%
%
%
%^^A--------------------------------------------------------------------
%
%
%
%
% \begin{macro}{\SOUL@dimchk}
% \begin{macro}{\SOUL@rangechk}
% We do not only want to check if a given font size |#1| matches |#2|,
% but also if it fits into a given range. An omitted lower boundary
% is replaced by \cs{\z@} and an omitted upper boundary by \cs{\maxdimen}.
% The first of a series of \cs{\SOUL@chk} and \cs{\SOUL@dimchk}
% statements, which detects that the arguments don't match, sets the
% \cs{\SOUL@match} flag to zero. A value of~1 indicates that an
% entry in the font database matches the currently used font.
%
%    \begin{macrocode}
\def\SOUL@dimchk#1#2{\if$#1$\else\SOUL@rangechk{#2}#1--\@ne\@@\fi}
\def\SOUL@rangechk#1#2-#3-#4\@@{%
    \count@=#4%
    \ifnum\count@>\z@
        \ifdim#1\p@=#2\p@\else\let\SOUL@match=0\fi
    \else
        \dimen@=\if$#2$\z@\else#2\p@\fi
        \ifdim#1\p@<\dimen@\let\SOUL@match=0\fi
        \dimen@=\if$#3$\maxdimen\else#3\p@\fi
        \ifdim#1\p@<\dimen@\else\let\SOUL@match=0\fi
    \fi
}
%    \end{macrocode}
% \end{macro}
% \end{macro}
%
%
%
%
%^^A--------------------------------------------------------------------
%
%
%
%
% \begin{macro}{\textcaps}
% Find a matching entry in the database and start the letterspacing
% mechanism with the given spacing values.
%
%    \begin{macrocode}
\DeclareRobustCommand*\textcaps{{%
    \def\\##1{\expandafter\SOUL@capsfind##1/}%
    \SOUL@capsbase
    \aftergroup\SOUL@caps
}}
%    \end{macrocode}
% \end{macro}
%
%
%
%
%^^A--------------------------------------------------------------------
%
%
%
%
% \begin{macro}{\SOUL@capsdefault}
% Define a default database entry and a default font.
%
%    \begin{macrocode}
\def\SOUL@capsdefault{%
    \capsdef{////}%
    \SOUL@capsdfltfnt
    {.028em\@plus.005em\@minus.01em}%
    {.37em\@plus.1667em\@minus.1em}%
    {.37em\@plus.111em\@minus.14em}%
}
\let\SOUL@capsdfltfnt\scshape
\capsreset
\fi
%    \end{macrocode}
% \end{macro}
%
%
%
%
%^^A--------------------------------------------------------------------
%
%
%
%
% \subsection{The \texorpdfstring{\ul{underlining}}{underlining} driver}
%
% \begin{macro}{\SOUL@ulleaders}
% This macro sets the underline under the following \cs{\hskip}.
%
%    \begin{macrocode}
\newdimen\SOUL@uldp
\newdimen\SOUL@ulht
\def\SOUL@ulleaders{%
    \leaders\hrule\@depth\SOUL@uldp\@height\SOUL@ulht\relax
}
%    \end{macrocode}
% \end{macro}
%
%
%
%
%^^A--------------------------------------------------------------------
%
%
%
%
% \begin{macro}{\SOUL@ulunderline}
% Set an underline under the given material. It draws the line first,
% and the given material afterwards. This is needed for highlighting,
% but gives less than optimal results for colored overstriking, which,
% however, will hardly ever be used, anyway.
%
%    \begin{macrocode}
\def\SOUL@ulunderline#1{{%
    \setbox\z@\hbox{#1}%
    \dimen@=\wd\z@
    \dimen@i=\SOUL@uloverlap
    \advance\dimen@2\dimen@i
    \rlap{%
        \null
        \kern-\dimen@i
        \SOUL@ulcolor{\SOUL@ulleaders\hskip\dimen@}%
    }%
    \unhcopy\z@
}}
%    \end{macrocode}
% \end{macro}
%
%
%
%
%^^A--------------------------------------------------------------------
%
%
%
%
% \begin{macro}{\SOUL@ulpreamble}
% Just set up the line dimensions and the space skip. Normally,
% \cs{\spaceskip} is unset and not used by \TeX. We need it, though,
% because we feed it to the \cs{\leaders} primitive.
%
%    \begin{macrocode}
\def\SOUL@ulpreamble{%
    \SOUL@uldp=\SOUL@uldepth
    \SOUL@ulht=-\SOUL@uldp
    \advance\SOUL@uldp\SOUL@ulthickness
    \spaceskip\SOUL@spaceskip
}
%    \end{macrocode}
% \end{macro}
%
%
%
%
%^^A--------------------------------------------------------------------
%
%
%
%
% \begin{macro}{\SOUL@uleverysyllable}
% By using \cs{\SOUL@everysyllable} we don't have to care about
% kerning values and get better results for highlighting, where
% negative kerning values would otherwise cut off characters.
%
%    \begin{macrocode}
\def\SOUL@uleverysyllable{%
    \SOUL@ulunderline{%
        \the\SOUL@syllable
        \SOUL@setkern\SOUL@charkern
    }%
}
%    \end{macrocode}
% \end{macro}
%
%
%
%
%^^A--------------------------------------------------------------------
%
%
%
%
% \begin{macro}{\SOUL@uleveryspace}
% Set a given penalty and an underlined \cs{\space} equivalent.
% The \cs{\null} prevents a nasty gap in |\textfrak| |{a \textswab{b}}|,
% while it doesn't seem to hurt in all other cases. I didn't investigate
% this.
%
%    \begin{macrocode}
\def\SOUL@uleveryspace#1{%
    \SOUL@ulcolor{%
        #1%
        \SOUL@ulleaders
        \hskip\spaceskip
    }%
    \null
}
%    \end{macrocode}
% \end{macro}
%
%
%
%
%^^A--------------------------------------------------------------------
%
%
%
%
% \begin{macro}{\SOUL@uleveryhyphen}
% If hyphenation takes place, output an underlined hyphen with the
% required hyphen kerning value.
%
%    \begin{macrocode}
\def\SOUL@uleveryhyphen{%
    \discretionary{%
        \unkern
        \SOUL@ulunderline{%
            \SOUL@setkern\SOUL@hyphkern
            \SOUL@sethyphenchar
        }%
    }{}{}%
}
%    \end{macrocode}
% \end{macro}
%
%
%
%
%^^A--------------------------------------------------------------------
%
%
%
%
% \begin{macro}{\SOUL@uleveryexhyphen}
% Underline the given hyphen, en-dash, em-dash or \cs{\slash} and care
% for kerning.
%
%    \begin{macrocode}
\def\SOUL@uleveryexhyphen#1{%
    \SOUL@setkern\SOUL@hyphkern
    \SOUL@ulunderline{#1}%
    \discretionary{}{}{%
        \SOUL@setkern\SOUL@charkern
    }%
}
%    \end{macrocode}
% \end{macro}
%
%
%
%
%^^A--------------------------------------------------------------------
%
%
%
%
% \begin{macro}{\SOUL@ulcolor}
% \begin{macro}{\setulcolor}
% Define the underline color or turn off coloring, in which case the lines are not
% just colored black, but remain uncolored. This makes them appear
% black, nevertheless, and has the advantage, that no Postscript
% \cs{\specials} are cluttering the output.
%
%    \begin{macrocode}
\let\SOUL@ulcolor\relax
\def\setulcolor#1{%
    \if$#1$
        \let\SOUL@ulcolor\relax
    \else
        \def\SOUL@ulcolor{\textcolor{#1}}%
    \fi
}
%    \end{macrocode}
% \end{macro}
% \end{macro}
%
%
%
%
%^^A--------------------------------------------------------------------
%
%
%
%
% \begin{macro}{\setuloverlap}
% \begin{macro}{\SOUL@uloverlap}
% Set the overlap amount, that helps to avoid gaps on sloppy output
% devices.
%
%    \begin{macrocode}
\def\setuloverlap#1{\def\SOUL@uloverlap{#1}}
\setuloverlap{.25pt}
%    \end{macrocode}
% \end{macro}
% \end{macro}
%
%
%
%
%^^A--------------------------------------------------------------------
%
%
%
%
% \begin{macro}{\SOUL@ulsetup}
% The underlining driver is quite simple. No need for \cs{\SOUL@postamble}
% and \cs{\SOUL@everytoken}.
%
%    \begin{macrocode}
\def\SOUL@ulsetup{%
    \SOUL@setup
    \let\SOUL@preamble\SOUL@ulpreamble
    \let\SOUL@everysyllable\SOUL@uleverysyllable
    \let\SOUL@everyspace\SOUL@uleveryspace
    \let\SOUL@everyhyphen\SOUL@uleveryhyphen
    \let\SOUL@everyexhyphen\SOUL@uleveryexhyphen
}
%    \end{macrocode}
% \end{macro}
%
%
%
%
%^^A--------------------------------------------------------------------
%
%
%
%
% \begin{macro}{\SOUL@textul}
% Describing self-explanatory macros is \emph{so} boring!
%
%    \begin{macrocode}
\DeclareRobustCommand*\textul{\SOUL@ulsetup\SOUL@}
%    \end{macrocode}
% \end{macro}
%
%
%
%
%^^A--------------------------------------------------------------------
%
%
%
%
% \begin{macro}{\setul}
% \begin{macro}{\SOUL@uldepth}
% \begin{macro}{\SOUL@ulthickness}
% Set the underlining dimensions. Either value may be omitted and
% lets the respective macro keep its current contents.
%
%    \begin{macrocode}
\def\setul#1#2{%
    \if$#1$\else\def\SOUL@uldepth{#1}\fi
    \if$#2$\else\def\SOUL@ulthickness{#2}\fi
}
%    \end{macrocode}
% \end{macro}
% \end{macro}
% \end{macro}
%
%
%
%
%^^A--------------------------------------------------------------------
%
%
%
%
% \begin{macro}{\resetul}
% Set reasonable default values that fit most latin fonts.
%
%    \begin{macrocode}
\def\resetul{\setul{.65ex}{.1ex}}
\resetul
%    \end{macrocode}
% \end{macro}
%
%
%
%
%^^A--------------------------------------------------------------------
%
%
%
%
% \begin{macro}{\setuldepth}
% This macro sets all designated ``letters'' (\cs{\catcode=11}) or the
% given material in a box and sets the underlining dimensions according
% to the box depth.
%
%    \begin{macrocode}
\def\setuldepth#1{{%
    \def\SOUL@n{#1}%
    \setbox\z@\hbox{%
        \tracinglostchars\z@
        \ifx\SOUL@n\empty
            \count@\z@
            \loop
                \ifnum\catcode\count@=11\char\count@\fi
            \ifnum\count@<\@cclv
                \advance\count@\@ne
            \repeat
        \else
            #1%
        \fi
    }%
    \dimen@\dp\z@
    \advance\dimen@\p@
    \xdef\SOUL@uldepth{\the\dimen@}%
}}
%    \end{macrocode}
% \end{macro}
%
%
%
%
%^^A--------------------------------------------------------------------
%
%
%
%
% \subsection{The \texorpdfstring{\st{overstriking}}{overstriking} driver}
%
% \begin{macro}{\SOUL@stpreamble}
% Striking out is just underlining with a raised line of a different
% color. Hence we only need to define the preamble accordingly and
% let the underlining preamble finally do its job. Not that colored
% overstriking was especially useful, but we want at least to keep
% it black while we might want to set underlines in some fancy color.
%
%    \begin{macrocode}
\def\SOUL@stpreamble{%
    \dimen@\SOUL@ulthickness
    \dimen@i=-.5ex
    \advance\dimen@i-.5\dimen@
    \edef\SOUL@uldepth{\the\dimen@i}%
    \let\SOUL@ulcolor\SOUL@stcolor
    \SOUL@ulpreamble
}
%    \end{macrocode}
% \end{macro}
%
%
%
%
%^^A--------------------------------------------------------------------
%
%
%
%
% \begin{macro}{\SOUL@stsetup}
% We re-use the whole underlining setup and just replace the preamble
% with our modified version.
%
%    \begin{macrocode}
\def\SOUL@stsetup{%
    \SOUL@ulsetup
    \let\SOUL@preamble\SOUL@stpreamble
}
%    \end{macrocode}
% \end{macro}
%
%
%
%
%^^A--------------------------------------------------------------------
%
%
%
%
% \begin{macro}{\textst}
% These pretzels are making me thirsty \dots
%
%    \begin{macrocode}
\DeclareRobustCommand*\textst{\SOUL@stsetup\SOUL@}
%    \end{macrocode}
% \end{macro}
%
%
%
%
%^^A--------------------------------------------------------------------
%
%
%
%
% \begin{macro}{\SOUL@stcolor}
% \begin{macro}{\setstcolor}
% Set the overstriking color. This won't be used often, but is required
% in cases, where the underlines are colored. You wouldn't want to
% overstrike in the same color. Note that overstriking lines are
% drawn \emph{beneath} the text, hence bright colors won't look good.
%
%    \begin{macrocode}
\let\SOUL@stcolor\relax
\def\setstcolor#1{%
    \if$#1$
        \let\SOUL@stcolor\relax
    \else
        \def\SOUL@stcolor{\textcolor{#1}}%
    \fi
}
%    \end{macrocode}
% \end{macro}
% \end{macro}
%
%
%
%
%^^A--------------------------------------------------------------------
%
%
%
%
% \subsection{The highlighting driver}
%
% \begin{macro}{\SOUL@hlpreamble}
% This is nothing else than overstriking with very thick lines.
%
%    \begin{macrocode}
\def\SOUL@hlpreamble{%
    \setul{}{2.5ex}%
    \let\SOUL@stcolor\SOUL@hlcolor
    \SOUL@stpreamble
}
%    \end{macrocode}
% \end{macro}
%
%
%
%
%^^A--------------------------------------------------------------------
%
%
%
%
% \begin{macro}{\SOUL@hlsetup}
% No need to re-invent the wheel. Just use the overstriking setup
% with a different preamble.
%
%    \begin{macrocode}
\def\SOUL@hlsetup{%
    \SOUL@stsetup
    \let\SOUL@preamble\SOUL@hlpreamble
}
%    \end{macrocode}
% \end{macro}
%
%
%
%
%^^A--------------------------------------------------------------------
%
%
%
%
% \begin{macro}{\texthl}
% \begin{macro}{\sethlcolor}
% \begin{macro}{\SOUL@hlcolor}
% Define the highlighting macro and the color setting macro with a
% simple default color. Yellow isn't really pleasing, but it's already
% predefined by the \package{color} package.
%
%    \begin{macrocode}
\DeclareRobustCommand*\texthl{\SOUL@hlsetup\SOUL@}
\def\sethlcolor#1{\if$#1$\else\def\SOUL@hlcolor{\textcolor{#1}}\fi}
\sethlcolor{yellow}
%    \end{macrocode}
% \end{macro}
% \end{macro}
% \end{macro}
%
%
%
%
%^^A--------------------------------------------------------------------
%
%
%
%
% \subsection*{The package postamble}
%
% \begin{macro}{\so}
% \begin{macro}{\ul}
% \begin{macro}{\st}
% \begin{macro}{\hl}
% \begin{macro}{\caps}
% OK, I lied. The short macro names are just abbreviations for their
% longer counterpart. Some people might be used to |\text*| style commands
% to select a certain font. And then it doesn't hurt to reserve
% these early enough.
%
%    \begin{macrocode}
\let\so\textso
\let\ul\textul
\let\st\textst
\let\hl\texthl
\let\caps\textcaps
%    \end{macrocode}
% \end{macro}
% \end{macro}
% \end{macro}
% \end{macro}
% \end{macro}
%
%
%
%
%^^A--------------------------------------------------------------------
%
%
%
%
% \noindent
% When used in an environment other than \LaTeX\ and the \package{german}
% package was already loaded, define the double quotes as accent.
%
%    \begin{macrocode}
\ifx\documentclass\@undefined
    \ifx\mdqoff\@undefined
    \else
        \soulaccent{"}%
    \fi
    \catcode`\@=\atcode
%    \end{macrocode}
%
%
%
%
%^^A--------------------------------------------------------------------
%
%
%
%
% \noindent
% If we have been loaded by a \LaTeX\ environment and the \package{color}
% package wasn't also loaded, we disable all colors. Then we add the umlaut accent
% |"| if the \package{german} package is present. The quotes character has to
% be \cs{\catcode}'d \cs{\active} now, or it won't get recognized later.
% The \option{capsdefault}
% option overrides the \cs{\caps} driver and lets \cs{\SOUL@} set an underline.
% Finally load the local configuration, process the |capsdefault|
% option and exit.
%
%    \begin{macrocode}
\else
    \bgroup
        \catcode`\"\active
        \AtBeginDocument{%
            \@ifundefined{color}{%
                \let\SOUL@color\relax
                \let\setulcolor\@gobble
                \let\setstcolor\@gobble
                \let\sethlcolor\@gobble
                \let\hl\ul
            }{}
            \@ifundefined{mdqoff}{}{\soulaccent{"}}
        }
    \egroup
    \DeclareOption{capsdefault}{%
        \AtBeginDocument{%
            \def\SOUL@capsdfltfnt#1{%
                \SOUL@ulsetup
                \SOUL@ulpreamble
                \scshape
            }%
        }%
    }
    \InputIfFileExists{soul.cfg}%
        {\PackageInfo{soul}{Local config file soul.cfg used}}{}
    \ProcessOptions
\fi
\endinput
%    \end{macrocode}
%
%
%
%
%^^A--------------------------------------------------------------------
%
%
%
%
% \vspace{2explus1fill}
%\begin{verbatim}
%$Id: soul.dtx,v 1.128 2003/11/17 22:57:24 m Rel $
%\end{verbatim}
% \Finale
%
%
%                                                       ^^A.E.I.O.U.^^
%^^A vim:ts=4:sw=4:et
