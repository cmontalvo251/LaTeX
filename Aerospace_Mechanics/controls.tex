\section{Conventional Aerospace Controls}

\subsection{Spacecraft Attitude Control Schemes}

Many control schemes are needed to orient a satellite and all depend
on the application. In LEO magnetorquers can be used to detumble a
satellite while thrusters must be used in deep space. In addition
reaction wheels can be used to detumble a satellite anywhere in space
provided the angular momentum in the satellite does not saturate the
reaction wheels. Sections that follow detail the control schemes
typically utilized on small sats. A section on PID control for
aircraft is also in this section.

\subsubsection{B-dot Controller}

In LEO, the standard B-dot controller reported in many sources
(\cite{Leomanni2012},\cite{Lovera2015},\cite{WInstitute},\cite{SanyalDick})
can be used to de-tumble a satellite. The standard B-dot controller
requires the magnetorquers to follow the control law shown below
\begin{equation}\label{e:bdotcompact}
  \vec{\mu}_B = k{\bf S}(\vec{\omega}_{B/I}){\bf T}_{BI}(\vec{q})\vec{\beta}_I
\end{equation}
where $k$ is the control gain. Using equation (\ref{e:magmoment}) it
is possible to write the current in component form again
using the identity that $\vec{a}\times\vec{b}=-\vec{b}\times\vec{a}$
\begin{equation}\label{e:bdot}
  \begin{Bmatrix} i_{x}
    \\ i_{y} \\ i_{z} \end{Bmatrix} = \frac{k}{nA}\begin{bmatrix} 0 & \beta_z & -\beta_y \\ -\beta_z & 0 &
  \beta_x \\ \beta_y & -\beta_x & 0 \end{bmatrix}\begin{Bmatrix} p
    \\ q \\ r \end{Bmatrix}
\end{equation}
This equation can then be substituted into equation
(\ref{e:magtorquecomponent}) to produce the total torque on the
satellite assuming that the magnetorquers can provide the necessary
current commanded by equation (\ref{e:bdot}).
\begin{equation}\label{e:bdotfinal}
  \begin{Bmatrix} L \\ M \\ N \end{Bmatrix} = -k \begin{bmatrix}
    \beta_y^2 + \beta_z^2 & -\beta_x\beta_y & -\beta_x\beta_z
    \\ -\beta_x\beta_y & \beta_x^2 + \beta_z^2 & -\beta_y\beta_z \\
    -\beta_x\beta_z & -\beta_y\beta_z & \beta_x^2 +
    \beta_y^2 \end{bmatrix} \begin{Bmatrix} p
    \\ q \\ r \end{Bmatrix} 
\end{equation}
The goal of the controller here is to drive
$\vec{\omega}_{B/I}\rightarrow 0$. The literature will show that
this is not completely achieved \cite{Celani}. There are multiple
explanations for this. For starters, equation (\ref{e:magtorque})
assumes that the magnetic moment is not co-linear with the magnetic
field of the Earth. If it is, the result is zero torque applied to the
satellite. Furthermore, equation (\ref{e:bdot})
results in zero current if the angular velocity vector of the
satellite is co-linear with the magnetic field. Thus, if the magnetic
field vector, angular velocity vector or the magnetic moment vector
are co-linear, the torque applied to the satellite will be zero. If a
new operator is defined such that
\begin{equation}
  {\bf W}({\bf T}_{BI}(\vec{q})\vec{\beta}_I) = \begin{bmatrix}
    \beta_y^2 + \beta_z^2 & -\beta_x\beta_y & -\beta_x\beta_z
    \\ -\beta_x\beta_y & \beta_x^2 + \beta_z^2 & -\beta_y\beta_z \\
    -\beta_x\beta_z & -\beta_y\beta_z & \beta_x^2 +
    \beta_y^2 \end{bmatrix}
\end{equation}
it is easy to see that the torque applied to a satellite is then
simply the angular velocity vector multiplied by this transition
matrix. If this transition matrix is put into row-reduced-echelon form
it is easy to see that the determinant of this matrix is equal to
zero (\cite{carlen2006linear}).
\begin{equation}
  rref({\bf W}({\bf T}_{BI}(\vec{q})\vec{\beta}_I) = \begin{bmatrix} 1 & 0
    & -\beta_x/ \beta_z \\ 0 & 1 & -\beta_y/ \beta_z \\ 0 & 0 & 0 \end{bmatrix}
\end{equation}
A zero determinant means that there exists a vector
$\vec{\omega}_{B/I}$ that will result in zero torque for a given value
of the magnetic field. This is typically avoided since the
magnetic field of the Earth is time and spatially varying
which results in a transition matrix that changes over time due to
orientation changes in the satellite as well as changes in the satellite's
orbit. However, for low inclination orbits, it's possible for the
magnetic field to stay relatively constant with $\beta_x \approx
\beta_y \approx 0$. If the satellite is tumbling about the yaw axis such
that $p=q=0$, the yaw torque on the satellite ($N$) will be
zero. Using this simple controller, there is no way to
remove the remaining angular velocity from the satellite unless
reaction wheels are used.

\subsubsection{Reaction Wheel Control}

Assuming each reaction is aligned with a principal axis of inertia the
control scheme is extremely simple. When the wheels are not aligned
the derivation will proceed similar to the reaction control thruster
section. The derivation here will just be for the aligned case. In
this analysis it is assumed that a torque can be applied to the
reaction wheel and thus the angular velocity of the reaction wheel
$\alpha_{Ri}$ can be directly controlled. Assuming this a simple PD
control law can be used to orient the satellite at any desired
orientation using Euler angles for this control law since the
satellites are aligned with the principal axes of rotation \cite{etkins}.
\begin{equation}
  \alpha_{Ri} = -k_p(\epsilon_i-\epsilon_{desired})-k_d(\omega_i-\omega_{desired})
\end{equation}
In the equation above $\epsilon$ denotes either roll $\phi$, pitch
$\theta$ or yaw $\psi$ depending on which reaction wheel is being
used. The Euler angles in this case would be obtained by converting
the quaternions to Euler angles as defined in Section \ref{s:quat}.

Often times however your reaction wheels are not pointed on the
principal axis of inertia. In this case a Least Squares Regression
model is needed. In this case the equation above is used to compute
the desired torque to be placed on the satellite such that
\begin{equation}
  \vec{M}_{desired} = -k_p(\epsilon_i-\epsilon_{desired})-k_d(\omega_i-\omega_{desired})
\end{equation}
This equation is then equated to the equation for torque placed on the
satellite where the angular accelerations are placed into a vector.
\begin{equation}
  \vec{M}_{desired} = \vec{M}_R = \sum\limits_{i=1}^{NR}
  I^{B}_{Ri}\alpha_{Ri}\hat{n}_{Ri} = \begin{bmatrix}
    I^{B}_{R1}\hat{n}_{R1} & \hdots &
    I^{B}_{RNR}\hat{n}_{RNR} \end{bmatrix} \begin{Bmatrix} \alpha_1
    \\ \vdots \\ \alpha_{NR} \end{Bmatrix} = {\bf J} \vec{\alpha}
\end{equation}
Since ${\bf J}$ is a $3\times{N_{RW}}$ matrix its impossible to simply
invert the matrix and solve for the vector of angular accelerations
$\vec{\alpha}$. In this case there are an infinite number of
solutions. As such a minimization routine is required where the
solution found also happens to be the lowest amount of angular
acceleration. In this case, Lagrange's method was used to find the
vector of angular accelerations \cite{lagrange}.
\begin{equation}
  \vec{\alpha} = {\bf J}^T\left({\bf J}{\bf J}^T\right)^{-1}\vec{M}_{desired}
\end{equation}

\subsubsection{Reaction Control Thrusters}

The control law for the thrusters is a bit complex if the location of
thrusters is not know apriori. If the location {\it is} known then
simple PID control laws can be generated by applying pure couples to
the correct thrusters that activate the correct axes. If the location
{\it is not} known then the following derivation will suffice. There
are $N_P$ thrusters and only 3 degrees of freedom that need to be
controlled; thus, the system is an overactuated system. Using equation
\ref{e:propulsion}, the equation can be written in matrix form as
given by the equation below where $\vec{M}_p$ is replaced by
$\vec{M}_{desired}$. The equation for $\vec{M}_{desired}$ is generated
using a similar PD control law as the reaction wheels. 
\begin{equation}
  \vec{M}_{desired} = p[{\bf S}(\vec{r}_{P1})\hat{n}_{P1}~~{\bf
      S}(\vec{r}_{P2})\hat{n}_{P2}~~\hdots~~{\bf
      S}(\vec{r}_{P{N_p}})\hat{n}_{P{N_p}}]\vec{\sigma} = {\bf M}\vec{\sigma}
\end{equation}
Since ${\bf M}$ is a $3\times{N_P}$ matrix its impossible to simply
invert the matrix and solve for the vector of pulses
$\vec{\sigma}$. In this case there are an infinite number of
solutions. As such a minimization routine is required where the
solution found also happens to be the least amount of pulses. In this
case, Lagrange's method was used to find the vector of pulses \cite{lagrange}.
\begin{equation}
  \vec{\sigma} = {\bf M}^T\left({\bf M}{\bf M}^T\right)^{-1}\vec{M}_{desired}
\end{equation}
Note that a similar equation can be derived for
$\vec{F}_{desired}$. The solution to the equation above results in
values of $\sigma$ that are bigger than 1 and sometimes negative. If a
value in this vector is bigger than 0 the value is set to 1 and
if the value is negative the value is set to 0. Thus, the solution
does not yield an exact solution but it does allow for flexibility in
the number of thrusters and their respective orientations. Sizing of
the thrusters depends on many independent variables 
including the thrust $T$ and the $I_{sp}$. Using the $I_{sp}$ the exit
velocity of the thruster can be obtained by using the equation below
\begin{equation}
  v_e = I_{sp}g_0
\end{equation}
where $g_0$ is the gravitational acceleration of the Earth at
sea-level. Then the mass flow rate of the thruster can be obtained
using the equation below.
\begin{equation}
  \dot{m}_P = T/v_e
\end{equation}
Using this mass flow rate total propellant mass required can be
computed assuming a certain duty cycle.

\subsubsection{Cross Products of Inertia Control}

An interesting form of control is to take advantage of momentum
dumping. Looking at the equation for angular acceleration again (Eqn \ref{e:pqrdot}) this
equation can be simplified for certain cases. For example, if the roll
rate of the satellite is set to be non-zero while the pitch rate and
yaw rates are set to zero it is easy to see that if the inertia is
diagonal the derivative of angular velocity is zero. However, if the
cross products of inertia are given by the matrix below
\begin{equation}
  {\bf I}_s = \begin{bmatrix} I_{xx} & I_{xy} & I_{xz} \\ I_{xy} &
    I_{yy} & I_{yz} \\ I_{xz} & I_{yz} & I_{zz} \end{bmatrix}
\end{equation}
the derivative of angular velocity becomes
\begin{equation}
  \dot{\vec{\omega}}_{B/I} = I_S^{-1} \left (\begin{Bmatrix} 0
    \\ -I_{xz}p^2 \\ I_{xy}p^2 \end{Bmatrix} \right )
\end{equation}
again assuming the roll rate is non zero and the pitch rate is
zero. This result shows that momentum can be transferred to different
axes provided the cross products of inertia are non-zero.

\subsection{PID Control of a Fixed Wing Aircraft}

For a conventional PID controller of an aircraft, the rudder, elevator
and aileron commands are set to 
\begin{equation}
\begin{matrix}
\delta_{r} =-K_v v \\
\delta_{e} = K_p(\theta-\theta_{c})+K_d{\dot \theta} \\ 
\delta_a = K_p(\phi-\phi_{c})+K_d{\dot \phi}
\end{matrix}
\end{equation}
The Euler angle commands $\phi_{c}$ and $\theta_{c}$ are set using the
following relationships:
\begin{equation}
  \begin{matrix}
  \phi_{c} = K_p(\psi-\psi_c)+K_d\dot{\psi} \\
  \theta_{c} = K_p(z-z_c) + K_d\dot{z} + K_{I}\int{z-z_c}dt
  \end{matrix}
\end{equation}
The control scheme defined above is a conventional inner loop-outer
loop control of a fixed wing aircraft using a PID tracking
controller.
