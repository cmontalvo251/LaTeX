\section{Particle Dynamics}

\subsection{Systems of Particles}

For this formulation we start with Newton's Second Law with no
approximations. Similar dynamic forumlations can be found in \cite{etkins,
  phil,nelson,astrodynamics}.
\begin{equation}
\sum\limits_{i=0}^N \vec{F}_{ji} = \frac{d\vec{p}_j}{dt}
\end{equation}
where $\vec{p}_j$ is the momentum of a particle. $\vec{F}_{ji}$ is a
force on the particle. The statement above states that sum of all
forces on a particle is equal to the time rate of change of
momentum. If two particles are then considered the equation can
be written for both particles.
\begin{equation}
\begin{matrix}
\sum\limits_{i=0}^N \vec{F}_{1i} + \vec{f}_{12} = \frac{d\vec{p}_1}{dt} &
\sum\limits_{i=0}^N \vec{F}_{2i} + \vec{f}_{21} = \frac{d\vec{p}_2}{dt} 
\end{matrix}
\end{equation}
Note that the forces $\vec{f}_{12}$ and $\vec{f}_{21}$ are internal forces
experienced by each particle exerted on each other since they are
rigidly connected. Newton's Third Law states that for every action
there is an equal and opposite reaction. That is, $\vec{f}_{12} = -\vec{f}_{21}$. Thus, if both equations are added the following equation is
created
\begin{equation}
\sum\limits_{j=0}^P \sum\limits_{i=0}^N \vec{F}_{ji} =
\sum\limits_{j=0}^P \frac{d\vec{p}_j}{dt}
\end{equation}
where P is the number of particles. Typically the double summation
in F is written just as $\vec{F}$.

\subsection{Rotational Dynamics for Systems of Particles}

Note that by construction, a system of particles rigidly connected can
now rotate about a center point. The center of mass of a system of
particles can be defined using the relationship below
\begin{equation}
\vec{r}_C = \frac{1}{m}\sum\limits_{j=0}^P m_j\vec{r}_{j}
\end{equation}
where
\begin{equation}
m = \sum\limits_{j=0}^P m_j
\end{equation}
This vector can then be used to create rotational dynamics starting
with the linear dynamics.
\begin{equation}
\sum\limits_{j=0}^P \sum\limits_{i=0}^N {\bf S}(\vec{r}_{Cj}) \vec{F}_{ji}  = \vec{M}_C = \sum\limits_{j=0}^P {\bf
  S}(\vec{r}_{Cj}) \frac{d\vec{p}_j}{dt}
\end{equation}
where ${\bf S}(\vec{r}_{Cj})$ is the skew symmetric matrix of the
vector from the center of mass to the jth particle which results in a
cross product. The skew symmetric operator is denoted by ${\bf
  S}()$. 
\begin{equation}
{\bf S}(\vec{r}_{Cj}) = \begin{bmatrix} 0 & -z_{Cj} & y_{Cj} \\ z_{Cj} & 0 &
  -x_{Cj} \\ -y_{Cj} & x_{Cj} & 0 \end{bmatrix}
\end{equation}
\section{Rigid Bodies}

At this point, many assumptions are made about the
system of particles.
\begin{enumerate}
\item The mass of each particle or rigid body is constant.
\item An inertial frame is placed at the center of the Earth that does
  not rotate with the Earth. We assume that the Earth is ``fixed" to
  this point but still rotates. The coordinates of our vehicle
  though are expressed in this non-rotating inertial frame. This is
  explained in more detail later.
\item The rigid body is not flexible and does not change shape. That
  is, the time rate of change of the magnitude of a vector
  $\vec{r}_{PQ}$ is zero for any arbitrary points P and Q attached to
  the rigid body.
\end{enumerate}

\subsection{Translational Dynamics}

Using all of these simplifications, the momentum term on the right can
be simplified to 
\begin{equation}
\sum\limits_{j=0}^P \vec{p}_j = m \vec{v}_{C/I}
\end{equation}
The derivation of the term above starts by deriving the position of
the center of mass as the following equation.
\begin{equation}
\vec{r}_{j} = \vec{r}_C + \vec{r}_{Cj}
\end{equation}
Taking one derivative results in the following equation
\begin{equation}
\vec{v}_{j/I} = \vec{v}_{C/I} + \frac{{}^Bd \vec{r}_{Cj}}{dt} +
{\bf S}(\vec{\omega}_{B/I}) \vec{r}_{Cj}
\end{equation}
where ${\bf S}(\vec{\omega}_{B/I})$ is the skew symmetric matrix of the
angular velocity vector which results in a cross product. This
equation comes from the derivative transport theorem. Since the body
is a rigid body the term $\frac{{}^Bd \vec{r}_{Cj}}{dt}=0$ resulting
in the equation below
\begin{equation}
\vec{v}_{j/I} = \vec{v}_{C/I} + {\bf S}(\vec{\omega}_{B/I}) \vec{r}_{Cj}
\end{equation}
which any dynamicist knows as the equation for two points fixed on a
rigid body. This equation can then be substituted into the equation
for momentum such that.
\begin{equation}
\sum\limits_{j=0}^P \vec{p}_j =  \sum\limits_{j=0}^P m_j \left(\vec{v}_{C/I}
+ {\bf S}(\vec{\omega}_{B/I}) \vec{r}_{Cj}\right)
\end{equation}
The first term reduces to 
\begin{equation}
\sum\limits_{j=0}^P m_j \vec{v}_{C/I} =  \vec{v}_{C/I}
\sum\limits_{j=0}^P m_j =  m \vec{v}_{C/I} 
\end{equation}
the second term reduces to zero since the sum of all particles from
the center of mass is by definition the center of mass and thus zero.
\begin{equation}
\sum\limits_{j=0}^P {\bf S}(\vec{\omega}_{B/I}) m_j\vec{r}_{Cj} =
{\bf S}(\vec{\omega}_{B/I})\sum\limits_{j=0}^P m_j\vec{r}_{Cj} = 0
\end{equation}
Plugging this result for momentum into Newton's equation of motion
yields. This is typically called Newton-Euler equations of motion.
\begin{equation}
\vec{F}_C = m \left(\frac{{}^Bd \vec{v}_{C/I}}{dt} +
{\bf S}(\vec{\omega})_{B/I} \vec{v}_{C/I} \right)
\end{equation}

\subsection{Rotational Dynamics}

Plugging in the expression for two points fixed on a rigid body
results in a much different expression. First let's expand the
rotational dynamic equations of particles using the assumptions made
for a rigid body.
\begin{equation}
\vec{M}_C = \frac{d}{dt}\sum\limits_{j=0}^P {\bf S}(\vec{r}_{Cj}) m_j\vec{v}_{j/I}
\end{equation}
Then the equation of two points fixed on a rigid body can be
introduced to obtain the following equation
\begin{equation}
\vec{M}_C = \frac{d}{dt}\sum\limits_{j=0}^P {\bf S}(\vec{r}_{Cj}) m_j\left(\vec{v}_{C/I} + {\bf S}(\vec{\omega}_{B/I}) \vec{r}_{Cj}\right)
\end{equation}
expanding this into two terms yields
\begin{equation}
\vec{M}_C =  \frac{d}{dt}\left(\sum\limits_{j=0}^P m_j{\bf S}(\vec{r}_{Cj}) {\bf S}(\vec{\omega}_{B/I}) \vec{r}_{Cj} + \sum\limits_{j=0}^P {\bf S}(\vec{r}_{Cj})m_j\vec{v}_{C/I}\right)
\end{equation}
To simplify this further a useful equality is used for cross
products. That is ${\bf S}(\vec{a})\vec{b}=-{\bf
  S}(\vec{b})\vec{a}$. The equation above then changes to
\begin{equation}
\vec{M}_C =  \frac{d}{dt}\left(\left(-\sum\limits_{j=0}^P m_j{\bf S}(\vec{r}_{Cj}){\bf
  S}(\vec{r}_{Cj})\right)\vec{\omega}_{B/I} - {\bf S}(\vec{v}_{C/I})
\sum\limits_{j=0}^P \vec{r}_{Cj}m_j\right)
\end{equation}
Notice, that parentheses were placed around the first term to isolate
the angular velocity. This is because the angular velocity is constant
across the system of particles. The term on the right has also been
altered slightly to isolate the fact that the velocity of the center
of mass is independent of the system of particles. With the equation
in this form it is easy to see that the term on the right is zero
because it is the definition of the center of mass. The equation then
reduces to 
\begin{equation}
\vec{M}_C =  \frac{d}{dt}\left(\sum\limits_{j=0}^P m_j{\bf S}(\vec{r}_{Cj}){\bf
  S}(\vec{r}_{Cj})^T\right)\vec{\omega}_{B/I} 
\end{equation}
Notice again that minus sign has been removed. The skew symmetric
matrix has an interesting property where the transpose is equal to the
negative of the original matrix. The term in brackets is a well known
value for rigid bodies and is known as the moment of inertia for rigid
bodies. 
\begin{equation}
{\bf I}_C =  \sum\limits_{j=0}^P m_j{\bf S}(\vec{r}_{Cj}){\bf
  S}(\vec{r}_{Cj})^T
\end{equation}
This results in the kinematic equations of motion for rigid bodies to
the simple equation below.
\begin{equation}
\vec{M}_C =  \frac{d}{dt}\left({\bf I}_C\vec{\omega}_{B/I} \right)
\end{equation}
With the equation in this form it is finally possible to carry out the
derivative
\begin{equation}
\vec{M}_C = \frac{{}^Bd ({\bf I}_C\vec{\omega}_{B/I})}{dt} + {\bf
  S}(\vec{\omega}_{B/I}){\bf I}_C\vec{\omega}_{B/I}
\end{equation}
The first term requires the chain rule to perform the derivative and
can thus result in a time varying moment of inertia matrix and the
derivative of angular velocity. Therefore the equation can simply be written as
\begin{equation}
\vec{M}_C = \dot{{\bf I}}\vec{\omega}_{B/I} + {\bf I}_C\frac{{}^Bd (\vec{\omega}_{B/I})}{dt} + {\bf
  S}(\vec{\omega}_{B/I}){\bf I}_C\vec{\omega}_{B/I}
\end{equation}

\subsection{Inertia Estimation}

There are several equations that can be used to compute the moment of
inertia depending on the geometry of the vehicle. For this example
we will look at a cuboid to demonstrate inertia
calculations. Firstly, the total mass $m$ and size (length $l$,
width $w$, and height $h$) are required. 
\begin{equation}
  \begin{matrix}
    I_x = \frac{m}{12}(l^2 + w^2) \\
    I_y = \frac{m}{12}(l^2 + d^2) \\
    I_z = \frac{m}{12}(d^2 + w^2) \\
  \end{matrix}
\end{equation}

Where $I_x$ is the cuboid’s moment of inertia around the x-axis, It
can be seen that the moment of inertia about the y and z-axis are
computed similarly, but by using different length parameters. Note
that cross products of inertia are obtained by using the parallel axis
theorem often caused by solar panels on satellites or payloads on
aircraft and quadcopters.

\subsection{Aerospace Convention}

Aerospace convention involves using the Newton-Euler equations of
motion to describe the vehicle\cite{etkins} as explained in the
previous section. Typically the position of the
vehicle is written as 
\begin{equation}
{\bf C}_I(\vec{r}_C) = \begin{Bmatrix} x \\ y \\ z \end{Bmatrix}
\end{equation}
The derivative of the position vector is the velocity vector is then
written as
\begin{equation}
{\bf C}_I(\vec{v}_{C/I}) = \begin{Bmatrix} \dot{x} \\ \dot{y} \\ \dot{z} \end{Bmatrix}
\end{equation}
However, body frame coordinates are typically used to
describe the velocity vector such that
\begin{equation}
{\bf C}_B(\vec{v}_{C/I}) = \begin{Bmatrix} u \\ v \\ w \end{Bmatrix}
\end{equation}
In order to relate the body frame components of the velocity vector
the inertial frame coordinates a transformation matrix is used to give
the following equation.
\begin{equation}\label{e:xyzdot}
\begin{Bmatrix} \dot{x} \\ \dot{y} \\ \dot{z}   \end{Bmatrix} = [\textbf{T}_{IB}]
\begin{Bmatrix} u \\ v \\ w \end{Bmatrix}
\end{equation}
Note that standard aircraft forces and moments are applied to the
body. The forces are typically written as X,Y and Z while the moments
are given as L,M and N. They can be written in component form using
the equations below.
\begin{equation}
{\bf C}_B(\vec{F}_{C}) = \begin{Bmatrix} X \\ Y
  \\ Z \end{Bmatrix} = X \hat{I}_B + Y \hat{J}_B + Z \hat{K}_B
\end{equation}
\begin{equation}
{\bf C}_B(\vec{M}_{C}) = \begin{Bmatrix} L \\ M
  \\ N \end{Bmatrix} = L \hat{I}_B + M \hat{J}_B + N \hat{K}_B
\end{equation}
