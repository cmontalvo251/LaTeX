\section{AIRCRAFT NOMENCLATURE}
\begin{tabbing}
  XXXXXXXXXX \= \kill% this line sets tab stop
  $x_i,y_i,z_i$ \> components of the mass center position vector in the
  inertial frame of aircraft $i$ (m) \\
  $\phi_i,\theta_i,\psi_i$ \> Euler roll,pitch, and yaw of aircraft
  $i$ (rad) \\
  $u_i,v_i,w_i$ \> components of the mass center velocity vector in the
  body frame of aircraft $i$ (m/s) \\
  $p_i,q_i,r_i$ \> components of the mass center angular velocity vector in the
  body frame of aircraft $i$ (rad/s) \\
  ${\vec r}_{A\rightarrow B}$ \> position vector from a generic point A
  to a generic point B(m) \\
  ${\vec V}_{A/B}$ \> velocity vector of a generic point A with respect
  to frame B (m/s) \\
  $\textbf{T}_{AB}$ \> generic transformation matrix rotating a vector from
  the frame B to frame A \\
  $\textbf{H}_i$ \> relationship matrix of Euler angle derivatives to
  body angular velocity components of aircraft $i$ \\
  $m_i$ \> mass of aircraft $i$ (kg) \\
  $I_i$ \> moment of inertia matrix of aircraft $i$ taken about the mass center
  in the body frame($kg-m^2$) \\
  $X_i,Y_i,Z_i$ \> components of the total force applied to aircraft $i$ in
  body frame(N) \\
  $L_i,M_i,N_i$ \> components of the total moment applied to aircraft
  $i$ in body frame(N-m) \\
  $X_{Wi},Y_{Wi},Z_{Wi}$ \> total weight force applied to aircraft
  $i$ (N) \\
  $L,D$ \> Lift and Drag on Aircraft (N) - Not to be confused with Roll moment\\
  $g$ \> gravitational constant on Earth $(m/s^2)$ \\
  $\rho$ \> atmospheric density($kg/m^3$) \\
  $S_i$ \> reference area of wing on aircraft $i$ ($m^2$) \\
  $b_i$ \> Wingspan of aircraft $i$ (m) \\
  $\bar{c}_i$ \> mean chord of wing on aircraft $i$ (m) \\
  $\alpha$ \> Angle of attack (rad) \\
  $\beta$ \> Slideslip angle (rad) \\
  $C_L,C_D,C_m$ \> Lift, Drag and Pitch Moment coefficients \\
  $\delta_t,\delta_a,\delta_r,\delta_e$ \> thrust, aileron, rudder,
  and elevator control inputs(rad) \\
  $S_B(\vec{r})$ \> skew symmetric matrix operator on a vector
  expressed in the body frame. \\
  $K_p,K_d,K_I$ \> proportional, derivative, and integral control
  gains\\
  $V$ \> Total airspeed (m/s) \\
  $\hat{p},\hat{q},\hat{r}$ \> Non-dimensional angular velocities \\
  $l$ \> Distance from center of mass to aerodynamic center of the
  tail (m) \\
  $l_t$ \> Distance from aerodynamic center of main wing to
  aerodynamic center of tail (m) \\
  $\alpha_0$ \> zero lift angle of attack (rad) \\
  $C_{L0}$ \> Zero angle of attack lift coefficient \\
  $C_{m\alpha}$ \> Pitch moment curve slope versus $\alpha$ \\
  $C_{L\alpha}$ \> Lift curve slope \\
  $C_{mq}$ \> Pitch damping coefficient \\
  $C_{m\delta_e}$ \> Pitch moment curve slope versus elevator
  deflection angle \\
  $a_{\infty}$ \> Speed of sound (m/s) \\
  $\mu_{\infty}$ \> Viscosity of Fluid $kg/(m-s)$ \\
  
\end{tabbing}

\newpage

\section{EQUATIONS}

\begin{multicols}{2}

\noindent Mach Number and Reynolds Number

\beq
\beqn
M_{\infty} = \frac{V}{a_{\infty}} \\
\ \\
Re = \frac{\rho V \bar{c}}{\mu_{\infty}}
\eeqn
\eeq

\noindent Total Velocity

\beq
V = \sqrt{u^2 + v^2 + w^2}
\eeq

\noindent Angle of Attack and Sideslip

\beq
\beqn
\alpha = tan^{-1}\left(\frac{w}{u}\right)\\
\beta = sin^{-1}\left(\frac{v}{V}\right)
\eeqn
\eeq

\noindent Lift Drag and Moment

\beq
\beqn
Lift~(L) = \frac{1}{2} \rho V^2 S C_L \\
Drag~(D) = \frac{1}{2} \rho V^2 S C_D\\
Roll~Moment~(L) = \frac{1}{2} \rho V^2 S b C_l\\
Pitch~Moment~(M) = \frac{1}{2} \rho V^2 S \bar{c}C_m\\
Yaw~Moment~(N) = \frac{1}{2} \rho V^2 S b C_n\\
\eeqn
\eeq

\noindent Lift and Drag Coefficients

\beq
\beqn
C_L = C_{L0} + C_{L\alpha}\alpha\\
C_L = C_{L\alpha}(\alpha-\alpha_0)\\
C_D = C_{D0} + C_{D\alpha}\alpha^2 \\
C_D = C_{D0} + k{C_L}^2 \\
\eeqn
\eeq

\noindent Non-dimensional Angular velocities

\beq
\beqn
\hat{p} = pb/2V\\
\hat{q} = q\bar{c}/2V\\
\hat{r} = rb/2V
\eeqn
\eeq

\noindent Pitch Moment equation

\beq
C_m = C_{m0} + C_{m\alpha}\alpha + C_{m\delta_e}\delta_e +
C_{mq}\hat{q}
\eeq

\beq
\beqn
C_{m0} = C_{MAC} + C_{L0}\bar{x}_{sm} \\
\bar{x}_{sm} = \frac{x_{cg}}{\bar{c}} - \frac{x_{acW}}{\bar{c}} \\
C_{m\alpha} = \left(C_{L\alpha,W} +
\frac{S_t}{S}C_{L\alpha,t}\right)\bar{x}_{sm} - V_HC_{L\alpha,t}\\
V_H = \frac{l_tS_t}{S\bar{c}} \\
C_{m\delta_e} = \left(C_{Lt\delta_e}\frac{S_t}{S}\right)\bar{x}_{sm}-V_HC_{Lt\delta_e}\\
C_{mq} = 2C_{L\alpha t}\frac{l^2}{\bar{c}^2}
\eeqn
\eeq

\noindent Max Lift to Drag Ratio (Only valid if ${C_{L0}=0}$)

\beq
\alpha_{max,L/D} = \sqrt{\frac{C_{D0}}{C_{D\alpha}}}
\eeq

\noindent Lift to Drag when $T=0$ (Sum of Forces still zero)

\beq
\beqn
\frac{D}{L} = tan(\alpha)\\ %%%Where does this come from? Sum of Forces = 0
Lcos(\alpha) + Dsin(\alpha) = W
\eeqn
\eeq

\noindent Airfoil and Wing Aerodynamics

\beq
\beqn
x_{ac} = c/4 & 
a = \frac{a_0}{1+\frac{a_0}{\pi e AR}} \\
AR = \frac{b^2}{S}
\eeqn
\eeq


\noindent Standard Atmosphere

\beq
\beqn
\rho = 1.225~kg/m^3 = 0.00238~slugs/ft^3\\
\mu_{\infty} = 1.81x10^{-5}~kg/(m-s)\\
a_{\infty} = 331.3~m/s
\eeqn
\eeq

\noindent General Notes

\begin{enumerate}
  \item In trim or steady and level or cruise $q=0$, $C_m=0$,$L=W$, $T=D$
  \item For symmetric airfoil $C_{MAC} = 0$ and
    $C_{L0}=0$ thus $C_{m0} = 0$
  \item For a flat plate all symmetric properties apply but
    $a_0 = 2\pi$
  \item Tail surfaces are always assumed to be flat plates
  \item For longitudinal problems, $\beta = 0$ so $v=0$ (side velocity)
\end{enumerate}

\end{multicols}
