\section{Linear Time Invariant Systems and Controls}

\subsection{Linear Dynamics}

Standard nonlinear dynamics can be placed into standard
nonlinear affine form as shown below after much simplification of
terms
\begin{equation}
  \dot{\vec{x}} = \vec{f}(\vec{x}) + \vec{g}(\vec{x})\vec{u}
\end{equation}
where $\vec{u}$ is the control input which could be the forces and
moments from reaction wheels or thrusters. The equation above can be
linearized to give the equation below. 
\begin{equation}
  \Delta \dot{\vec{x}} = {\bf A}\Delta {\vec{x}} + {\bf B}\Delta \vec{u}
\end{equation}
where $\Delta \vec{x} = \vec{x} - \vec{x}_e$ and $\vec{x}_e$ is an
equilibrium point. In this formulation ${\bf A} = \partial \vec{f}/\partial \vec{x}$. and 
${\bf B} = \partial \vec{g}/\partial \vec{x}$ which are partial derivatives of the state matrices.

\subsection{Example Second Order System Formulation}

A second order system undergoing free motion will have dynamics that look like this
\begin{equation}
\ddot{q} + 2\zeta \omega_n \dot{q} + {\omega_n}^2 = \sigma f
\end{equation}
where $q$ is a generalized coordinate, $\sigma$ is a forcing term proportional to the mass of the object and the forcing function $f$, $\omega_n$ is the natural frequency of the system and $\zeta$ is the damping ratio. Examples of these types of systems include mass, spring dampers in linear translation as well as torsional systems and penduluums. Anything that oscillates will exhibit this behavior. 
