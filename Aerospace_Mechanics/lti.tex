\section{Linear Time Invariant Systems and Controls}

There is a branch of controls called Linear Time Invariant (LTI)
Systems that is often taught at the undergraduate level. Although
almost every system encountered in standard applications whether
aerospace or not are non-linear, it is still beneficial and more
simple to learn about control system dynamics when the system
parameters are constant and linear. 

\subsection{Linear Dynamics}

Standard nonlinear dynamics can be placed into standard
nonlinear affine form as shown below after much simplification of
terms
\begin{equation}
  \dot{\vec{x}} = \vec{f}(\vec{x}) + \vec{g}(\vec{x})\vec{u}
\end{equation}
where $\vec{u}$ is the control input which could be the forces and
moments from reaction wheels or thrusters. The equation above can be
linearized to give the equation below. 
\begin{equation}
  \Delta \dot{\vec{x}} = {\bf A}\Delta {\vec{x}} + {\bf B}\Delta \vec{u}
\end{equation}
where $\Delta \vec{x} = \vec{x} - \vec{x}_e$ and $\vec{x}_e$ is an
equilibrium point. In this formulation ${\bf A} = \partial \vec{f}/\partial \vec{x}$. and 
${\bf B} = \partial \vec{g}/\partial \vec{x}$ which are partial derivatives of the state matrices.

\subsection{Example Second Order System Formulation}

A second order system undergoing free motion will have dynamics that look like this
\begin{equation}
\ddot{q} + 2\zeta \omega_n \dot{q} + {\omega_n}^2 = \sigma f
\end{equation}
where $q$ is a generalized coordinate, $\sigma$ is a forcing term
proportional to the mass of the object and the forcing function $f$,
$\omega_n$ is the natural frequency of the system and $\zeta$ is the
damping ratio. Examples of these types of systems include mass, spring
dampers in linear translation as well as torsional systems and
penduluums. Anything that oscillates will exhibit this behavior. For a
standard mass spring damper system as shown in Figure \ref{f:msd}

\hl{INCLUDE MSD FIGURE}

\hl{INCLUDE MSD EOM FORMULATIONS}

\subsection{Solutions to Differential Equations}

\subsubsection{Characteristic and Particular Solutions}

\subsubsection{Laplace Solutions}

\subsection{Time Response of First and Second Order Systems}

\subsection{Proportional Derivative Integral Control}

\subsection{Lyapunov Control}

\subsection{Sliding Mode Control}

\subsection{Adaptive Control}

\subsection{Controllability}

Controllability is formally stated as a system where any initial
state $x(0)=x_0$ and final state $x_1,t_1>0$, there exists a piecewise
continuous input $u(t)$ such that $x(t_1)=x_1$. 
For a fixed wing aircraft the system has 12 states with 8 dynamic
modes and 4 zero or rigid body modes. For a fixed wing aircraft the system has 12 states with 8 dynamic
modes and 4 zero or rigid body modes. A conventional aircraft has 4
controls to control these 12 modes. The easiest way to test the 
controllability of a system  is to compute the
controllability matrix. However, the controllability matrix must be
computed using a linearized model such that
$\dot{\vec{x}}=A\vec{x}+B\vec{u}$. In order to do this the aircraft
must be in equilibrium. For this example the aircraft is
set with an initial velocity of $20~m/s$ at an altitude of
$200~m$. The altitude command is set to $200~m$ and the heading
command is set to zero. Given the zero heading angle command and the
symmetry of the configurations investigated the rudder and aileron
commands are set to zero. Thus, only the thrust and elevator controls
are activated for the trimming procedure. Each configuration is
simulated for 200 seconds or until the derivatives of all states
except $\dot{x}$ are within a required tolerance. Using this
equilibrium point a linear model can be computed by using forward
finite differencing assuming that the
aircraft model is put in the form $\dot{\vec{x}} = F(\vec{x},\vec{u})$.
\begin{equation}
\dot{\vec{\delta x}} = \frac{F(\vec{x_0}+\Delta \vec{x_0},\vec{u_0})-F(\vec{x_0},\vec{u_0})}{\Delta
  \vec{x}}\vec{\delta x} + \frac{F(\vec{x_0},\vec{u_0}+\Delta
  \vec{u})-F(\vec{x_0},\vec{u_0})}{\Delta \vec{u}}\vec{\delta u}
\end{equation}
This linear model is the classic linear model where
$\dot{\vec{\delta x}}=A\vec{\delta{x}}+B\vec{\delta{u}}$. Using this linear model, the
controllability matrix can be computed as
\begin{equation}
W_C = [B~AB~A^2B~A^3B~...~A^{N-1}B]
\end{equation}
where N is the number of states in the system. With the controllability
matrix formulated, the rank of the matrix is computed. If the
$rank(W_C)=N$ the system is said to be controllable.

