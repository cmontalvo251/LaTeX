%%%%%%%%%%%%%%%%%%%%%%%%%%%%%%%%%%%%%%%%%%%%%%%%%%%%%%%%%%%%%%%%%%%
%%%%%%%%%%%%%%%%%%%%%%%%%%%%%%%%%%%%%%%%%%%%%%%%%%%%%%%%%%%%%%%%%%%
%%%%%%%%%%%%%%%%%%%%%%%%%%%%%%%%%%%%%%%%%%%%%%%%%%%%%%%%%%%%%%%%%%%

\section*{}

\section*{Current Edition}

This manuscript was last updated on \today. 
The latest edition can be found on \href{https://github.com/cmontalvo251/LaTeX/blob/master/Aerospace_Mechanics/aerospace_mechanics.pdf}{Github}
\ \\
\ \\
\noindent {\it A note on AI:} Note that many sections have been written with the help of
Github's Copilot \cite{GithubCopilot} and even Google's Gemini \cite{Gemini}. Originially I had plans to properly cite each section that was written with the help of AI but I lost track given it's seamless integration with VScode and Android smart phones as well as any browser at this point. As such it is important to note that after September of 2025 you can assume that most if not all sections were written with the help of AI.

\section*{Manuscript Changes}

\begin{enumerate}[itemsep=-5pt]
\item June 10th, 2020 - Magnetic Field Model section updated to
  reflect the difference between the East North Vertical and the North
  East Down reference frames. The Figure showing the magnetic field
  for an example Low Earth Orbit has also been update. Also added this
  page for manuscript changes and the following pages that list where
  this file can be found.
\item December 10th, 2020 - Moved to public Github Repo separate from
  MATLAB
\item December 30th, 2020 - Moved papers.bib to parent directory Added
  datetime to title page. Added  section headers to RC aircraft
  design. Wrote text all the way through the airfoil selection. Added
  a references section.
\item December 31st, 2020 - Finished RC aircraft section.
\item September 30th, 2021 - Renamed report to Aerospace Mechanics and
  Controls
\item December 21st, 2021 - Moved CubeSAT abstract to introduction on
  CubeSATs. Imported entire Aircraft Mechanics textbook into here
\item December 22nd, 2021 - Added a section on GNC design for
  CubeSats. Added acknowledgements section.
\item June 2nd, 2022 - Added some items to changes needed and fixed
  two references
\item July 30th, 2022 - Included a derivation of direct measurement of Euler Angles using an IMU.
\item July 31st, 2022 - Added GPS coordinate conversion to cartesian coodinates as well as heading angle and speed estimation from GPS.
\item October 27th, 2022 - Added computation of lat, lon, alttiude to ECI frame.
\item March 20th, 2024 - Added a Current Edition section above
  Manuscript Changes. Also added color to hyperlinks
\item May 7th, 2024 - Added a battery sizing section to the aircraft
  design chapter
\item July 20th, 2024 - separated sections into separate files...Also started adding the quacopter aerodynamic model
\item July 22nd, 2024 - Finished the quadcopter aerodynamics section and made a quick edit to the GNC aircraft PID control scheme
\item December 23rd, 2024 - Added a better description to the gravity model to explain the vector notation of the equations
\item January 6th, 2025 - A title change was performed for the GPS to Cartesian coordinate transform. Also made orbital elements its own section.
\item January 9th, 2025 - Edited the position of different sections so that it's ready for the linear controls section
\item May 13th, 2025 - Added drag equation to aerodynamics
\item September 7-8th, 2025 - Used Github's copilot on the Mobile App to create the first order example of a differential equation. Also used Gemini as well to create the second order example. I also added a citation for Gemini in the papers.bib file. I also edited the foreword to include a note on AI since from here on out I will use it to help write many sections of this manuscript.
\item September 10th, 2025 - Added the equations of motion for a second order mass spring damper system.
\item September 28th, 2025 - Moved the section on numerical integration to be before the LTI section since it makes more sense to have it there. Also added the second order solution showing underdamped, overdamped and critically damped solutions.
\item September 29th, 2025 - Added the time response section for first order systems and finished the derivation of first and second order systems.
\item September 30th, 2025 - added more section headers to the LTI section to make it easier to navigate.
\item October 10th, 2025 - Split the LTI Controls section into its own file and started writing the feedback control section.
\item October 11th, 2025 - Added the attitude dynamics of a satellite/quadcopter section. Also rearranged the sections overall
\item October 12th, 2025 - Finished the rocket dynamics equations and started on the stability section.
\item October 13th, 2025 - Added the pitch dynamics of an aircraft
\item October 14th, 2025 - Finished the dynamics section of LTI systems and am working on the time response section now
\item October 15th, 2025 - Added the week by week schedule for the project based learning section
\item October 18th, 2025 - Added the code for simulating the mass spring damper system
\end{enumerate}

\section*{Changes Needed}

\begin{enumerate}[itemsep=-5pt]
  \item Aircraft Changes
    \begin{enumerate}[itemsep=-5pt]
    \item Include some plots on RC aircraft design
    \item Add an example RC aircraft design
    \end{enumerate}
\item Spacecraft Changes
  \begin{enumerate}[itemsep=-5pt]
  \item Add some work on Kerbal Space Program
  \item Discuss how to get position and velocity from orbital elements and orbital elements
    from position and velocity.
  \item Need to explain the difference between Geodetic and Geocentric
    coordinates
  \item Derivation of ground path taking into account orbital
    precession, rotation of the Earth and swath angle from projection of
    a satellite onto the Earth. See pdf from the Air Force. A Google
    search will hopefully turn up the paper I'm thinking of.
  \item Consider adding the section on pointing analysis
  \end{enumerate}
\item General Changes
  \begin{enumerate}[itemsep=-5pt]
  \item Add an abbreviations section
  \item Add Project Based learning to this manuscript
  \item Direct the reader to my Instrumentation textbook to build a
    datalogger to put on an airplane or rocket.
  \item Direct reader to FASTkit to run dynamics in simulation
  \item Include some simulation results of aircraft, rocket and
    satellite from FASTkit.
  \item Need to add derivation of a complimentary filter using
    transfer functions
  \item Need to finish discrete dynamics section
  \item Need example ID and ADs
  \item Need to add parallel axis theorem for inertia computation
  \item LTI Sections for general system dynamics
  \item Add GNC for all vehicles: spacecraft, rocket, aircraft, quadcopter
  \end{enumerate}
\end{enumerate}

\section*{Acknowledgements}

Carlos Montalvo would like to thank numerous students for their
contribution to this document. They have been instrumental in making
this textbook a reality and this textbook would not be where it is
today without them. Those students are: Weston Barron, Colin Mcgee,
Darcey D'Amato, Ruthie Hill, Drew Russ, William Sherman, Maxwell Cobar, Wei Min Patrick, Caroline
Franklin, Andrew Givens, Aaron Godfrey, Nghia Huynh, Lisa Schibelius,
and Brandon Troub.

\newpage

\tableofcontents

\newpage
