\section{Feedback Control}

Feedback control is a large and complex topic. This section will
introduce some of the basic concepts and methods used in feedback
control. The sections that follow will provide many examples for linear time invariant systems as defined in the previous section. However, more complex examples will be discussed in later subsections. 

\hl{include graphic of Feedback.png}

\subsection{Controllability}

Before beginning to talk about control there is a necessary discussion about controllability. Controllability is formally stated as a system where any initial state $x(0)=x_0$ and final state $x_1,t_1>0$, there exists a piecewise continuous input $u(t)$ such that $x(t_1)=x_1$. What this means is that a system is controllable if it can be driven from any initial state to any final state in a finite time. This is an important concept because if a system is not controllable then no matter how good the control system is it will never be able to drive the system to the desired state.

For a fixed wing aircraft the system has 12 states with 8 dynamic
modes and 4 zero or rigid body modes. For a fixed wing aircraft the system has 12 states with 8 dynamic
modes and 4 zero or rigid body modes. A conventional aircraft has 4
controls to control these 12 modes. The easiest way to test the 
controllability of a system  is to compute the
controllability matrix. However, the controllability matrix must be
computed using a linearized model such that
$\dot{\vec{x}}=A\vec{x}+B\vec{u}$. In order to do this the aircraft
must be in equilibrium. For this example the aircraft is
set with an initial velocity of $20~m/s$ at an altitude of
$200~m$. The altitude command is set to $200~m$ and the heading
command is set to zero. Given the zero heading angle command and the
symmetry of the configurations investigated the rudder and aileron
commands are set to zero. Thus, only the thrust and elevator controls
are activated for the trimming procedure. Each configuration is
simulated for 200 seconds or until the derivatives of all states
except $\dot{x}$ are within a required tolerance. Using this
equilibrium point a linear model can be computed by using forward
finite differencing assuming that the
aircraft model is put in the form $\dot{\vec{x}} = F(\vec{x},\vec{u})$.
\begin{equation}
\dot{\vec{\delta x}} = \frac{F(\vec{x_0}+\Delta \vec{x_0},\vec{u_0})-F(\vec{x_0},\vec{u_0})}{\Delta
  \vec{x}}\vec{\delta x} + \frac{F(\vec{x_0},\vec{u_0}+\Delta
  \vec{u})-F(\vec{x_0},\vec{u_0})}{\Delta \vec{u}}\vec{\delta u}
\end{equation}
This linear model is the classic linear model where
$\dot{\vec{\delta x}}=A\vec{\delta{x}}+B\vec{\delta{u}}$. Using this linear model, the
controllability matrix can be computed as
\begin{equation}
W_C = [B~AB~A^2B~A^3B~...~A^{N-1}B]
\end{equation}
where N is the number of states in the system. With the controllability
matrix formulated, the rank of the matrix is computed. If the
$rank(W_C)=N$ the system is said to be controllable.

\subsection{Proportional Derivative Integral Control}

\subsubsection{Bang Bang Control of a Satellite}

\subsubsection{Proportional Control of a Satellite}

\hl{include graphic of Feedback.png}

\subsubsection{Proportional Derivative Control of a Satellite}

\subsubsection{Proportional Control of a Car}

\subsubsection{Proportional Integral Control of a Car}

\subsubsection{Proportional Integral Derivative Control of an Aircraft}

\subsection{Inner and Outer Loop Control of an Aircraft}

\subsubsection{Velocity Control}

\subsubsection{Altitude Control}

\subsubsection{Roll Angle Control}

\subsubsection{Heading Angle Control of a Car}

\subsubsection{Heading Angle Control of an Airplane}

\subsubsection{Waypoint COntrol of a Car}

\subsubsection{Waypoint Control of an Airplane}

\subsection{Lyapunov Control}

\subsection{Sliding Mode Control}

\subsection{Adaptive Control}

