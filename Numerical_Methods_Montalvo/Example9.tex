\subsection{Example Problems}

\begin{enumerate}

\item Write an interpolation program that will create Linear Splines
  for your data set. You merely need to compute the point slope form
  equations for every data point. Use your splines to compute the
  value of ystar. 

  \begin{framed}
    \textcolor{blue}{function} ystar = myinterp(X,Y,xstar)
  \end{framed}

  Note, X is a vector of data points and Y is a vector of sampled
  points occuring at the values of X. Test your code using the
  table of data below. Let xstar = 1.5, 3.5 and 5.5. Again make your
  code general enough to handle any number of data points. 

  \begin{equation}\nonumber
    \begin{matrix}
      K-value & Pressure \\
      7.5 & 0.635\\
      5.58 & 1.035\\
      4.35 & 1.435\\
      3.55 & 1.835\\
      2.97 & 2.235\\
      2.53 & 2.635\\
      2.2 & 3.035\\
      1.93 & 3.435\\
      1.7 & 3.835\\
      1.46 & 4.235\\
      1.28 & 4.635\\
      1.11 & 5.035\\
      1.0 & 4.435\\
    \end{matrix}
  \end{equation}

\item Edit your code from problem 1 and use polynomial
  interpolation instead of linear splines. Use the entire data set to
  compute the polynomial fit. Feel free to use an expansion point. It
  is entirely up to you. Your function header will look like this. 

  \begin{framed}
    \textcolor{blue}{function} ystar = interpoly(X,Y,xstar,N)
  \end{framed}

  All inputs are the same except N is now the order of the
  approximation. Thus if N=0 you use a zero order method and if N=1
  you use a linear approximation. Test your code using the data
  below and set N=1, did you get something different than problem 1?
  Explain why if so. Then set N=2. What do you get? Test it using the
  same values of xstar in problem 1.

\item Problem 2 from last weeks homework could technically be used as
  a 2-D polynomial interpolation method. 

  \begin{equation}\nonumber
    z = a_0 + a_1sin(2x) + a_2sin(2y)
  \end{equation}

  In this problem, you were given X,Y, and Z and you found $A^*$. Use
  this fit to compute $z^*$ for $x^*=1,y^*=1$ and compare it to a
  simple 2-D linear interpolation method. Explain the difference in
  your answers. 

\end{enumerate}
