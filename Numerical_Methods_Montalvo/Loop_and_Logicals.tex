\subsection{Loops and Logicals - Palm - Chapter 4}

\begin{enumerate}

\item \textbf{Logical Operators}

For now MATLAB has just been a calculator. Howevever MATLAB is much
more than that. One main thing that is done in MATLAB is to test
whether or not something is true. If it is the code does one thing and
then if not it does another. For example

$>>$ x = (1 == 0)

tests whether or not 1 is equal to zero. In this case it is not thus
the value of x is false or 0. However it is possible to do other tests
such as

$>>$ x = linspace(0,10,5);

$>>$ t = length(x) $>$ 2

In this case the length(x) is 5 which is greater than 2 thus the value
of t is true or 1. Notice that if you type in whos you will see that t
is a logical variable rather than a double. A logical can only be a
zero or a 1 thus it only needs 1 byte rather than 8 bytes. The other
logical tests are $>=$, $<=$, $\sim=$. The last is 'not equal to'.

\item \textbf{If/Else/End Statements}

If/else/end statements are called logical statements. They are
typically used in a script. The basic structure of an if statement is
as follows

if {\it statement}

~~~~~~{\it execute this block of code if true}

else

~~~~~~{\it execute this block of code if false}

end

For example the block of code below is a script that uses the
if/else/end structure to do one thing or the other. Note that this
function takes one input but has no outputs.

\begin{framed}

function logicals(N)

x = linspace(0,10,N);

if length(x) $>$ 5

~~~~~~~disp('The vector is longer than 5')

else

~~~~~~~disp('The vector is shorter than or equal to 5')

end

\end{framed}

Try coding this example and see what it does for different values of N.

\item \textbf{For Loops}

A for loop has the following structure

for {\it index = start:increment:end}


end

The block of code above will create an index that starts at the
variable start, increments by increment and stops at end. For
example the code below will add up the numbers from 1 to 10

\begin{framed}
I = 0;

for idx = 1:1:10

~~~~~~~I = I + idx;

end

\end{framed}

\item \textbf{While Loops}

A while is used when you don't know how far you are looping. For
example, assume you want to loop until the square of a number exceeds
100. Then you would need a while loop. The code below will stop when
the square of the index exceeds 100. Note that you will have to keep
track of the index rather than the for loop which does it
intrinsically.

\begin{framed}

I = 0;
idx = 1;

while idx*idx $<=$ 100

~~~~~~I = idx;

~~~~~~idx = idx + 1;

end

\end{framed}

Running the script above will stop when idx = 11 since 11 squared is
greater than 100. The value of I will be 10 however since the code
will break out of the loop before I is set to 11. Try it out and see
for yourself.



\end{enumerate}
