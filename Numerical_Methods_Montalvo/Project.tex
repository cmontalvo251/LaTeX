\section{ME 228 - FINAL PROJECT - Due April 24th, 2015}

This assignment is the final project indicated on your syllabus that
will be 10\% of your grade. In this assignment you and your team will
be required to use the engineering design method to answer a certain
problem in the engineering world. Some of the projects include data
analysis, animation and even video game design. You must write all
code by yourself however special consideration will 
be given since some of these projects are tougher than others. 

There are about 60 students between both classes and 10 projects thus each
group will consist of 5-6 members. Given the fact that one computer code
will be written, most likely two or three students will be responsible
for the code therefore, the other students should be tasked with
creating the slide presentation and written report. Note, most of
these codes are modular and can be split between members. The presentation
will be a super quick 10 minute presentation given on April 24th and
April 27th. You must have your presentation and report due by the 24th
since presentations will be drawn at random.  

Your task  for spring break is to look over the projects below and do 
as much research as you can about each project. In addition, you want
to start thinking about who you want in your group. When we return we
will go over each project in detail and finalize our group
members. You may then submit your votes via email including your team
name and your top 5 projects you would like to work on.

\subsection{\bf Timeline of Events}

Final Project Released to Students - February 27th, 2015\\
Projects Discussed in Detail - March 9th, 2015\\
\ \\
{\bf List of Members, Team Name and Top 5 Project Choices uploaded to Sakai - \underline{11:59PM} - March 13th, 2015 - One email per group is fine}\\
\ \\
Projects Assigned to Each Specific Group - March 16th, 2015\\
\ \\
{\bf Final Report, Presentation and MATLAB Code uploaded to Sakai - \underline{10:00AM} - April 24th, 2015}

\subsection{\bf List of Deliverables - 100 pts}

Member List and Team Name - 10 pts\\
Final Report - 40 pts\\
10 Minute Presentation - 30 pts\\
MATLAB Code - 20 pts

\subsection{\bf Final Report Outline - 40 pts}

Your final report will be much like the homework. The outline of the
report is given below.

\begin{enumerate}[I]
  \setlength\itemsep{-2pt}

  \item Introduction and Problem Statement

  \item Mathematical Model - Explanation of Theory

  \item Simulation Setup - What data did you require?

  \item Results - Explain your simulation results or data analysis or
    video game

  \item Conclusion

  \item References
   
  \item Appendix - MATLAB Code

\end{enumerate}

\subsection{Final Report Grading Rubric}

\begin{enumerate}[A]
  \setlength\itemsep{-2pt}

  \item Sources - All sources (information and graphics) are
    accurately documented in the desired format.

  \item Diagrams and Illustrations - Diagrams and illustrations are
    neat, accurate and add to the reader's understanding of the topic

  \item Organization - Information is very organized with well
    constructed paragraphs and subheadings

  \item Amount of Information - All topics are addressed and all
    questions answered with at least 2 sentences about each

  \item Quality of Information - Information clearly relates to the
    main topic. It includes several supporting details and/or
    examples.

  \item Mechanics - No grammatical, spelling or punctuation errors.

\end{enumerate}

\subsection{Final Presentation - 40 pts}

The final presentation will be a quick, informal 10 minute presentation
given by a minimum of 2 members in your group. Since it is informal
there is no need to dress up. The recommendation is usually 1
minute per slide thus you're looking at perhaps 8 slides for 8 minutes
and then 2 minutes of Q\&A. Your presentation should mirror your final
report. A guide of slides has been created below note however that
this is just a guide and your slides might be different. {\bf DO NOT
  PUT YOUR MATLAB CODE IN THE PRESENTATION}

\begin{enumerate}[1]
\setlength\itemsep{-2pt}

  \item Title Slide

  \item Description of Topic (1)
    
  \item Description of Topic (2)

  \item Mathematical Model (1)

  \item Mathematical Model (2)

  \item Explanation of Results (1)

  \item Explanation of Results (2)

  \item Explanation of Results (3)

  \item Final Slide - Open the Floor for Questions
   
\end{enumerate}

\subsection{\bf List of Projects}

\subsubsection{The Pendulum Problem}

Assume for the moment you have a standard
pendulum with one degree of freedom (the angle of the pendulum). It
is possible to write the equations of motion of a pendulum using the
equation below.

\begin{equation}
  m L^2 \ddot{\theta} = -m g L sin(\theta) - c \dot{\theta} + T(t)
\end{equation}

Your task is to simulate the system above using Euler's Method and
plot the angle and angular velocity as a function of
time. Furthermore, you must linearize the system using forward
finite differencing and midpoint finite differencing assuming an
equilibrium point of 0 and $\pi$. With this linear model compute the
eigenvalues of the system 
and comment on the stability of the system. Furthermore, using this
linearized model you must simulate all three models and 
compare the results of each model and comment on your
results. Finally, assume you have a sensor that can measure $\theta$
and $\dot{\theta}$. Create a controller T(t) that will stabilize the
system at the unstable equilibrium point 

\subsubsection{Output Error Method}

The Output Error Method is a way to solve for unknowns where there is
no analytical solution available. For example, let us examine a simple
paper airplane launched of a cliff. The equations of motion are
\begin{equation}\nonumber
\begin{Bmatrix}
\ddot{x} \\ \ddot{z} 
\end{Bmatrix}
= 
\begin{Bmatrix} 0 \\ g \end{Bmatrix} - \frac{1}{2m}\rho\sqrt{\dot{x}^2+\dot{z}^2}S \begin{Bmatrix} \dot{x}C_X
  \\ \dot{z}C_Z \end{Bmatrix}
\end{equation}
I will provide the actual data $z(t)$ and $x(t)$ in a table format. Then assume that
the initial conditions and all parameters are given except $C_X$ and
$C_Z$. Your task is then to compute a multi-dimensional optimization
routine using the Newton-Raphson technique to estimate $C_X$ and
$C_Z$. The Newton-Raphson technique can be written using the formula
below.

\begin{equation}
\vec{v}_{n+1} = \vec{v}_n + [\nabla^2 E_n]^{-1}\vec{\nabla} E_n
\end{equation}

\noindent where E is the error between the tabular data and the
simulated data obtained from numerically integrating the equations of
motion from above and $\vec{v} = [C_X,C_Z]^T$

\subsubsection{Lifting Line Theory}

The lift on a wing can be approximated assuming the wing is split into
N panels. Each panel produces its own lift by creating a horseshoe
vortex at the panel and can be computed using the circulation formula
below. 
\begin{equation}\label{e:kutta}
  \Gamma_n = C_{Ln} c_n V_{n}/2
\end{equation}
where the lift coefficient $C_{Ln}$ at each panel is given by 
\begin{equation}\label{e:liftdrag}
C_{Ln} = C_{L0n} + C_{L\alpha n}\alpha_n
\end{equation}
The accuracy of this model increases as you increase the number of
panels on each wing. There are a few more equations required to
compute everything in question however the main property of these
vortices is that each vortex interacts with each other. Thus, the task
requires the program to solve for all lift coefficients and
circulations ($\Gamma_n$) simultaneously. This requires the use of a
modified simple fixed point iteration formula to converge on the
system. Your team will need to compute the lift for a given wing as a
function of the number of panels and then compute the optimal location in the x-y
plane for another wing to be to increase the lift on the wing to a
maximum. This maxima optimization can be done using a simple grid
search. 

\subsubsection{Aircraft Steady State Controller}

I will provide the derivatives routine for a six-degree of freedom aircraft
required to simulate a fixed wing aircraft. Your task will be to
develop a simple PD controller to stabilize the aircraft. A PD
controller can be written using the equation below where e is the
error between your state and your command.

\begin{equation}
u = K_p e + K_D \dot{e}
\end{equation}

Although simple here, an aircraft has 4 control surfaces thus 4 PD
controllers will need to be created. In addition, aircraft typically
operate on an inner-loop, outer-loop control system. The control
engineer must design an inner-loop control law in addition to the
outer-loop control law. 

\subsubsection{Rope Simulation}

A rope can be approximated using a series of simple spring mass
dampers. That is, it is possible to break a rope into numerous
particles and place a spring/damper system in between each
particle. The equations of motion of one particle are then simply
\begin{equation}
\begin{Bmatrix} \ddot{x}_n \\ \ddot{y}_n \end{Bmatrix} = \begin{Bmatrix} 0
  \\ -g \end{Bmatrix} + \frac{1}{m_n}\begin{Bmatrix} Fx_{n-1} + Fx_{n+1}
  \\ Fz_{n-1} + Fz_{n+1} \end{Bmatrix}
\end{equation}
where the force between particles is equal to
\begin{equation}
\begin{Bmatrix} Fx_{n} \\ Fz_{n} \end{Bmatrix} = \begin{Bmatrix}
  -k(x_{n+1} - x_{n}) - c(\dot{x}_{n+1} - \dot{x}_{n}) \\ -k(z_{n+1} -
  z_{n}) - c(\dot{z}_{n+1} - \dot{z}_{n}) \end{Bmatrix}
\end{equation}
Your task will be to simulate a rope of at least 10 particles fixed at
two ends and create an animation of the simulation.

\subsubsection{Heat Equation}

The flow of heat through a pipe can be simulated using the standard 1D
pipe flow equations as shown below. 
\begin{equation}
\frac{\partial u_n}{\partial t} = \frac{k}{c_p\rho}\frac{\partial^2
  u_n}{\partial x^2}
\end{equation}
In order to simulate this equation you must discretize the pipe into
multiple elements and use finite differencing to compute the second
derivative of x while using Euler's method to compute the time
derivative of u. Your task is then to simulate the flow of heat
through a pipe when the initial temperature of one end is really
hot. In order to plot your results you can take a few snapshots of the
temperature as a function of position along the pipe or you can
animate the temperature as a function of time.

\subsubsection{Simulate a Car}

Cars can be a lot simpler than airplanes and can be approximated as 1-D
vehicles however they need to operate in the vicinity of other
vehicles. Your task is to simulate multiple vehicles traveling down
the highway while not colliding with each other. The caveat is that
each vehicle will probably be traveling at different speeds. Assume
you can measure the distance between each vehicle as well as the
velocity of each vehicle. The simple equation for a 1D ground
vehicle can be approximated using the equation below.
\begin{equation}
\ddot{x} = F(t) - m g \mu 
\end{equation}
You can be as creative as possible when it comes to the control laws
here including forcing all vehicles to travel the same speed or even
changing lanes if you think that is better. A candidate example would
be all cars stopped at a red light and the light changed to green. 

\subsubsection{Create a Unique Video Game}

I'm slightly afraid to put this as a project mostly because of how
open ended it is however I'd like to see how creative some of you can
be. The task for this project is to create a MATLAB video game. The
game must have at least 5 keyboard inputs with a 
physics based background. If you don't want any keyboard inputs I will
accept a video game that is more puzzle oriented and requires mouse
click instead of keyboard inputs. Since this project is so open ended we will
probably have to have a few sessions to decide what your game will
do. {\bf Multiple teams can work on this one provided they develop a
  different game}.

\subsubsection{Large Data Analysis} 

Most of what engineers do is analyze data. The projects above involve
creating a model that is used to draw conclusions about different
systems. However, it is possible to import massive data sets inside
MATLAB and process the data to compute interesting results and draw
conclusions. For this project you have the option of choosing a large
data set and using at least two numerical methods to analyze the
data. Furthermore, make sure to create figures and plots to support
your analysis. {\bf Again, this is another open ended project therefore
multiple teams can work on this one.} I've listed a few large data sets
below.

\begin{enumerate}

\item UltraSkate 2015 - There is a skateboard race that has been held
  for the past 3 years. The task of each racer is to complete as many
  laps on a indy 500 track in Miami in 24 hrs. The world record is
  held by Andrew Andras who completed 283.24 miles in 24 hrs. All
  racers wear a wrist band that logs each lap. The results are
  tabulated at this website here.

  \href{http://jms.racetecresults.com/results.aspx?CId=16370\&RId=121}{UltraSkate
  2015 Results - http://jms.racetecresults.com/results.aspx?CId=16370\&RId=121}

  You can do alot of analysis with this data like plotting laps times
  as a function of time or numerically differentiating the position
  curve to get velocity as a function of time. I have a few friends
  who did this race this year including, Jeff Crowe and Jude
  Breaux. I'd like to see a full analysis of their race as well as a
  few of the top racers. I suggest creating a script that can analyze
  a racer and then you can pretty much analyze everyone in the race. 

\item Weather in Mobile - The website noaa.org has massive tables of
  weather data for practically every city in the US. Since we live in
  Mobile your task is to analyze whatever data you can find on the
  noaa.org site about the weather in Mobile, AL. I did a bit of digging and found this website here
  which links from noaa.org

  \href{http://www.sercc.com/cgi-bin/sercc/cliMAIN.pl?al5483}{Mobile
    Weather - http://www.sercc.com/cgi-bin/sercc/cliMAIN.pl?al5483}

  Some ideas I have for the data is computing a polynomial fit to the
  data or perhaps computing the time period with the largest
  temperature variation. You could also try and plot multiple years if
  you can find the data and create a time average of the data or
  create a fourier series of the cyclical data. 

\item {\bf Feel free to find large data sets of your own and analyze
  them yourself}

\end{enumerate}

\subsubsection{Pick Your Own Project}

This last project is a place holder in case you can come up with
another project of your own. Perhaps you can simulate playing Texas
Holdem or Blackjack which is technically a video game. Still, if you come up
with an idea of your own I'd love to hear it.

