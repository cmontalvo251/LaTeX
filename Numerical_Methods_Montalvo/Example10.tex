\subsection{Example Problems}

\begin{enumerate}

\item Create an empty function that will compute a sine wave of the form

  \begin{equation}
    y = Asin(\omega x)
  \end{equation}
  
  Your function header will be as follows

  \textcolor{blue}{function} y = myfunction(x,w,A)

  where x can be a scalar or a vector, A is the amplitude of your wave
  and w is the frequency of your sine wave. Test your function by
  plotting it from a value of $-\pi$ to $\pi$. Make sure to use enough data
  points such that the figure is smoothly plotted. As always label
  your axes and make it as pretty as possible.

\item Take your function from problem 1 and compute the Taylor Series
  expansion. Find a way to have MATLAB compute the Taylor series at
  any arbitrary point x. Your function header will look like this.

  \textcolor{blue}{function} yest = mytaylor(x,w,A,N)

  where x is the input variable which can be a vector or a scalar, w
  is the frequency of the sine wave, A is the amplitude and N is the 
  order of your Taylor Series expansion. Choose an expansion point of
  zero to make the math simple. Starting with your plot from
  problem 1, add 3 more lines showing a Taylor Series expansion of
  order 1,3 and 5. You should have 4 lines on your graph. The first
  should be the output of myfunction, the other three should be the
  output of your taylor series expansion.

\item Assume I have a NACA 0012 airfoil across a wing. The total lift
  can be given using the equation below

  \beq
  L = 2\int\limits_0^{b/2}L'dy
  \eeq

  where $L'$ is the lift per unit span. Let $L' = \frac{1}{2}\rho V^2 c C_l$
  where $\rho$ is the atmospheric density, 1.225 $kg/m^3$, $c$ is the
  chord 0.3125 m, and V is the velocity, 20 m/s and b is the span of
  the wing, 2.04 m. Let $c_l$ the lift
  coefficient be defined using the equation below where $C_{l0} = 1.0$

  \beq
  C_{l} = C_{l0}\sqrt{1-(2y/b)^2}
  \eeq

  Use the trapezoidal rule to compute the total lift across the
  wing. Plot the lift as a function of span from -b/2 to b/2. 

\item Write a loop that will compute the taylor series expansion for
  exp(x) from -2 to 2. How many orders does it take to get within 10\%
  of the true value? How many orders does it take to get within 1\% of
  the true value. Make plots of exp(x) along with your two fits. Then
  create plots of percent error between your two fits to prove that
  the taylor series expansion converges.  

\item The solution to the integrand below is shown. Write a computer
  code that will use Simpson's 1/3 Rule to compute the integral (dt =
  0.01). When transforming the integrand use (dx = 0.001). Furthermore,
  split the integral at t = 100. Compute the absolute error between
  the analytical solution and the numerical solution.  

\begin{equation}
I = \int\limits_{0}^{\infty} \frac{1}{t^2+1} dt = \pi/2
\end{equation}

\item Using Romberg integration, compute $I_{1,4}$ for the equation
  below starting with the Trapezoidal Rule. Let $n_1$ = 10. Compute
  the error for all estimates ($I_{j,k}$). Note in order to compute
  the error you will need to compute the solution to this equation
  analytically. 

\begin{equation}
  I = \int\limits_{0}^{\pi} (5 + 3~sin~x) dx
\end{equation}

\item Determine the distance traveled for the following data:

\begin{center}
\begin{tabular}{l | l l l l l l l l l l l}
t (minutes) & 0 & 1 & 2 & 3.25 & 4.5 & 6 & 7 & 8 & 9 & 9.5 & 10 \\
\hline
v (m/s) & 0 & 5 & 6 & 5.5 & 7 & 8.5 & 8 & 6 & 7 & 7 & 5 
\end{tabular}
\end{center}

You may use a computer code to solve this or simply solve it by
hand. It is up to you, however you must use the trapezoidal rule.

\item Integrate the following equation by hand. In addition, write a
  computer code to compute the integral with the 
  Standard Reimmann Sum (dx = 0.01,dy = 0.01). Compute the absolute
  error between your analytical solution and your numerical solution.

\begin{equation}
\int\limits_{-1}^1\int\limits_{0}^{2}(x^2-2y^2+xy^3)dx~dy
\end{equation}

  Finally, create of Mesh of the equation above. That is, use the
  mesh() function in MATLAB and generate a plot
  that shows the surface from x = [-1,1] and y = [0,2]. Label all your
  axes and include the figure in your homework.

\item Consider the differential equation below:

  \beq
  \ddot{y} = -2\dot{y} - y
  \eeq

  Euler's method is an iterative method that can be used to solve
  differential equations. The iterative equations are shown below.

  \beq
  \beqn
  \dot{y}_{n+1} = \dot{y}_{n} + (-2\dot{y}_n-y_n)\Delta t \\
  y_{n+1} = y_{n} + \dot{y}_n\Delta t \\
  t_{n+1} = t_{n} + \Delta t
  \eeqn
  \eeq

  Write a MATLAB code that will use Euler's method to compute y until
  t is equal to 10 seconds. The function header is shown below. 

  \textcolor{blue}{function} myEuler(deltat)\\

  Assume deltat is the timestep $\Delta t$. Let t(1) = 0, y(1) = 2 and
  $\dot{y}$(1) = -2; Run the function for smaller and smaller timesteps until your graph
  does not change. Put in your report what timestep you chose and
  why. Obviously include your final graph in your homework assignment.

\end{enumerate}

