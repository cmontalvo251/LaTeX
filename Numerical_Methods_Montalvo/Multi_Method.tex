\subsection{Output Error Method}

\begin{enumerate}

\item{\bf Output Error Method}

Newton-Raphson works great when you have one output and one variable
trying to be determined using a direct correlation. For example, let's
assume I place an aircraft in a wind tunnel and I vary the wind
tunnel speed while changing the angle of attack such that Lift is
equal to Weight. I have a load sensor on board that measures Drag such
that I can plot Drag as a function of velocity. Explicitly I
would have $D = f(V)$. Using this function, I could compute the
drag coefficient using the simple formula. 

\begin{equation}
D_i = \frac{1}2\rho V_i^2 S C_{Di}
\end{equation}

The subscript $i$ is in the formula above because I have multiple
velocity and drag readings. The method defined above is typical for
operators in a wind tunnel. Using the drag as a function of velocity
you can obtain the drag coefficient as a function of velocity. Using
the drag coefficient I could also find the speed for minimum drag
using the standard Newton-Raphson technique.

\begin{equation}
V_{i+1} = V_i - \frac{f'(V_i)}{f''(V_i)}
\end{equation}

The issue is that you can only obtain the drag coefficient and nothing
else using a wind tunnel. The reason for this is that the aircraft in
the wind tunnel is static and you can only measure certain
coefficients at a time. For example, you could measure lift as well
and estimate the lift coefficient but let's assume you want damping
coefficients. To do this you need to use the output error method.
Let's assume instead that I actually fly the airplane and instead of
measuring velocity and drag, I measure time and distance. Thus I have
$x=f(t)$. Here, I want to find the drag coefficient but notice that I
do not have have the drag anymore. Thus, I can't use Newton-Raphson
technique explicitly but with some massaging I can cast the problem
into a better form. First if we assume the aircraft flies in one
dimension I can write the equations of motion of the aircraft to be

\begin{equation}
\ddot{\tilde{x}} = -\frac{1}{2m}\rho \dot{\tilde{x}}^2 S \tilde{C}_D
\end{equation}

Notice, here that suddenly our drag coefficient pops up. What we can
do then is create an initial guess for our drag coefficient, then use
a numerical integration technique to get an estimate for x. Using this
we can actually create an error estimate as a function of time.

\begin{equation}
\tilde{E}(t) = x(t) - \tilde{x}(t)
\end{equation}

from here we can create an average error square estimate such that

\begin{equation}
J = \frac{1}N\sum\limits_{i=1}^N |\tilde{E}(t_i)|^2
\end{equation}

Let's recap all we've done. We now have an error estimate $J$
which is a function of our drag estimate $\tilde{C}_D$. Formally we have
$J=f(\tilde{C}_D)$. Typically J is called a cost function. If we choose
$\tilde{C}_D$ that is the same as the actual value of $C_D$ then we
would have a cost of 0. This is our ideal situation. It might be
tempting to simply use a root finding method but in fact we actually
want to minimize J. Often times we cannot reach a zero cost given
modeling errors, truncation errors, sensor errors, and round off
errors. However we can still estimate $C_D$ using the recursive
algorithm as defined in the Newton-Raphson method.

\begin{equation}
\tilde{C}_{D,i+1} = \tilde{C}_{D,i} - \frac{J'(\tilde{C}_{D,i})}{J''(\tilde{C}_{D,i})}
\end{equation}

Note that the first and second derivatives of the cost function must
be computed numerically. This is a very involved process with multiple
parts from this course. This entire method is called the Output Error
Method.

\item {\bf Example}

This assignment will involve the students splitting into groups to
write codes that perform the Output Error Method. The Output Error
Method is a way to solve for unknowns where there is no analytical
solution available. For example, let us examine the parachute in free
fall. The equations of motion are

\begin{equation}\nonumber
\ddot{z} = -g + \frac{1}{2m}\rho\dot{z}^2SC_D
\end{equation}

This equation can be simulated on a modest computer assuming all
parameters are known. Assume for the moment that the data $z(t)$ has
been given in a table format. Then assume that the initial conditions
and all parameters except $C_D$ are given:
\ \\

\begin{center}\begin{tabular}{l l l l l l}
$z_0$ = 100 m & $\dot{z}_0$ = 0 m/s & g = 9.81 $m/s^2$ & $\rho$ =
    1.225 $kg/m^3$ & S = 1 $m^2$ & m = 10 kg
\end{tabular}\end{center}

The question then is how to solve for the drag coefficient $C_D$. To
do this the output error method is used. 

\begin{enumerate}

\item The method starts with an initial guess $\tilde{C_D}$.

\item This initial guess can be used to obtain $\tilde{z}$.

\item The total error between the computed value $\tilde{z}$ and the
  measured data from the input file $z$ can be computed using the
  equation below.

  \begin{equation}
    E = \frac{1}N\sum\limits_{i=1}^N (z(t_i)-\tilde{z}(t_i))^2
  \end{equation}
  
  The goal then naturally is to perform an optimization technique and
  compute $\tilde{C_D}$ such that the error drops to zero or at least
  a minimum.

 \item To minimize this problem the Newton-Raphson technique is used
   such that

   \begin{equation}
     \tilde{C}_{D,i+1} = \tilde{C}_{D,i} - \frac{E'(\tilde{C}_{D,i})}{E''(\tilde{C}_{D,i})}
   \end{equation}

   Obviously the first and second derivatives will need to be computed numerically.

\end{enumerate}

To solve this problem the students will break into groups of 6 or 7. Each
member of the group will have to complete one of the codes
below. Built in MATLAB functions are allowed unless noted specifically
in the text. Note that the group may opt to work as a team
on certain sections of the code.

\begin{enumerate}

\item function [t,z] = readdata(filename), this function will use the
  dlmread() command to read the text file of data provided and return
  the t vector, z vector and the timestep used to generate the data.

\item function ploteverything(t,z,ztilde), this function will take a
  vector t, z and ztilde and plot everything in a nice pretty graph

\item function zdot = Derivs(t,z,CDtilde), this function will compute the
  derivative of z for a given value of CDtilde. Remember that z will
  be a vector containing $z$ and $\dot{z}$ thus zdot will be a
  vector containing $\dot{z}$ and $\ddot{z}$ 

\item function [ttilde,ztilde] = RK4(CDtilde), this function will take
  a value of the drag coefficient and use the RK4 method to compute
  the height of the parachute as a function of time. Note that you
  need to make sure you use the same timestep as the one used in the
  data set provided. You may not use the ode45() function.

\item function E = compute\_error(CDtilde,z), this function will
  compute the total error between the estimated height ztilde and the
  actual height z.

\item function fprime = Eprime(CDtilde,z), this function will compute
  the first derivative of the error function at the value of
  CDtilde. Remember that you will have to choose the value of
  $\Delta$. 

\item function fdblprime = Edoubleprime(CDtilde,z), this function will
  compute the second derivative of the error function at the value of
  CDtilde. Again you must choose $\Delta$

\item function main(), this function will perform the optimization. I
  have written this part of the code for you.

\begin{framed}

function main()

filename = 'drc\_data.txt';

[t,z] = readdata(filename);

CDtilde = 0.1;

Ei = compute\_error(CDtilde,z);

while abs(Ei) $>$ 1e-2

~~~~~~CDtilde = CDtilde - Eprime(CDtilde,z)/Edoubleprime(CDtilde,z);

~~~~~~Ei = compute\_error(CDtilde,z);

end

[ttilde,ztilde] = RK4(CDtilde);

ploteverything(t,z,ztilde);

\end{framed}

\end{enumerate}

The winning team must solve for $C_D$ and email the instructor a graph
showing z and ztilde on top of each other. The first team to correctly
email the instructor with the value of $C_D$ and the correct graph
will get a prize. Ready go!

\end{enumerate}


